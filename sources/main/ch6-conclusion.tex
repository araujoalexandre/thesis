%%%%%%%%%%%%%%%%%%%%%%%%%%%%%%%%%%%%%%%%%%%%%%%%%%%%%%%%%%%%%%%%%%%%%%%%%%%%%%%%
\chapter{Conclusion}
\label{chapter:ch6-conclusion}
%%%%%%%%%%%%%%%%%%%%%%%%%%%%%%%%%%%%%%%%%%%%%%%%%%%%%%%%%%%%%%%%%%%%%%%%%%%%%%%%
\localtoc

%%%%%%%%%%%%%%%%%%%%%%%%%%%%%%%%%%%%%%%%%%%%%%%%%%%%%%%%%%%%%%%%%%%%%%%%%%%%%%%%
\section{Summary of the Contributions}
\label{section:ch6-summary_of_the_contributions}
%%%%%%%%%%%%%%%%%%%%%%%%%%%%%%%%%%%%%%%%%%%%%%%%%%%%%%%%%%%%%%%%%%%%%%%%%%%%%%%%


State-of-the-art in a variety of domains, deep neural networks exhibit important limitations.
Indeed, current neural networks tend to be very large in terms of their number of parameters which make them difficult to train and to deploy in real-world applications.
Furthermore, they exhibit instability to small perturbations of their inputs which lead to adversarial attacks. 

In this thesis, we have used structured matrices from the Toeplitz family to make contributions to the field of deep learning.
Our contributions are twofold.
First, we studied deep diagonal-circulant neural networks, which are deep neural networks in which weight matrices are the product of diagonal and circulant ones.
Using diagonal and circulant matrices instead of dense ones allows for an important reduction in the number of parameters which make them more efficient and cost-effective.
In addition to being more compact than fully connected neural networks, diagonal-circulant neural networks have a high expressivity that makes them useful for numerous use cases.
In order to characterize the expressive power of diagonal-circulant neural networks, we build upon the work of~\citet{huhtanen2015factoring} which states that any matrix can be decomposed into a product of alternating diagonal and circulant matrices.
Based on this result, we have successfully demonstrated that neural networks with diagonal and circulant matrices are \emph{universal approximators} and characterized their expressive power with respect to their depth.
We also demonstrated the effectiveness of this class of compact neural networks to video classification with a real-world dataset.

% Secondly, we study the properties of doubly-block Toeplitz matrices which are the matrix equivalent of the convolution operation. 
% Using the properties of this type of structured matrix allows us to develop a new regularization scheme of convolutional neural networks that improve their robustness against adversarial attacks.
% This regularization scheme reduces a bound of the Lipschitz constant of the neural networks thus making it less sensitive to small perturbations of its input.

% From the properties of doubly-block Toeplitz matrices and a Fourier representation introduced by~\citet{grenander1958toeplitz}, we demonstrated an upper-bound on the singular values of convolution layers.
% In order to use this upper-bound in a large-scale setting, we introduce the PolyGrid algorithm (see \Cref{algorithm:ch5-polygrid}) which efficiently and accurately computes an approximation of this upper-bound.
% Finally, we demonstrated that employing this bound as a regularizer improved the generalization and the robustness of neural networks trained with the adversarial training scheme.
%
% ---

Secondly, we studied the properties of doubly-block Toeplitz matrices which are equivalent to the convolution operation. 
Using the properties of this type of structured matrix and a Fourier representation introduced by~\citet{grenander1958toeplitz}, we devised an upper-bound on the singular values of convolution layers leading to a new regularization scheme that improves the robustness of neural networks against adversarial attacks.
In order to use this upper-bound in a large-scale setting, we introduced the PolyGrid algorithm (see \Cref{algorithm:ch5-polygrid}) which efficiently and accurately computes an approximation of this upper-bound.
% Finally, we demonstrated that employing this bound as a regularizer reduces the sensitivity of the network to small perturbations, thus, improving the generalization and the robustness of the network.



% This regularization scheme reduces a bound of the Lipschitz constant of the neural networks thus making it less sensitive to small perturbations of its input.



% Secondly, we proposed a contribution based on the observation that the convolution operation of convolution layers can be interpreted as a matrix-multiplication where the matrix is the concatenation of multiple doubly-block Toeplitz matrices.
%
% a new regularization scheme of convolutional neural networks that improve their robustness against adversarial attacks.
%
% This contribution is based 




%%%%%%%%%%%%%%%%%%%%%%%%%%%%%%%%%%%%%%%%%%%%%%%%%%%%%%%%%%%%%%%%%%%%%%%%%%%%%%%%
\section{Perspectives and Future Works}
\label{section:ch6-perspectives_and_future_works}
%%%%%%%%%%%%%%%%%%%%%%%%%%%%%%%%%%%%%%%%%%%%%%%%%%%%%%%%%%%%%%%%%%%%%%%%%%%%%%%%

\subsection{Designing Compact Transformers for Natural Language Processing}


In order to improve upon our work on compact neural networks, one idea follows naturally.
The race towards larger convolutional neural networks seemed to have slowed following the work of \citet{tan2019efficientnet} which devised compact state-of-the-art neural networks for image recognition.
However, other types of architecture, \eg, \emph{Transformers} which rely heavily on dense matrices, have seen their number of parameters exploding in recent years.
The latest model which was designed by \citet{fedus2021switch} has 1 trillion parameters, 5.7 times than the second largest, proposed by \citet{brown2020language}, which had 175 billion parameters.

In~\Cref{chapter:ch4-diagonal_circulant_neural_network} and Appendix~\ref{appendix:ap2-diagonal_circulant_neural_networks_for_video_classification}, we have used the diagonal-circulant decomposition for compressing embedded layers in the context of video classification.
This decomposition could also be used to compress attention layers of Transformers networks~\cite{vaswani2017attention} where the attention layer is described as follows:
\begin{equation}
  \text{Attention}(\Qmat, \Kmat, \Vmat) = \text{softmax} \left( \frac{\Qmat \Kmat^\top}{\sqrt{d_k}} \right) \Vmat \enspace,
\end{equation}
where $\Qmat, \Kmat$ and $\Vmat$ are dense matrices.
Taking this layer as a building block leads to large neural networks as demonstrated by the GPT-3 architecture with 96 attention layers and 175 billion parameters.
Although, the diagonal-circulant decomposition could successfully reduce the number of parameters of attention layers, it may have limited impact on \emph{multi-head attention layers} which are a concatenation of small attention layers due to the reduced dimension of each matrix.
% , the total number of parameters is similar to that of single-head attention with full dimensionality.
% \begin{align}
%   \text{MultiHead}(\Qmat, \Kmat, \Vmat) &= \text{concat}(\Hmat^{(1)}, \dots, \Hmat^{(h)}) \Wmat_O \\
%   \text{where } \Hmat^{(i)} &= \text{Attention}(\Qmat \Wmat^{(i)}_Q, \Kmat \Wmat^{(i)}_K, \Vmat \Wmat^{(i)}_V) 
% \end{align}
% where $\Wmat^{(i)}_Q$, $\Wmat^{(i)}_K$ and $\Wmat^{(i)}_V$ are projection matrices.
% Due to the reduced dimension of each matrix of the multi-head attention layer, the total number of parameters is similar to that of single-head attention with full dimensionality.



% ---------------
%
% \todotext{update here}
%
% The regularization of the Lipschitz constant of neural networks has been a growing interest in the training of neural networks.
% Indeed, Lipschitz regularization goes beyond robustness of neural networks and can be used in a wide variety of settings, for example, to stabilize the training of Generative Adversarial Networks and invertible networks, or to improve generalization capabilities of classifiers.
%
%
% Currently, 
%
% It is important to note that the result achieved by combining adversarial training and Lipschitz regularization allow an improvement of 
%
% and devising techniques to better control the expressiveness of neural network is  



\subsection{Regularization on the Condition Number of Convolution Layers}


In \Cref{chapter:ch5-lipschitz_bound}, we have proposed an upper-bound on the largest singular value of convolution layers which allow us to regularize the Lipschitz constant of the network thus improving the robustness.
% Indeed, a Lipschitz regularization during training combined with a local smoothing technique such as adversarial training allow an important gain in robustness. 
However, an important reduction of the Lipschitz constant seems to prevent the network from learning correctly. 
Indeed, as demonstrated by the work of \cite{zhang2019theoretically}, accuracy and robustness are actually at odds, meaning that improving the robustness (\ie in our case, reducing the expressivity) hurts the training and the natural accuracy of the network. 
In our experiments, the phenomenon called \emph{rank collapse} \cite{saxe2014exact} where the rank of the weights matrices tend to decrease during training combined with a strong Lipschitz regularization would prevent convergence.
% as the expressivity is concentrated in the singular vector associated with the largest singular values.
An interesting solution would be to regularize the largest singular value and promoting the smallest in order to enforce orthogonality.
% regularize the condition number of the weight matrices (ratio between the highest and lowest singular values).
The following bound on the condition number of general matrices \citet{guggenheimer1995simple} could be studied:
\begin{equation}
  \kappa(\Wmat) \le \frac{2}{\abs{\det(\Wmat)}} \left(\frac{\norm{\Amat}_\fro}{\sqrt{r}}\right)^r
\end{equation}
where $r$ is the rank of $\Wmat$.
As such, using the bound as a regularizer will enforce the orthogonality, just like a layer normalization.
Therefore, we could design some heuristic regularizers to encourage a smaller $\left(\frac{\norm{\Wmat}_\fro}{\sqrt{r}}\right)^r$ and larger $\abs{\det(\Wmat)}$ separately, as given by the following objective function: 
\begin{align} 
  \min_{\Omega} \Ebb_{\xvec, y \sim \Dset} \left[ L(N_\Omega(\xvec), y)  + C_1 \sum_{i=1}^\depth \norm{\Wmat^{(i)}}_\fro + C_2 \sum_{i=1}^{\depth} \abs{\det(\Wmat^{(i)})} \right]
\end{align}
where the determinant of doubly-block Toeplitz matrices under some assumption could be expressed with the Szeg\"{o} Theorem \cite{szego1915grenzwertsatz} and can be approximated with the help of \emph{Random Matrix Theory} \cite{basor2017asymptotics}.

% \subsection{Stronger Provably Adversarial Robustness with Noise Injection and Lipschitz Regularization}
%
% xxx
%
% In Appendix xxx, we have shown that injecting random noise from the Exponential family both during training and inference phases provably defend against adversarial attacks.
% In our experimental  
%






\subsection{Going Beyond the Lipschitz Constant} 

Finally, in order to better understand the behavior of neural networks and the transformation they perform, it would be interesting to go beyond the Lipschitz constant and consider their full spectrum.
Indeed, the spectrum of a linear map is a set that contains the eigenvalues and can be seen as a description of the properties and behavior of the operator.
For example, for a linear operator $\Lmat: \Xset \rightarrow \Xset$, the spectrum gives precise information on the solvability of the following linear equation
\begin{equation}
  \lambda \xvec - \Lmat \xvec = \yvec
\end{equation}
It is natural to ask if we could define a spectrum that equivalently gives information on the following nonlinear equation
\begin{equation}
  \lambda \xvec - \Fmat(\xvec) = \yvec \enspace.
\end{equation}

In this line of research, \citet{kachurovskii1969regular} have defined a spectrum for nonlinear continuous Lipschitz operators which share important properties with the spectrum of linear operators.
More precisely, let $\Fmat: \Xset \rightarrow \Xset$ be a nonlinear continuous Lipschitz map, the \emph{Kachurovskij spectrum} of $\Fmat$ is given by
\begin{equation}
  \sigma(\Fmat) \triangleq \left\{ \lambda \in \Cbb \mid \lambda \Imat - f \text{ is not a lipeomorphism} \right\} 
\end{equation} 
where a nonlinear Lipschitz continuous operator is a \emph{lipeomorphism} if its inverse is also nonlinear Lipschitz continuous.
We can also define the complement of the spectrum, \ie, the \emph{Kachurovskij resolvent set} as follows: 
\begin{equation}
  \mu(\Fmat) \triangleq \Cbb \setminus \mu(\Fmat) 
\end{equation}
The resolvent set can be seen as the set of complex numbers for which the operator is \emph{well behaved}.
The Kachurovskij spectrum is a compact subset of the complex plane but may be empty.
Kachurovskij have also shown that the emptiness of this spectrum can be prevented if we restrict ourselves to nonlinear continuous Lipschitz operators that admit a Fréchet-derivative $\Fmat'(\xvec_0)$ at some point $\xvec_0 \in \Xset$.
In this case, the Kachurovskij spectrum share all the properties of a linear operator which are: closed, compact, bounded and non-empty. 

Neural networks are differentiable nonlinear Lipschitz continuous functions, therefore the study of the Kachurovskij spectrum could give important insight on their stability, invertibility and robustness to adversarial examples.



%%%%%%%%%%%%%%%%%%%%%%%%%%%%%%%%%%%%%%%%%%%%%%%%%%%%%%%%%%%%%%%%%%%%%%%%%%%%%%%%
\section{Discussion}
\label{section:ch6-discussion}
%%%%%%%%%%%%%%%%%%%%%%%%%%%%%%%%%%%%%%%%%%%%%%%%%%%%%%%%%%%%%%%%%%%%%%%%%%%%%%%%


% The global regularization of neural network during training combined with a local smoothing technique such as adversarial training allow an important gain in robustness. 
% However, our contribution also highlight some important difficulties in training neural networks.


Although our contributions offer concrete techniques for building compact and reliable neural networks, they also highlight some important difficulties in them.
First, if we discard techniques such as pruning or quantization for building compact neural networks due to the necessity of training a large neural network prior to compression, designing parameters-efficient neural networks that are compact \emph{by design} requires rethinking the whole architecture.
For computer vision tasks, the convolution operation is a compact and powerful transform, however, we still haven't found such equivalent transforms for other use cases.
Although the multi-head attention layer is more efficient than the attention layer for NLP tasks, using this type of transform in a neural network is still very parameter-hungry as demonstrated by the recent state-of-the-art for language models \cite{brown2020language}.


Secondly, defense techniques against adversarial attacks have shown great improvements in the last few years.
However, with current state-of-the-art techniques, it is still difficult to reach an accuracy higher than 60\% on CIFAR10 (which is considered a small dataset) and the accuracy decreases further on datasets with a larger dimensionality.
Consequently, building robust neural networks still remain very much an open question.
We believe that further breakthroughs in this area will come as a by-product on research on \emph{understanding neural networks}. 
Accordingly, we hope that our contribution to the understanding of diagonal-circulant and convolution neural networks is a small step in this direction.



% while new techniques successfully improve the robustness of neural networks, it has been shown that accuracà and robustness are actually at odds \cite{zhang2019theoretically}, 










