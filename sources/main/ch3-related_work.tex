%%%%%%%%%%%%%%%%%%%%%%%%%%%%%%%%%%%%%%%%%%%%%%%%%%%%%%%%%%%%%%%%%%%%%%%%%%%%%%%%
\chapter{Related Work}
\label{chapter:ch3-related_work}
%%%%%%%%%%%%%%%%%%%%%%%%%%%%%%%%%%%%%%%%%%%%%%%%%%%%%%%%%%%%%%%%%%%%%%%%%%%%%%%%
\localtoc
\vspace{1cm}


% This thesis makes contributions on building compact and robust neural networks with help from Toeplitz matrix theory.
This chapter aims at presenting an overview of the state-of-the-art related to our contributions.
First, we present the current methods to build compact neural networks.  
Given that the scope of these techniques is large, we choose to focus mainly on works which use tools from linear algebra and more particularly structured matrices.
The second part of this chapter presents current methods for regularizing the Lipschitz constant of neural networks with the ail of improving their robustness. 
% Our second contribution focuses on building robust neural networks by regularizing the Lipschitz constant of neural networks.
% Hence, we present in a second part recent works on regularizing the Lipschitz constant of neural networks.


% The second part of this chapter presents current methods for regularizing this constant that aim at improving the robustness of neural networks.
% We omit methods that are orthogonal to our approach for clarity and conciseness.


%%%%%%%%%%%%%%%%%%%%%%%%%%%%%%%%%%%%%%%%%%%%%%%%%%%%%%%%%%%%%%%%%%%%%%%%%%%%%%%%
\section{Related Work on Compact Neural Networks}
\label{section:ch3-related_work_on_compact_neural_networks}
%%%%%%%%%%%%%%%%%%%%%%%%%%%%%%%%%%%%%%%%%%%%%%%%%%%%%%%%%%%%%%%%%%%%%%%%%%%%%%%%


% In this chapter, we review the literature on techniques for building compact neural networks.
% First of all, we present, in detail, related work which uses tools from linear algebra and structured matrices. 
% Finally, we present in a more concise way concurrent techniques like using specific memory representation or using neural architecture search.
% These techniques are mostly orthogonal to our contributions.


% We have seen in the Introduction (\Cref{chapter:ch2-introduction}) and Background (\Cref{chapter:ch2-background}) that neural networks tend to be over-parametrized which lead to difficult and expensive training and overfitting.


% In this section, we review the literature for building compact neural networks.
% As seen in the Introduction (\Cref{chapter:ch1-introduction}) and Background (\Cref{chapter:ch2-background}), scaling up networks can lead to an increase in accuracy.
% Researchers have demonstrated that increasing the width of shallow neural networks increased their performance~\cite{howard2017mobilenets,sandler2018mobilenetv2,tan2019mnasnet,zagoruyko2016wide} due to their capacity to capture more fine-grained features.
% Increasing depth is a common and effective way to scale neural networks and many deep architectures have been proposed~\cite{he2016deep, huang2016deep, szegedy2016rethinking,szegedy2017inception,xiao2018dynamical}. 
% The intuition is that deep neural network can capture richer and more complex features.


As seen in the Introduction (\Cref{chapter:ch1-introduction}), scaling up networks can lead to better performance.
Increasing the width of shallow neural networks can increase their performance~\cite{howard2017mobilenets,sandler2018mobilenetv2,tan2019mnasnet,zagoruyko2016wide} due to their capacity to capture more fine-grained features.
As well, increasing depth is a common and effective to capture richer and more complex features and increase performance, many deep architectures have been proposed~\cite{he2016deep,huang2016deep, szegedy2016rethinking,szegedy2017inception,xiao2018dynamical}.
However, large neural networks lead to difficult and expensive training and overfitting and after observing that a lot of parameters in large neural networks were redundant~\cite{dai2018compressing,frankle2018lottery}, an important question arises: \emph{do neural networks needs to be over-parameterized? And if not, how to build accurate and compact neural networks?} 

Numerous directions have been investigated to build compact and cost-effective neural networks, for example \citet{gupta2015deep,micikevicius2018mixed} have proposed to represent weights with limited numerical precision to reduce training time and memory requirements.
They used half-precision floating-point format instead of single-precision floating-point format which uses 32 bits of computer memory.
In the same direction, \citet{courbariaux2015binaryconnect} have proposed a method to train neural networks with binary weights.

An important idea in model compression proposed by~\citet{bucilua2006model}, is based on the observation that the model used for training is not required to be the same as the one used for inference, indeed, compressed models after training can be deployed on smartphones or IoT devices.
Based on this idea, multiple post-processing techniques have been developed: a quantization procedure which consists in converting the weights into a binary or integer formats \emph{after} the training phase~\cite{mellempudi2017ternary,rastegariECCV16}, pruning techniques~\cite{dai2018compressing,han2015deep,lin2017runtime} or sparsity regularizers~\cite{collins2014memory,dai2018compressing,liu2015sparse} which consists in removing redundant weights after training and taking advantage of the sparse structure of the weights matrices.
Sparse neural networks have also been extensively studied since the \emph{Lottery Ticket Hypothesis} proposed by \citet{frankle2018lottery}.
This hypothesis states that it exists a sparse subnetwork of dense neural network that when trained in isolation can match the test accuracy of the original dense network after training for at most the same number of iterations. 
This hypothesis led to a series of works on sparse neural networks \cite{zhou2019deconstructing,malach2019proving,evci2020rigging}.
Moreover, \citet{ba2014deep} have empirically demonstrated that shallow neural networks can learn the complex functions previously learned by another deep neural networks.
This result led \citet{hinton2015distilling} to propose a technique called \emph{model distillation} which consists in training a large complex model using all the available data and resources to be as accurate as possible, then a smaller and more compact model is trained to approximate the first model.
Although interesting for deployment purposes, this approach still requires to train one large network and one shallow, which entails a significant training cost.
More recently, \citet{zoph2018learning,real2019regularized} have designed algorithms that automatically tune the width and depth of neural network architectures to obtain the best trade-off between compactness and accuracy.
With this approach, \citet{tan2019efficientnet} found a new compound scaling method that uniformly scales network width and depth leading to efficient and compact architectures.

Finally, in the context of deep learning, compact representations have gained much attention over the past years as a way to compress models or to reduce memory requirements.
In this thesis, we focus on building compact neural networks with structured matrices.
Hereafter, we present a comprehensive overview of the existing techniques in this line of research.



% --> Compact Neural Networks Architecture

% Without consideration of specific memory representations, data structures or structured linear layers, it is still possible to design compact neural networks.
%
% However, researchers have demonstrated that increasing the width of shallow neural networks increased their performance~\cite{howard2017mobilenets,sandler2018mobilenetv2,tan2019mnasnet,zagoruyko2016wide} due to their capacity to capture more fine-grained features.
% Finally, depth is a common and effective way to scale neural networks and many deep architectures have been proposed~\cite{he2016deep, huang2016deep, szegedy2016rethinking,szegedy2017inception,xiao2018dynamical}. 
% The intuition is that deep neural network can capture richer and more complex features.
%
% After observing that large and deep neural networks outperformed shallow ones \cite{huang2019gpipe,brown2020language} and the observation that a lot of parameters in large neural networks were redundant~\cite{dai2018compressing,frankle2018lottery}, an important question arises: \emph{do neural networks needs to be large \ie, deep and wide, and if not, which architecture provides the best accuracy?} 

% \citet{ba2014deep} tried to answer this question and empirically demonstrated that shallow neural networks can learn the complex functions previously learned by another neural network. 
% This observation was then leveraged by~\citet{hinton2015distilling} for compressing trained neural networks.
% Their technique, called \emph{model distillation}, consists to train a large complex model using all the available data and resources to be as accurate as possible, then a smaller and more compact model is trained to approximate the first model.
% Although, this approach can be interesting for deployment purposes, it is still required to train one large network and one shallow, which entails a significant training cost.
%
% Instead of compressing the model after the training step, researchers still tried to design architectures that are compact by nature but finding the best trade-off between depth, width and performance has proved to be a tedious work.
% In order to scale the search, recent works have devised algorithms to automatically find the best architecture for a specific use case.
% \citet{zoph2018learning,real2019regularized} have tried to tune the wide and depth of neural network architectures to obtain the best trade-off between efficiency and accuracy but these methods required a lot of manual tuning.
% More recently, with a similar method, \citet{tan2019efficientnet} found a new compound scaling method to uniformly scales network width, depth, and resolution leading to groundbreaking result in terms of efficiency and accuracy.




%%%%%%%%%%%%%%%%%%%%%%%%%%%%%%%%%%%%%%%%%%%%%%%%%%%%%%%%%%%%%%%%%%%%%%%%%%%%%%%%
\subsection{Building Compact Neural Networks with Structured Matrices}
\label{subsection:ch3-building_compact_neural_networks_with_structured_matrices}
%%%%%%%%%%%%%%%%%%%%%%%%%%%%%%%%%%%%%%%%%%%%%%%%%%%%%%%%%%%%%%%%%%%%%%%%%%%%%%%%

% An effective method to build compact neural networks is to constrain the hypothesis space in which the learning algorithm ``chooses'' the predictor.
% As seen in \Cref{chapter:ch2-background}, this constraint is called \emph{inductive bias}. 

% Another way of constraining the weight representation and reduce the memory requirement of neural networks is to impose a \emph{structure} on weight matrices. 

% The idea of building compact neural networks with structured matrices consists of replacing the weight matrices $\Wmat^{(i)}$ with \emph{structured matrices}.

% \todotext{this paragraph makes no sense $\downarrow$}



% A structured matrix is a $n \times n$ matrix whose entries have a formulaic relationship, allowing the matrix to be represented with fewer than $n^2$ parameters.
% The formulaic relationship between entries is an important feature to consider, for example, a sparse matrix has fewer than $n^2$ parameters but does not have a clear relationship between its entries.


An effective method to build compact neural networks is to constrain the hypothesis space on \emph{structured neural networks}. 
The idea of structured neural networks consists of replacing the dense weight matrices with \emph{structured matrices}.
A structured matrix is a $n \times n$ matrix which can be represented with fewer than $n^2$ parameters and where the entries are distributed along specific rules.

% %%%%%%%%%%%%%%%%%%%%%%%%%%%%%%%%%%%%%%%%%%%%%%%%%%%%%%%%%%%%%%%%%%%%%%%%%%%%%%%%
% \subsubsection{Neural networks with Specific Structured Linear Maps}
% \label{subsubsection:ch3-neural_networks_with_specific_structured_linear_maps}
% %%%%%%%%%%%%%%%%%%%%%%%%%%%%%%%%%%%%%%%%%%%%%%%%%%%%%%%%%%%%%%%%%%%%%%%%%%%%%%%%


\drawstar

A classic example of structured matrices are \emph{low-rank} matrices which result from the product of two rectangular matrices.
Let $\Umat \in \Rbb^{n \times r}$ and $\Vmat \in \Rbb^{r \times m}$ with $r \ll \min(m, n)$ then the matrix $\tilde{\Wmat} = \Umat \Vmat$ is low-rank of rank $r$.  
By representing the matrix $\tilde{\Wmat}$ with the low-rank decomposition, we can reduce storage from $mn$ parameters to $(mr + nr)$ parameters and accelerate the matrix-vector product from $\mathcal{O}(mn)$ to $\mathcal{O}(mr + rn)$.
\citet{sainath2013lowrank} were among the first to use low-rank matrices in deep learning contexts followed by the work of~\citet{jaderberg2014speeding,yu2017compressing}.
To enforce the low-rank constraint, reduce storage and computation time, they learned the low-rank decomposition directly, \ie the matrices $\Umat$ and $\Vmat$ instead of $\tilde{\Wmat}$ leading to a neural network layer defined as follows:
\begin{equation}
  \phi^\rho_{\Umat, \Vmat, \bvec} (\xvec) = \rho\left( \Umat \Vmat \xvec + \bvec \right) .
\end{equation}
% By replacing the dense weight matrices $\Wmat^{(i)}$ with the product of two rectangular matrices and learning them instead of the dense matrices, they \emph{impose} on the learning procedure the low-rank constraint.
% We can remark that the matrix resulting from the product of $\Umat^\top \Vmat$ has $m^2$ unique values but can still be represented with only $2nm$ values in memory.
The rank of the low-rank weight matrices becomes a hyper-parameter and governs the trade-off between expressivity (\ie constraint on the hypothesis space) and compactness of the final predictor. 
In the same vein, \citet{novikov2015tensorizing} have proposed to replace the dense weight matrices of fully connected layers by the \emph{Tensor Train decomposition} allowing an important reduction of the number of parameters while preserving the expressive power of the layers.


% Although, less expressive, low-rank matrices offer several advantages: they can be represented in computer memory by their decomposition  
%
% For a neural network $N_\Omega$ parameterized by the set of weight matrices and bias vector $\Omega = \left\{ \left( \Wmat^{(i)}, \bvec^{(i)} \right) \right\}_{i \in [d]}$, they proposed to constraint 
%
% replaced the dense weight matrices $\Wmat^{(i)}$ with a product of two rectangular matrices $\Umat, \Vmat \in \Rbb^{n \times m}$ with $n < m$ which led to neural network layer defined as follows:

% In linear algebra, many structures have been discovered and studied~\cite{pan2001structured}. 
% for example, circulant matrices have been used to efficiently solve linear systems~\cite{golub1996matrix} and years later 
% This effectively replaces the complex transform modeled by the fully connected layer by a simple dimensionality reduction.
% Indeed, the last 3 fully connected layers of the AlexNet~\cite{krizhevsky2012imagenet} architecture use 58M out of the 62M total trainable parameters ($> 90\%$ of the total number of parameters).

% comes from the observation that a neural network layer can be seen as dimensionality reduction layer.

\drawstar

A variety of structured matrices have been used to build compact neural networks.
For example, \citet{cheng2015exploration} had the idea to replace the weight matrix of a fully connected layer by the product of a circulant and diagonal matrices where the circulant matrix is learned by a gradient-based optimization algorithm and the diagonal matrix entries are sampled at random in $\{-1, 1\}$ leading the layer to be parameterized by only $n$ values instead of $n^2$ and the matrix-vector product to be efficiently computed with the Fast Fourier Transform and the \Cref{algorithm:ch2-matrix_vector_product_circulant_matrix}.
Despite the reduction of expressivity, their experiments demonstrated good accuracy using only a fraction of the original size of the network (90\% reduction).
The idea of using circulant matrices and other types of structured matrices in neural networks is motivated by the celebrated \citeauthor{johnson1984extensions} lemma which states that a set of points in a high-dimensional space can be embedded into a space of much lower dimension with near preservation of distance between the points.
This result can be formalized as follows:
\begin{lemma}[\citet{johnson1984extensions}]
  Let $\mathcal{X} \subset \Rbb^n$ be a set of $m$ vectors, then for any $0 < \epsilon < \frac{1}{2}$, there exists a map $f: \mathcal{X} \rightarrow \Rbb^k$ for some $k = \mathcal{O}(\epsilon^{-2} \log m)$ such that
  \begin{equation}
    \forall \xvec, \yvec \in \mathcal{X}, \quad (1 - \epsilon) \norm{\xvec - \yvec}_2^2 \leq \norm{f(\xvec) - f(\yvec)}_2^2 \leq (1 - \epsilon) \norm{\xvec - \yvec}_2^2 
  \end{equation}
\end{lemma}
\noindent
More practically, \citet{hinrichs2011johnson} have shown that the \DC transform $\xvec \mapsto \Dmat \Cmat \xvec$, where the matrix $\Dmat$ is a diagonal matrix with entries sampled from $\{1, -1\}$ and $\Cmat$ is a circulant matrix based on a sequence of independent identically distributed random variables respect the \citeauthor{johnson1984extensions} lemma.
Although \citet{cheng2015exploration} used \emph{learned} circulant matrices, therefore relaxing the theoretical guarantee of \citet{hinrichs2011johnson} they showed good empirical results. 

Several transform with structured matrices have been proposed.
The Fastfood transform~\cite{le2013fastfood}, which was originally used for approximating kernel expansions, was later used in neural networks by~\citet{yang2015deep} leading to the Deep Fried Convnets architecture.
The authors have replaced dense matrices of fully connected layers with adaptive structured matrices of the form: $\Smat\Hmat\Gmat\mathbf{\Pi}\Hmat\Bmat$ where $\Smat$, $\Gmat$, and $\Bmat$ are adaptive diagonal matrices, $\mathbf{\Pi}$ is a random permutation matrix, and $\Hmat$ is the Walsh-Hadamard matrix.
Later, the \emph{Structured Spinners} transform of the form: $\Hmat\Dmat^{(3)} \Hmat\Dmat^{(2)} \Hmat\Dmat^{(1)}$, where $\Hmat$ is the Walsh-Hadamard matrix, and $\Dmat^{(i)}$ for $i \in {1, 2, 3}$ is a random $\pm1$-diagonal matrix, was originally proposed by~\citet{andoni2015practical} and used in deep learning settings by~\citet{bojarski2017structured}.

\citet{moczulski2016acdc} build upon the work of~\citet{cheng2015exploration} and \citet{huhtanen2015factoring} and introduced two \emph{Structured Efficient Linear Layers} (SELL) based on the Fourier and cosine transform.
First, by observing that the \DC transform cannot express an arbitrary linear operator they propose to apply the result of \citet{huhtanen2015factoring} which states that almost all matrices can be decomposed as a product of \DC transform.
More formally, 
\begin{theorem}[Reformulation from \citet{huhtanen2015factoring}]
  For every matrix $\Mmat \in \Cnn$, for any $\epsilon > 0$, there exists a sequence of matrices $\Amat^{(1)} \ldots \Amat^{(2n-1)}$ where $\Amat^{(i)}$ is a circulant matrix if $i$ is odd, and a diagonal matrix otherwise, such that $\norm{\Amat^{(1)} \ldots \Amat^{(2n-1)} - \Mmat} < \epsilon$.
  \label{theorem:ch3-huhtanen}
\end{theorem}
\noindent
Based on this result, they proposed the following structured layer:
\begin{equation}
  \Phi^\rho_{\Dmat, \Cmat, \bvec} (\xvec) = \rho \left( \left(\prod_{i = 1}^{k} \Dmat^{(i)} \Cmat^{(i)} \right) \xvec + \bvec \right)
\end{equation}
where $\Dmat$ and $\Cmat$ are sequences of $k$ diagonal and circulant matrices respectively.
This structured is therefore parameterized by $n(2k+1)$ values and the value $k$ becomes an hyper-parameter controlling the trade-off between compactness and expressivity. 
The linear layer can be efficiently computed with the fast Fourier transform and the diagonalization of the circulant matrix. 
Recall that if the circulant matrix $\Cmat^{(i)} =  \circulant\left(\cvec^{(i)}\right)$ where $\cvec^{(i)}$ is the vector characteristic then we can express the circulant matrix $\Cmat^{(i)}$ as follows: $\Cmat^{(i)} = \frac{1}{n} \Umat_n^* \diag(\Umat_n \cvec^{(i)}) \Umat_n$.
% \begin{equation}
%   \Cmat^{(i)} = \frac{1}{n} \Umat_n^* \diag(\Umat_n \cvec^{(i)}) \Umat_n \enspace.
% \end{equation}
The structured layer can then be written as the following:
\begin{equation}
  \Phi^\rho_{\dvec, \cvec, \bvec} (\xvec) = \rho \left(\frac{1}{n^k} \left(\prod_{i = 1}^{k} \diag\left(\dvec^{(i)}\right) \Umat_n^* \diag\left(\Umat_n \cvec^{(i)}\right) \Umat_n \right) \xvec + \bvec \right)
\end{equation}
\noindent
Although interesting and demonstrating good empirical results, their work suffers form from multiple limitations. 
First, the result of~\citet{huhtanen2015factoring} is expressed with respect to $n$, the size of the matrices $\Amat$.
Therefore, the theorem does not provide any insights regarding the expressive power of $k$ factors when $k$ is much lower than $2n-1$ as it is the case in most practical scenarios they consider.
Finally, for practical reasons, they replaced the Fourier transform with the cosine transform.
As far as we can tell, the theoretical guarantees available with the Fourier Transform do not apply to the cosine transform because the cosine transform does not diagonalize circulant matrices \cite{sanchez1995diagonalizing}.



\begin{figure}[t]
   \centering
   \begin{subfigure}[b]{0.32\textwidth}
       \centering
       \begin{equation*}
	  \leftmatrix
	    \tvec_0      & \tvec_1 & \cdots     & \tvec_{n-1} \\
	    \tvec_{-1}   & \tvec_0 & \ddots     & \vdots      \\
	    \vdots       & \ddots  & \ddots     & \tvec_1     \\
	    \tvec_{-n+1} & \cdots  & \tvec_{-1} & \tvec_0
	  \rightmatrix
       \end{equation*}
       \caption*{Toeplitz}
   \end{subfigure}
   \hfill
   \begin{subfigure}[b]{0.32\textwidth}
       \centering
       \begin{equation*}
	  \leftmatrix
	    1 & \vvec_0     & \cdots & \vvec_0^{n-1} \\
	    1 & \vvec_1     & \cdots & \vvec_1^{n-1} \\
	    1 & \vdots      &        & \vdots        \\
	    1 & \vvec_{n-1} & \cdots & \vvec_{n-1}^{n-1}
	  \rightmatrix
       \end{equation*}
       \caption*{Vandermonde}
   \end{subfigure}
   \hfill
   \begin{subfigure}[b]{0.32\textwidth}
       \centering
       \begin{equation*}
	  \leftmatrix
	  \frac{1}{\uvec_0 - \vvec_{0}}     & \cdots & \frac{1}{\uvec_0 - \vvec_{n-1}} \\
	  \frac{1}{\uvec_1 - \vvec_{0}}     & \cdots & \frac{1}{\uvec_1 - \vvec_{n-1}} \\
	  \vdots                            & \cdots & \vdots                          \\
	  \frac{1}{\uvec_{n-1} - \vvec_{0}} & \cdots & \frac{1}{\uvec_{n-1} - \vvec_{n-1}}
	  \rightmatrix
       \end{equation*}
       \caption*{Cauchy}
   \end{subfigure}
   \caption{Representation of Toeplitz, Vandermonde and Cauchy matrices}
  \label{figure:ch3-example_structure_matrices}
\end{figure}


\drawstar

% %%%%%%%%%%%%%%%%%%%%%%%%%%%%%%%%%%%%%%%%%%%%%%%%%%%%%%%%%%%%%%%%%%%%%%%%%%%%%%%%
% \subsubsection{General Representation of Structured Linear Maps: LDR and K-Matrices}
% \label{subsubsection:ch2-general_representation_of_structured_linear_maps:_ldr_and_k-matrices}
% %%%%%%%%%%%%%%%%%%%%%%%%%%%%%%%%%%%%%%%%%%%%%%%%%%%%%%%%%%%%%%%%%%%%%%%%%%%%%%%%


Other types of structured matrices that reduce the memory footprint but also accelerate the matrix-vector product have been used to build compact neural networks.
Bellow a description of known matrix structured:
\begin{itemize}
  \item \textbf{Toeplitz matrix}: As seen in \Cref{subsection:ch2-toeplitz_matrices}, a Toeplitz matrix, named after Otto Toeplitz, has constant values along each of their diagonals. When the matrix as the same property for anti-diagonals, the matrix is a \emph{Hankel matrix}.
  % \item \textbf{Block Circulant matrix}: A block circulant matrix is a matrix in which each block is repeated identically along diagonals (like block Toeplitz matrices) and each ``row of blocks'' is cyclic right shift of the previous one.
  \item \textbf{Vandermonde matrix}: A Vandermonde matrix, named after Alexandre-Théophile Vandermonde, is a matrix with the terms of a geometric progression in each row.  
    A very important special case is the complex matrix associated with the Discrete Fourier transform (DFT) presented in \Cref{definition:ch2-fourier_matrix} which has Vandermonde structure.
  \item \textbf{Cauchy matrix}: A Cauchy matrix, named after Augustin Louis Cauchy, is a $m \times n$ matrix with elements $a_{ij}$ such that $a_{ij} = (\uvec_i - \vvec_j)^{-1}$ with $\uvec_i - \vvec_j \neq 0$, $i \in [m-1]_0$ and $j \in [n-1]_0$.
\end{itemize}
\Cref{figure:ch3-example_structure_matrices} shows the representation of the parameters sharing of Toeplitz, Vandermonde and Cauchy matrices.
These matrices can be unified into a framework with the notion of \emph{Low Displacement Rank} (LDR).
Each of these matrices can be associated with a displacement operator $\Rmat: \Rbb^{m \times n} \rightarrow \Rbb^{m \times n}$ which takes a matrix, $\Mmat$, and outputs a low rank matrices such that $\rank(\Rmat(\Mmat)) \ll \min(m,n)$.
This displacement rank approach has been initially proposed by~\citet{kailath1979displacement} and have been further studied by~\citet{kailath1995displacement,pan2001structured}. 
More formally, we can define two displacement operators (for simplification, we consider $m = n$):
\begin{definition}[\emph{Sylvester} \& \emph{Stein} displacement operators]
  Let $\Amat, \Bmat \in \Rbb^{n \times n}$, the \emph{Sylvester} displacement operator denoted $\Rmat = \triangleopdown_{\Amat, \Bmat}: \Rbb^{n \times n} \rightarrow \Rbb^{n \times n}$ is defined as follows:
  \begin{equation}
    \triangleopdown_{\Amat, \Bmat} (\Mmat) \triangleq \Amat \Mmat - \Mmat \Bmat
  \end{equation}
  The \emph{Stein} displacement operator denoted $\Rmat = \triangleopup_{\Amat, \Bmat}: \Rbb^{n \times n} \rightarrow \Rbb^{n \times n}$ is defined as follows:
  \begin{equation}
    \triangleopup_{\Amat, \Bmat} (\Mmat) \triangleq \Mmat - \Amat \Mmat \Bmat
  \end{equation}
  where $\triangleopdown_{\Amat, \Bmat} = \Amat \triangleopup_{\Amat^{-1}, \Bmat}$ if the operator matrix $\Amat$ is non-singular, and $\triangleopdown_{\Amat, \Bmat} = -\triangleopup_{\Amat, \Bmat^{-1}}$ if the operator matrix $\Bmat$ is non-singular.
\end{definition}
\begin{table}[t]
  \centering
  {\small
  \begin{tabular}{c|c|c|c}
    \toprule
    \multicolumn{2}{c|}{\textbf{Operator Matrices}} & \textbf{Class of structured} & \textbf{Rank of } \\
    \textbf{A} & \textbf{B} & \textbf{matrices M} & $\triangleopdown_{\Amat, \Bmat}(\Mmat)$ \\
    \midrule
    $\Cmat_1$                & $\Cmat_0$                & Toeplitz               & $\leq 2$ \\
    $\Cmat_1$                & $\Cmat_0^\top$           & Hankel                 & $\leq 2$ \\
    % $\Cmat_0 + \Cmat_0^\top$ & $\Cmat_0 + \Cmat_0^\top$ & Toeplitz + Hankel      & $\leq 4$ \\
    $\diag(\vvec)$           & $\Cmat_0$                & Vandermonde            & $\leq 1$ \\
    % $\Cmat_0$                & $\diag(\vvec)$           & Inverse of Vandermonde & $\leq 1$ \\
    $\diag(\uvec)$           & $\diag(\vvec)$           & Cauchy                 & $\leq 1$ \\
    % $\diag(\vvec)$           & $\diag(\uvec)$           & Inverse of Cauchy      & $\leq 1$ \\
    \bottomrule
  \end{tabular}
  }
  \caption{Displacing Matrices Associated with Families of Structured Matrices \cite{pan2001structured}}
  \label{table:ch2-displacing_matrices}
\end{table}
\noindent
Based on this definition, we can then say that, if $\Mmat$ is Toeplitz, it exists operator matrices $\Amat$ and $\Bmat$ such that $\triangleopdown_{\Amat, \Bmat} (\Mmat)$ or $\triangleopup_{\Amat, \Bmat} (\Mmat)$ is low rank.
In particular, $\Amat$ and $\Bmat$ can be chosen to be diagonal or $f$-unit-circulant matrices for several classes of structured matrices.
\Cref{table:ch2-displacing_matrices} shows some specific choices of operators for the four basic classes of structured matrices and for some related matrices.
An important result allows us to express structured matrices with low-displacement rank directly as a function of its low displacement generators.  
For the Stein type displacement operator, we have the following result:
\begin{theorem}[Krylov Decomposition~\citet{pan2003inversion,sindhwani2015structured}] ~\\
  If an $n \times n$ matrix $\Mmat$ is such that $\triangleopup_{\Amat, \Bmat}(\Mmat) = \Gmat \Hmat^\top$ where 
  $\Gmat = (\gvec^{(1)} \ldots \gvec^{(r)}), \Hmat = (\hvec^{(1)} \ldots \hvec^{(r)}) \in \Rbb^{n \times r}$ 
  and the operator matrices satisfy: $\Amat^n = a \Imat$, $\Bmat^n = b \Imat$ for some scalars $a, b$, then $\Mmat$ can be expressed as: 
  \begin{equation}
    \Mmat = \frac{1}{1 - ab} \sum_{j=1}^{r} ~\krylov(\Amat, \gvec^{(j)}) ~\krylov(\Bmat^\top, \hvec^{(j)})^\top
    \label{equation:ch3-krylov_decomposition}
  \end{equation}
  where $\krylov(\Amat, \vvec)$ is defined by:
  \begin{equation}
    \krylov(\Amat, \vvec) = [\vvec~~\Amat\vvec~~\Amat^2 \vvec \ldots \Amat^{n-1} \vvec]
  \end{equation}
  \label{theorem:ch3-krylov_decomposition}
\end{theorem} 
\noindent
This theorem can be simplified for \emph{Toeplitz-like matrices} as follows:
\begin{theorem}[Toeplitz-like matrix decomposition \citet{pan2001structured}]
  If an $n \times n$ matrix $\Mmat$ satisfies $\triangleopdown_{\Cmat_1, \Cmat_{-1}}(\Mmat) = \Gmat \Hmat^T$ ($\Mmat$ is Toeplitz-like) where $\Gmat = (\gvec ^{(1)} \ldots \gvec^{(r)}], \Hmat = (\hvec^{(1)} \ldots \hvec^{(r)}) \in \Rbb^{n \times r}$, then $\Mmat$ can be written as: 
  \begin{equation} 
    \Mmat = \frac{1}{2} \sum_{j=1}^{r} \Cmat_1(\gvec^{(j)}) \Cmat_{-1}(\Jmat_n \hvec^{(j)})
    \label{equation:ch3-toeplitz_like_matrix_decomposition}
  \end{equation}
  where $\Jmat_n$ is the reflection matrix of size $n \times n$.
\end{theorem}
\noindent
\citet{sindhwani2015structured} have used this theorem in the context of deep learning.
More precisely, they proposed to parameterized neural networks layers as in \Cref{equation:ch3-toeplitz_like_matrix_decomposition} and learn the displacement factors $\Gmat$, $\Hmat$ by gradient descent. 
For some $\Gmat = \leftmat \gvec^{(1)} \ldots \gvec^{(r)} \rightmat, \Hmat = \leftmat \hvec^{(1)} \ldots \hvec^{(r)} \rightmat \in \Rbb^{n \times r}$, their layer is defined as follows:
\begin{equation}
  \Phi^\rho_{\Gmat, \Hmat, \bvec}(\xvec) = \rho\left( \left(\sum_{j=1}^{r} \Cmat_1\left(\gvec^{(j)}\right) \Cmat_{-1}\left(\Jmat_n \hvec^{(j)}\right) \right) \xvec + \bvec \right)
\end{equation}
\noindent
They also show that this class of layer are very rich of a modeling perspective:
\begin{theorem}[LDR expressivity \citet{pan2001structured,sindhwani2015structured}] ~\\
  The set of all $n \times n$ matrices that can be written as, $\sum_{i=1}^{r} \Cmat_1(\gvec^{(i)}) \Cmat_{-1}(\hvec^{(i)})$
  for some $\Gmat = \leftmat \gvec^{(1)} \ldots \gvec^{(r)} \rightmat,
  \Hmat = \leftmat \hvec^{(1)} \ldots \hvec^{(r)} \rightmat \in \Rbb^{n \times r}$ contains:
  \begin{compactitem}
    \item All $n \times n$ Circulant and Skew-Circulant matrices for $r \geq 1$.
    \item All $n \times n$ Toeplitz matrices for $r \geq 2$.
    \item Inverses of Toeplitz matrices for $r \geq 2$.
    \item All products of the form $\Amat^{(1)} \ldots \Amat^{(t)}$ for $r \geq 2t$.
    \item All linear combinations of the form $\sum_{i=1}^p \beta_i \Amat^{(1, i)} \ldots \Amat^{(t, i)}$ where $r \geq 2tp$.
    \item All $n\times n$ matrices for $r=n$.
  \end{compactitem}
  where each $\Amat^{(i)}$ above is a Toeplitz matrix or the inverse of a Toeplitz matrix. 
\end{theorem}
\noindent
In the same line of work, \citet{thomas2018learning} have proposed neural network layers directly form \Cref{equation:ch3-krylov_decomposition} which encompasses an even larger family of structured matrices including Toeplitz-like, Vandermonde-like, Cauchy-like.
Despite being elegant and general, we found that the LDR framework suffers from several limits which are inherent to its generality and makes it difficult to use in the context of large and deep neural networks.
First, the training procedure for learning LDR matrices is highly involved and implies many complex mathematical objects such as Krylov matrices.
Then, as acknowledged by the authors, the number of parameters required to represent a given structured matrix (a Toeplitz matrix) in practice is unnecessarily high (higher than required in theory). 

\drawstar

More recently, another type of generalization of structured linear maps has been proposed by~\citet{dao2020kaleidoscope} and used in the context of deep neural networks.
They introduced a family of matrices called \emph{kaleidoscope matrices} (K-matrices) that capture any sparse matrix with near-optimal space (parameter) and time (arithmetic operation) complexity.
More precisely, their representation is based on products of a particular building block known as a butterfly matrix~\cite{parker1995random,dao2019learning}.






% \todotext{observation that Hadamard matrices when combined with diagonal Gaussian matrices exhibit properties similar to dense Gaussian random matrices. Yet unlike the latter, Hadamard and diagonal matrices are inexpensive to multiply and store}
%
% This result have led to multiple \emph{Fast Johnson-Lindenstrauss Transform} to embed high-dimensional vectors into lower dimensional space with fast algorithm with reduce memory requirement and fast matrix-vector products.
% For example, \citet{ailon2009fast} have proposed the PHD transorm: $\yvec = \Pmat \Hmat \Dmat \xvec$, where $\Pmat$ is a sparse random matrix with Gaussian entries $\Hmat$ is a 
% Hadamard matrix and $\Dmat$ is a diagonal matrix with $\{1, -1\}$ entries drawn independently with probability $1/2$.

% Another fast transform have been proposed involving circulant matrices.
% Recall from \Cref{subsection:ch2-circulant_matrices} that circulant matrices are a special case of Toeplitz matrices, they have constant values along each of their diagonals but also exhibit a circulant pattern where each row of a circulant matrix is a cyclic right shift of the previous one. 


% that circulant, which have been used to perform dimensionality reduction~\cite{hinrichs2011johnson,vybiral2011variant}, binary embedding~\cite{yu2014circulant} and kernel approximation~\cite{yu2015compact} in the context of pattern recognition and machine learning 

% A variety of structured matrices have been used to build compact neural networks.

% circulant matrices are used to perform dimensionality reduction~\cite{hinrichs2011johnson,vybiral2011variant}, binary embedding~\cite{yu2014circulant} and kernel approximation~\cite{yu2015compact} in the context of pattern recognition and machine learning.


% CirCNN: Accelerating and Compressing Deep Neural Networks Using Block-Circulant Weight Matrices
% \cite{ding2017circnn}
%
% Accelerating Deep Neural Networks by Combining Block-Circulant Matrices and Low-Precision Weights
% \cite{qin2019accelerating}
%
% Exploring GPU acceleration of Deep Neural Networks using Block Circulant Matrices
% \cite{dong2020exploring}
%
% Energy-efficient, high-performance, highly-compressed deep neural network design using block-circulant matrices
% \cite{liao2017energy}
%
% CircConv: A Structured Convolution with Low Complexity
% \cite{liao2019circconv}


% \todo{transition}
% -> \cite{ding2017circnn,qin2019accelerating,dong2020exploring,liao2017energy,liao2019circconv}























%%%%%%%%%%%%%%%%%%%%%%%%%%%%%%%%%%%%%%%%%%%%%%%%%%%%%%%%%%%%%%%%%%%%%%%%%%%%%%%
% SAVED DRAFT %%%%%%%%%%%%%%%%%%%%%%%%%%%%%%%%%%%%%%%%%%%%%%%%%%%%%%%%%%%%%%%%%
%%%%%%%%%%%%%%%%%%%%%%%%%%%%%%%%%%%%%%%%%%%%%%%%%%%%%%%%%%%%%%%%%%%%%%%%%%%%%%%

\comment{

In this chapter, we review the literature on the existing techniques for building compact neural networks.
As stated in the introduction (\Cref{chapter:ch1-introduction}), a neural network is a function that can be analytically described as a composition of linear functions interlaced with non-linear functions (also called activation functions).

The number of parameters in a neural network corresponds to the total number of values in each weight matrix of the network.
The goal of building compact neural networks is to reduce the memory footprint, the number of parameters and the computational complexity of the network. 
Mainly three methods exist to achieve this goal:
\begin{itemize}
  \item Leveraging memory representations and data structures;
  \item Using structured matrices instead of dense matrices;
  \item Building compact neural networks architecture.
\end{itemize}
Hereafter, we describe existing techniques that fall into these categories and we review their advantages and their drawbacks.


%%%%%%%%%%%%%%%%%%%%%%%%%%%%%%%%%%%%%%%%%


% Other types of structured matrices with reduce memory requirement and allow fast matrix-vector products and gradient computations have been used in deep learning settings and in many other context \cite{pan2001structured}.
% Bellow are class of structured matrices which have a different type of parameters sharing:
Other types of structured matrices, which reduce memory requirements and allow fast matrix-vector products, have been used in deep learning contexts and in many other contexts \cite{pan2001structured}.
The structured matrices below are a class of structured matrices that have a different type of parameter sharing :
\begin{itemize}
  \item \textbf{Toeplitz matrix}: As seen in \Cref{subsection:ch2-toeplitz_matrices}, a Toeplitz matrix, named after Otto Toeplitz, have constant values along each of their diagonals. When the matrix as the same property for anti-diagonals, the matrix is a \emph{Hankel matrix}.
  \item \textbf{Circulant matrix}: Circulant matrices are a special case of Toeplitz matrices, they have constant values along each of their diagonals but also exhibit a circulant pattern where each row of a circulant matrix is a cyclic right shift of the previous one. 
  % \item \textbf{Block Circulant matrix}: A block circulant matrix is a matrix in which each block is repeated identically along diagonals (like block Toeplitz matrices) and each ``row of blocks'' is cyclic right shift of the previous one.
  \item \textbf{Doubly-Block Toeplitz matrix}: A doubly-block Toeplitz matrix is a block Toeplitz matrix in which each block is itself a Toeplitz matrix.
  \item \textbf{Vandermonde matrix}: A Vandermonde matrix, named after Alexandre-Théophile Vandermonde, is a matrix with the terms of a geometric progression in each row.  
    A very important special case is the complex matrix associated with the Discrete Fourier transform (DFT) presented in \Cref{definition:ch2-fourier_matrix} which has Vandermonde structure.
  \item \textbf{Cauchy matrix}: A Cauchy matrix, named after Augustin Louis Cauchy, is a $m \times n$ with elements $a_{ij}$ such that $a_{ij} = (\uvec_i - \vvec_j)^{-1}$ with $\uvec_i - \vvec_j \neq 0$, $i \in [m-1]_0$ and $j \in [n-1]_0$.
\end{itemize}
\Cref{figure:ch3-example_structure_matrices} shows the representation of the parameters sharing of Toeplitz, Vandermonde and Cauchy matrices.

%%%%%%%%%%%%%%%%%%%%%%%%%%%%

% Other, more complex, structured projections have been used in the context of compact neural networks.  % The Fastfood transform~\cite{le2013fastfood}, which was originally used for approximating kernel expansions, was later used in neural networks by~\citet{yang2015deep} leading to the Deep Fried Convnets architecture.
% The authors have replaced dense matrices of fully connected layers with adaptative structured matrices of the form: $\Smat\Hmat\Gmat\mathbf{\Pi}\Hmat\Bmat$ where $\Smat$, $\Gmat$, and $\Bmat$ are adaptive diagonal matrices, $\mathbf{\Pi}$ is a random permutation matrix, and $\Hmat$ is the Walsh-Hadamard matrix.
% Later, the \emph{Structured Spinners} transform of the form: $\Hmat\Dmat_3\Hmat\Dmat_2\Hmat\Dmat_1$, where $\Hmat$ is the Walsh-Hadamard matrix, and $\Dmat_i$ for $i \in {1, 2, 3}$ is a random $\pm1$-diagonal matrix, was originally proposed by~\citet{andoni2015practical} and used in deep learning settings by~\citet{bojarski2017structured}.
%


}



%%%%%%%%%%%%%%%%%%%%%%%%%%%%%%%%%%%%%%%%%%%%%%%%%%%%%%%%%%%%%%%%%%%%%%%%%%%%%%%%
\subsection{Discussion}
%%%%%%%%%%%%%%%%%%%%%%%%%%%%%%%%%%%%%%%%%%%%%%%%%%%%%%%%%%%%%%%%%%%%%%%%%%%%%%%%

In this section, we have shown current methods and techniques for designing compact neural networks with structured matrices. 
Our contributions on \emph{Deep Diagonal Circulant Neural Networks} are a direct follow-up to the work of~\citet{cheng2015exploration,sindhwani2015structured,moczulski2016acdc,thomas2018learning} focusing on compact neural networks with \emph{structured matrices}.
More precisely, we extend the work of \citet{moczulski2016acdc} by training \emph{fully structured networks} (\ie, networks with structured layers only) hence demonstrating that diagonal circulant layers are able to model complex relations between inputs and outputs.
Although, this diagonal circulant layers fit in the low displacement rank framework, we demonstrate much better performances in practice.
Indeed, thanks to a solid theoretical analysis and thorough experiments, we were able to train deep (up to 40 layers) circulant neural networks, and apply, for the first time, this structured architecture in the context of large-scale video classification.
This contrasts with previous experiments in which only one or a few dense layers were replaced inside a large redundant network such as VGG~\cite{simonyan2014very}.

\pagebreak

%%%%%%%%%%%%%%%%%%%%%%%%%%%%%%%%%%%%%%%%%%%%%%%%%%%%%%%%%%%%%%%%%%%%%%%%%%%%%%%%
\section{Related Work on Lipschitz Regularization}
\label{section:ch3-related_work_on_lipschitz_regularization}
%%%%%%%%%%%%%%%%%%%%%%%%%%%%%%%%%%%%%%%%%%%%%%%%%%%%%%%%%%%%%%%%%%%%%%%%%%%%%%%%

%%%%%%%%%%%%%%%%%%%%%%%%%%%%%%%%%%%%%%%%%%%%%%%%%%%%%%%%%%%%%%%%%%%%%%%%%%%%%%%%
\subsection{The Global Lipschitz Constant of Neural Networks}
\label{subsection:ch3-the_global_lipschitz_constant_of_neural_networks}
%%%%%%%%%%%%%%%%%%%%%%%%%%%%%%%%%%%%%%%%%%%%%%%%%%%%%%%%%%%%%%%%%%%%%%%%%%%%%%%%

\noindent
The regularization of the Lipschitz constant of neural networks has seen a growing interest in the last few years.
Indeed, numerous results have shown that neural networks with a low Lipschitz constant exhibit better generalization~\cite{bartlett2017spectrally} and higher robustness to adversarial attacks~\cite{szegedy2013intriguing,tsuzuku2018lipschitz, farnia2018generalizable}.

The Lipschitz constant, defined in~\Cref{definition:ch2-lipschitz_constant}, is a measure of the stability of the network.
If the Lipschitz constant is high, the network will tend to be more sensitive to input perturbations, meaning, if the input changes by $\epsilon$, the output changes by at most $k\epsilon$.
The Lipschitz constant of a function can also be expressed using the differential operator as follows:
\begin{theorem}[Rademacher's Theorem] \label{theorem:ch3-lipschitz_differential_op}
  If $f: \Rbb^n \rightarrow \Rbb^m$ is a Lipschitz continuous function, then $f$ is differentiable almost everywhere.
  Moreover, if $f$ is Lipschitz continuous, then
  \begin{align}
    \lip{f} = \sup_{\xvec \in \Rbb^n} \norm{\mathrm{D}_\xvec f(\xvec)}_2
  \end{align}
  where $\mathrm{D}_\xvec$ is the differential operator of $f$ at $\xvec$.
\end{theorem}

\citet{tsuzuku2018lipschitz} have studied the relationship between the robustness and the Lipschitz constant and the margin of neural networks. 
By the definition of the Lipschitz constant, we have the following:
\begin{equation}
  \norm{N_\Omega(\xvec) - N_\Omega(\xvec + \adv)}_2 \leq \lip{N_\Omega} \norm{\adv}_2
\end{equation}
Recall the margin operator $\margin: \Rbb^k \times [k] \rightarrow \Rbb$ from~\Cref{subsection:ch2-recent_results_on_the_theory_of_neural_networks} defined as:
\begin{equation}
  \margin(\vvec, j) \triangleq \vvec_j - \max_{i \neq j} \vvec_i
\end{equation}
Then, we have the following proposition which characterizes the robustness of a neural network with respect to its margin and Lipschitz constant.
\begin{proposition}[\citet{tsuzuku2018lipschitz}]
  \begin{equation} \label{equation:ch3-margin_guarded_area}
    \margin \big( N_\omega(\xvec), y \big) \geq \sqrt{2} \lip{N} \norm{\adv}_2 \quad \Longrightarrow \quad \margin \big( N_\Omega(\xvec + \adv), y \big) \geq 0
  \end{equation}
  \removespace
\end{proposition}
\noindent
If the inequality on the right-hand side of \Cref{equation:ch3-margin_guarded_area} is verified then the adversarial margin is positive, \ie, the network correctly predicts the label. 
From this proposition, we can conclude that for a given neural network with specific margins, a lower Lipschitz constant allows for an increase in robustness. 
Note that the margin is already maximized in a multi-class setting with the cross-entropy loss as stated in~\citet{hein2017formal}.
A multitude of work have tried to reduce the Lipschitz constant in order to improve adversarial robustness.
However, \citet{scaman2018lipschitz} have shown that computing the exact Lipschitz constant of a neural network is NP-hard.
The following theorem shows that, even for shallow neural networks, exact Lipschitz computation is not achievable in polynomial time:
\begin{theorem}[\citet{scaman2018lipschitz}] \label{theorem:ch3-lipschitz_computation}
  Let us define the problem associated with the exact computation of the Lipschitz constant of a $2$-layer neural network with $\relu$ activation:
  \begin{itemize}%[topsep=0pt,noitemsep]
    \item[] \textbf{Input:} Two matrices $\Wmat^{(1)} \in \Rbb^{l \times n}$ and $\Wmat^{(2)} \in \Rbb^{m \times l}$, and a constant $c \geq 0$.
    \item[] \textbf{Question:} Let $N = \Wmat^{(2)} \circ \rho \circ \Wmat^{(1)}$ where $\rho$ is the $\relu$ activation function. \emph{Is the Lipschitz constant $\lip{N} \leq c$ ?}
  \end{itemize}
  Then, assuming that $\textbf{P} \neq \textbf{NP}$, the problem above is NP-hard. 
\end{theorem}


\noindent
To overcome this difficulty, researchers have relied on devising a tight upper bound of the Lipschitz constant.
For example, \citet{scaman2018lipschitz} have shown that the Lipschitz constant of a neural network $N$ can be explicitly formulated using \Cref{theorem:ch3-lipschitz_differential_op} and the chain rule:
\begin{equation} \label{equation:ch3-decomposition_jacobian_lipschitz}
  \lip{\nn} = \sup_{x \in \Rbb^n} \norm{\Wmat^{(p)} \diag(\rho'_\depth(\theta_\depth)) \dots \Wmat^{(2)} \diag(\rho'_1(\theta_1)) \Wmat^{(1)}}_2,
\end{equation}
where $\theta_i = \layer^{\act_i}_{\Wmat^{(i)}, \bvec^{(i)}} \circ \cdots \circ \layer^{\act_1}_{\Wmat^{(1)}, \bvec^{(1)}}(\xvec)$ is the intermediate output after $i$ layers.
The Lipschitz of the neural network $N$ can then be upper bounded as follows:
\begin{align}
  \lip{\nn} &\leq \max_{\forall i,\ \sigma_i \in [0, 1]^{w^{(i+1)}}} \norm{\Wmat^{(\depth)} \diag(\sigma_{\depth-1}) \dots \diag(\sigma_1) \Wmat^{(1)}}_2 \notag \\
  &\leq \max_{\forall i,\ \sigma_i \in [0, 1]^{w^{(i+1)}}} \norm{ \pmb{\Sigma}^{(\depth)} \Vmat^{(\depth)\top} \diag(\sigma_{\depth-1}) \dots \diag(\sigma_1) \Umat^{(1)} \pmb{\Sigma}^{(1)}}_2 \notag \\
  &\leq \prod_{i=1}^{\depth-1} \max_{\sigma_i \in [0, 1]^{w^{(i+1)}}} \norm{\widetilde{\pmb{\Sigma}}^{(i+1)} \Vmat^{(i+1)\top} \diag(\sigma_{i+1}) \Umat^{(i)} \widetilde{\pmb{\Sigma}}^{(i)}}_2 
\end{align}
where $\widetilde{\pmb{\Sigma}}^{(i)} = \pmb{\Sigma}^{(i)}$ if $i \in \{1, \depth\}$ and $\widetilde{\pmb{\Sigma}}^{(i)} = {\pmb{\Sigma}^{(i)}}^{1/2}$ otherwise.
The first inequality is due to the fact that the derivatives of the activation functions are bounded, \ie, $\rho_i(\xvec) \in [0, 1]^{w^{(i+1)}}$, the second inequality is obtained by decomposing each weight matrix $\Wmat^{(i)}$ with the \emph{Singular Value Decomposition} such that $\Wmat^{(i)} = \Umat^{(i)} \pmb{\Sigma}^{(i)} \Vmat^{(i)\top}$; and finally, the last inequality is due to the submultiplicativity of the operator norm.
Although accurate, this bound is still computationally expensive to compute due to the singular value decomposition and the optimization for each layer. 
In the same line of research, recent work~\cite{fazlyab2019safety,fazlyab2019efficient,latorre2020lipschitz} has proposed a tight bound on the Lipschitz constant of the full network with the use of semi-definite programming.
More precisely, \citet{fazlyab2019efficient} have demonstrated the following result:
\begin{theorem}[Lipschitz bounds \citet{fazlyab2019efficient}] \label{theorem:ch3-lipschite_semidefinite_programming}
  Consider a neural network $N: \Rbb^n \rightarrow \Rbb^m$ such that $N(\xvec) = \Wmat^{(2)} \rho(\Wmat^{(1)} \xvec + \bvec^{(1)}) + \bvec^{(2)}$.
  Suppose the activation function $\rho$ is \emph{slope-restricted} in the sector $[\alpha,\beta]$, \ie,
  \begin{equation}
    \alpha \leq \frac{\rho(y) - \rho(x)}{y-x} \leq \beta \quad \forall x,y \in \Rbb. 
  \end{equation}
  Define the set $\mathcal{T}_{n}$ as the following:
  \begin{equation*}
    \mathcal{T}_n = \{\Tmat \in \Sbb^n \mid \Tmat = \sum_{i=1}^{n} \lambda_{ii} \evec^{(i)} \evec^{(i)\top} + \sum_{1 \leq i<j \leq n} \lambda_{ij}(\evec^{(i)} - \evec^{(j)})(\evec^{(i)}-\evec^{(j)})^\top, \lambda_{ij} \geq 0 \}.
  \end{equation*}
  where $\Sbb^d$ is the set of all symmetric matrices of size $n \times n$.
  Suppose there exists a constant $c>0$ such that the matrix inequality
  \begin{align}
    \Mmat(c,\Tmat) \triangleq
      \leftmatrix
      -2\alpha \beta \Wmat^{(1)\top} \Tmat \Wmat^{(1)} - c \Imat_n & (\alpha+\beta) \Wmat^{(1)\top} \Tmat  \\
      (\alpha+\beta) \Tmat \Wmat^{(1)} & -2\Tmat+\Wmat^{(2)\top} \Wmat^{(2)}
      \rightmatrix
      \leq 0,
  \end{align}
  holds for some $\Tmat \in \mathcal{T}_{n}$. Then $\norm{N(\xvec)-N(\yvec)}_2 \leq \sqrt{c} \norm{\xvec-\yvec}_2$ for all  $\xvec,\yvec \in \Rbb^n$.
\end{theorem}
\noindent
From \Cref{theorem:ch3-lipschite_semidefinite_programming}, the constant $c$ is an upper bound on the Lipschitz constant of the network.
The authors proposed to find the tightest bound by solving the following optimization problem (Semidefinite Program):
\begin{align}
  \textrm{minimize} \quad c \quad \text{ subject to} \quad \Mmat(c,\Tmat) \leq 0 \quad \text{and} \quad \Tmat \in \mathcal{T}_{n},
\end{align}
where the decision variables are $(c,\Tmat) \in \Rbb_+ \times \mathcal{T}_n$.
Note that $\Mmat(c,\Tmat)$ is linear in $c$ and $\Tmat$ and the set $\mathcal{T}_n$ is convex.
Although, these works on devising a global bound on the Lipschitz constant of a neural network are theoretically interesting, they lack scalability
They can only be computed on small networks and cannot be used during the training of large neural networks for regularization purposes.


%%%%%%%%%%%%%%%%%%%%%%%%%%%%%%%%%%%%%%%%%%%%%%%%%%%%%%%%%%%%%%%%%%%%%%%%%%%%%%%%
\subsection{Lipschitz Constant of Individual Layers}
\label{subsection:ch3-lipschitz_constant_of_individual_layers}
%%%%%%%%%%%%%%%%%%%%%%%%%%%%%%%%%%%%%%%%%%%%%%%%%%%%%%%%%%%%%%%%%%%%%%%%%%%%%%%%

\noindent
% In order to constrain the Lipschitz constant of neural networks, 
Instead of regularizing the ERM using the global Lipschitz constant, researchers have devised techniques to reduce the Lipschitz constant of \emph{individual layers} instead. 
The global Lipschitz of a neural network can easily be upper bounded by the product of the spectral norm of each weight matrix as follows:
\begin{proposition}[\citet{scaman2018lipschitz}] \label{proposition:ch3-naive_upper_bound_lipschitz}
  Let $N$ be a neural network of $\depth$ layers with 1-Lipschitz activation functions (\eg ReLU,
  Leaky ReLU, Tanh, Sigmoid, etc.), then, the Lipschitz constant of the neural network can be upper bounded as follows:
  \begin{equation} \label{equation:ch3-naive_upper_bound_lipschitz}
    \lip{N} \leq \prod_{i=1}^\depth \norm{\Wmat^{(i)}}_2 \enspace,
  \end{equation}
  where $\Wmat^{(i)}$ are the weights matrices of the neural network.
\end{proposition}

\begin{remark}
  The Lipschitz constant of a layer $\layer_{\Wmat, \bvec}^\rho$ (with a 1-Lipschitz activation function) is equal to the spectral norm of the matrix $\Wmat$ (largest singular value).
  Let $\layer_{\Wmat, \bvec}^\rho: \Rbb^n \rightarrow \Rbb^m$ such that $\layer_{\Wmat, \bvec}^\rho = \rho(\Wmat \xvec + \bvec)$ then by definition of the Lipschitz constant (see \Cref{definition:ch2-lipschitz_constant}) and of the operator norm, we have:
  \begin{equation}
    \lip{\layer_{\Wmat, \bvec}^\rho} = \sup_{\substack{\xvec \in \Rbb^n \\ \xvec \neq 0}} \frac{\norm{\Wmat \xvec}_2}{\norm{\xvec}_2} = \norm{\Wmat}_2
  \end{equation}
  \removespace
\end{remark}

The trivial bound given by the product of layer-wise Lipschitz constants in \Cref{equation:ch3-naive_upper_bound_lipschitz} is known to be loose and pessimistic.
Furthermore, we can show that reducing the Lipschitz constant of each layer independently does not imply that the global Lipschitz constant of the network will be reduced. 




\begin{proposition} \label{proposition:ch3-limit_bound_lipschitz}
  Let $N$ be a neural network, then decreasing the Lipschitz constant of one or more layers does not imply reducing the Lipschitz constant of the network, \ie, $\lip{N}$.
\end{proposition}

% \begin{proposition} \label{proposition:ch3-limit_bound_lipschitz}
%   Let $N: \Rbb^n \rightarrow \Rbb^n$ be a neural network such that $N(\xvec) = \Wmat^{(2)} \rho(\Wmat^{(1)} \xvec)$ with $\Wmat^{(1)}, \Wmat^{(2)} \in \Rbb^{n \times n}$.
%   Then, decreasing the Lipschitz constant of each layer does not imply that the Lipschitz constant of the network will be lower. 
% \end{proposition}

% \begin{proposition} \label{proposition:ch3-limit_bound_lipschitz}
%   Let $N_1(\xvec) = \Amat^{(2)} \rho(\Amat^{(1)} \xvec)$ and $N_2(\xvec) = \Bmat^{(2)} \rho(\Bmat^{(1)} \xvec)$ where $\rho$ is the $\relu$ activation function, then $\norm{\Amat^{(1)}}_2 \leq \norm{\Bmat^{(1)}}_2$ and $\norm{\Amat^{(2)}}_2 \leq \norm{\Bmat^{(1)}}_2$, does not imply that $\lip{N_1} \leq \lip{N_2}$.
% \end{proposition}


\begin{proof}[\Cref{proposition:ch3-limit_bound_lipschitz}]
  Let us prove this claim with a counter-example.
  Let $N_1(\xvec) = \Amat^{(2)} \rho(\Amat^{(1)} \xvec)$ and $N_2(\xvec) = \Bmat^{(2)} \rho(\Bmat^{(1)} \xvec)$ where $\rho$ is the $\relu$ activation function.
  Let
  \begin{align*}
    \Amat^{(1)} &= \leftmatrix 
      \phantom{+}0 & -1 \\ -1 & \phantom{+}0
    \rightmatrix \quad
    \Amat^{(2)}  = \leftmatrix
      -1 & -1 \\ -1 & \phantom{+}0
    \rightmatrix \\
    \Bmat^{(1)} &= \leftmatrix
      \phantom{+}0 & \phantom{+}0 \\ \phantom{+}0 & -1
    \rightmatrix \quad
    \Bmat^{(2)} = \leftmatrix
      -1 & -1 \\ -1 & -1
    \rightmatrix
  \end{align*}
  then: \vspace{-0.5cm}
  \begin{equation*}
    \norm{\Amat^{(1)}}_2 = 1,\ \norm{\Amat^{(2)}}_2 = \sqrt{2}
    \quad \text{and} \quad
    \norm{\Bmat^{(1)}}_2 = 1,\ \norm{\Bmat^{(2)}}_2 = 2
  \end{equation*}
  From \Cref{theorem:ch3-lipschitz_differential_op} and the chain rule, the Lipschitz constant of the networks $N_1$ and $N_2$ can be expressed as follows:
  \begin{align*}
    \lip{N_1} &= \sup_{\xvec \in [0, 1]^2} \norm{\Amat^{(2)} \diag\left(\xvec\right) \Amat^{(1)}}_2 \\
    \lip{N_2} &= \sup_{\xvec \in [0, 1]^2} \norm{\Bmat^{(2)} \diag\left(\xvec\right) \Bmat^{(1)}}_2
  \end{align*}
  It is easy to verify that:
  \begin{equation*}
    \lip{N_1} = \frac{1 + \sqrt{5}}{2} \approx 1.618 \quad \text{and} \quad \lip{N_2} = \sqrt{2} \approx 1.414
  \end{equation*}
  which concludes the proof.
\end{proof}
\noindent
While we cannot have a guarantee that the global Lipschitz will be reduced, we could still have an idea of the value of the global Lipschitz with the upper bound presented in~\Cref{equation:ch3-naive_upper_bound_lipschitz}.


\citet{huster2018limitations} have demonstrated several limitations on the expressive power of neural networks where the product of layer-wise Lipschitz constants is constrained.
In the same vein, \citet{couellan2019coupling} empirically showed that Lipschitz Regularization offers a trade-off between adversarial robustness and expressivity of the network.
However, the bound in \Cref{equation:ch3-naive_upper_bound_lipschitz} appears in multiple generalization bound~\cite{neyshabur2017,bartlett2017spectrally,golowich2018} and adversarial generalization~\cite{farnia2018generalizable} (see \Cref{chapter:ch2-background}) which could suggest that reducing the bound would improve the generalization capabilities of neural networks and its robustness.

Based on this theoretical insight, researchers have developed several techniques to constrain the Lipschitz constant of each layer in order to improve the generalization and robustness of neural networks.
A technique to enforce 1-Lipschitz layers is to impose or promote an orthogonality constrain of the weight matrices.
A square orthogonal matrix $\Mmat$ is a matrix whose columns and rows are orthogonal unit vectors and all eigenvalues are equal to 1.
\citet{cisse2017parseval} and more recently \citet{wang2020orthogonal,huang2020controllable} have proposed to minimize the following term:
\begin{equation} \label{equation:ch3-orthogonality_constraint}
  \frac{\beta}{2} \norm{\Wmat^\top \Wmat - \Imat}_2  \enspace, 
\end{equation}
to promote the orthogonality constraint, in addition to the usual loss function:
In the above equation, the hyper-parameter $\beta$ controls the constraint.
A higher $\beta$ would lead to a better orthogonality constraint and therefore, a Lipschitz constant ``almost'' equal to 1 for all the layers.

On the other hand, \citet{anil2019sorting} proposed to enforce the orthogonality of weight matrices by directly optimizing on the Stiefel
manifold (\ie, the manifold of orthogonal matrices, see~\citet{absil2009optimization}).
To perform this optimization, they made use of an iterative algorithm first introduced by~\citet{bjorck1971iterative}.
For a given matrix $\Wmat = \Wmat^{(0)}$, the algorithm finds the closest orthonormal matrix by computing the following term:
\begin{equation}
  \Wmat^{(k+1)} = \Wmat^{(k)} \left( \Imat + \frac{1}{2} \Vmat^{(k)} + \cdots + (-1)^r {-\frac{1}{2} \choose r}  \left(\Vmat^{(k)}\right)^r \right)
\end{equation}
where $\Vmat = \Imat - \Wmat^{(k)\top} \Wmat^{(k)}$.
Although this algorithm works well on dense matrices, it can be difficult to apply it to convolutions. 
\citet{li2019preventing} built upon this idea and proposed an algorithm to enforce the orthogonality of convolutional layers.
They used the orthogonal projection proposed by \citet{kautsky1994matrix} and \citet{xiao2018dynamical} to build convolutional neural networks with orthogonal convolutions.

All techniques that impose an orthogonality constraint on the weights matrices successfully reduce the Lipschitz constant of the layers of the networks.
Moreover, when the Lipschitz constants of all the layers are low, we could have an idea of the value of the global Lipschitz with the upper bound of~\Cref{equation:ch3-naive_upper_bound_lipschitz} (\ie, if Lipschitz constant of all the layers are equal to 1, then, the network is $1$-Lipschitz).
However, enforcing the orthogonality constraint, either by regularizing with the term of~\Cref{equation:ch3-orthogonality_constraint} or by optimizing on the Stiefel manifold, is the costly operation which make it difficult to scale on large neural networks.




\begin{algorithm}[tb]
  \caption{Power method for producing the largest singular value, $\sigma_1$, of a non-square matrix, $\Wmat$ \cite{gouk2018regularisation,golub2000eigenvalue}}
  \begin{algorithmic}[1]
    \Require{affine function $f(\xvec) = \Wmat \xvec + \bvec$, number of iteration $N$}
    \Ensure{approximation of the Lipschitz constant $\lip{f}$}
    \State Randomly initialise $\xvec$
    \For{$i = 1$ \textbf{to} $N$}
      \State $\xvec \gets \Wmat^\top \Wmat \xvec / \norm{\xvec}_2$
    \EndFor
    \State \textbf{return} $\norm{\Wmat \xvec}_2 / \norm{\xvec}_2$
  \end{algorithmic}
  \label{algorithm:ch3-power_method}
\end{algorithm}

\begin{algorithm}[tb]
  \caption{Convolutional power method \cite{farnia2018generalizable}}
  \begin{algorithmic}[1]
    \Require{2d-convolution function $f: \Rbb^{n \times n} \rightarrow \Rbb^{m \times m}$ with kernel $k$, 2d-convolution-transpose function $g: \Rbb^{n \times n} \rightarrow \Rbb^{m \times m}$ with kernel $k$ number of iteration $N$}
    \Ensure{approximation of the Lipschitz constant $\lip{f}$}
    \State Initialize $\xvec$ with a random vector matching the shape of the convolution input
    \For{$i = 1$ \textbf{to} $N$}
      \State $\xvec \gets f(\xvec) / \norm{f(\xvec)}_2 $
      \State $\xvec \gets g(\xvec) / \norm{g(\xvec)}_2$
    \EndFor
    \State \textbf{return} $\norm{f(\xvec)}_2 / \norm{\xvec}_2$
  \end{algorithmic}
  \label{algorithm:ch3-power_method_generic}
\end{algorithm}


Another technique, called \emph{Spectral Normalization}, consists in normalizing each weight matrix by its largest singular value, thus imposing each layer to be 1-Lipschitz.
As with the orthogonality constraint, this technique leads the network to have a global Lipschitz constant of 1.
\citet{yoshida2017spectral} were the first to propose this method to improve the generalization of neural networks followed by~\cite{miyato2018spectral,gouk2018regularisation,farnia2018generalizable} for improving generalization and robustness against adversarial attacks.
In order to perform spectral normalization, they divided the values of each weight matrix by an approximation of its largest singular value.
The approximation of the largest singular was computed using the power method~\cite{golub2000eigenvalue}.

% used the power method~\cite{golub2000eigenvalue} to compute an approximation of the largest singular value of each weight matrix and divided all the values of the weight matrix by this 

The power method is an iterative eigenvalue algorithm (also known as the Von Mises iteration \cite{mises1929praktische}).
Given a matrix $\Wmat$ and a random vector $\bvec^{(0)}$, the eigenvector associated with the largest eigenvalue of the matrix $\Wmat$ can be computed with the following recurrence relation:
\begin{equation}
  \bvec^{(k+1)} = \frac{\Wmat \bvec^{(k)}}{\norm{\bvec^{(k)}}_2}  
\end{equation}
Then, the largest eigenvalue (when we talk about ``largest eigenvalue'' we mean in absolute value) can be optained with the \emph{Rayleigh quotient}:
\begin{equation}
  \sigma_1\left( \Wmat \right) = \frac{\bvec^{(k)\top} \Wmat \bvec^{(k)}}{\bvec^{(k)\top} \bvec^{(k)}}
\end{equation}
With a sufficient number of iterations, the algorithm provably converges to the largest eigenvalue of the matrix.
To find the largest singular value, we can leverage the relation between eigenvalues and singular values:
\begin{equation}
  \sigma \left( \Wmat \right) = \sqrt{ \lambda \left( \Wmat^\top \Wmat \right) }
\end{equation}
The rate of convergence of the algorithm depends on the ratio between the second-largest eigenvalue and the largest eigenvalue.
Indeed, a high ratio can lead to slow convergence.
The pseudocode of the power method is given in \Cref{algorithm:ch3-power_method}.
Altough, \Cref{algorithm:ch3-power_method} needs explicit matrix for computing the largest singular value, \citet{farnia2018generalizable,ryu2019plug} extended the power method to convolutional layers where the matrix $\Wmat$ is not explicitly constructed.
The pseudocode of their method is presented in \Cref{algorithm:ch3-power_method_generic}. 

In the context of deep learning and spectral normalization, the largest singular value needs to be computed for each layer of the network at each step of the training. 
Given that current state-of-the-art architecture have between 50 and 100 layers \cite{he2016deep,tan2019efficientnet}, using the power method \emph{until convergence} is prohibitive.
In~\Cref{chapter:ch5-lipschitz_bound}, we propose a new regularization scheme for reducing the Lipschitz constant of individual layers.
We will shown in~\Cref{subsection:ch5-comparison_of_lipbound_with_other_state-of-the-art_approaches} that our approach is more efficient that the power method even with a small number of iterations.



%%%%%%%%%%%%%%%%%%%%%%%%%%%%%%%%%%%%%%%%%%%%%%%%%%%%%%%%%%%%%%%%%%%%%%%%%%%%%%%%
\subsection{Singular Values of Convolutional Layers}
\label{subsection:ch3-singular_values_of_convolutional_layers}
%%%%%%%%%%%%%%%%%%%%%%%%%%%%%%%%%%%%%%%%%%%%%%%%%%%%%%%%%%%%%%%%%%%%%%%%%%%%%%%%

The power method is not the only technique available for approximating the largest singular value (Lipschitz constant) of a convolutional layer.
Several works have devised bounds or approximations on the largest singular value of convolutional layers by exploiting the \emph{structure} of the convolution operation \cite{sedghi2018singular,bibi2019deep,singla2019bounding,jia2017improving}.

% \citet{sedghi2018singular} have observed that a doubly-block Toeplitz matrix can be approximated by a 
%
% \citet{sedghi2018singular} have exploited the properties 

% \citet{gray2006toeplitz} have observed that band-Toeplitz matrices can be `approximated' by band-circulant matrices.
% This `approximation' is formalized by a mathematical concept called \emph{}j

To approximate the singular values of a convolutional layer, \citet{sedghi2018singular} have exploited the properties of doubly-block circulant matrices (\ie, a circulant block matrix where each block is also a circulant matrix).
Indeed, a doubly-block circulant matrix is the matrix representation of a convolution with circulant padding.
In their work, \citet{sedghi2018singular} assume that the properties of doubly-block circulant matrices are `close' to the properties of a doubly-block Toeplitz matrix.

To compute the singular values of doubly-block circulant matrices, \citet{sedghi2018singular} have demonstrated the following result:
\begin{theorem}[Theorem 5 from \citet{sedghi2018singular}] \label{theorem:ch3-singular_values_doubly_block_circulant}
  Let $\Amat$ be a doubly-block circulant matrix such that:
  \begin{equation*}
    \Amat = \leftmatrix
      \Cmatsf^{(0)}   & \Cmatsf^{(n-1)} & \Cmatsf^{(n-2)} & \cdots        & \cdots          & \Cmatsf^{(1)}   \\
      \Cmatsf^{(1)}   & \Cmatsf^{(0)}   & \Cmatsf^{(n-1)} & \ddots        &                 & \vdots          \\
      \Cmatsf^{(2)}   & \Cmatsf^{(1)}   & \ddots          & \ddots        & \ddots          & \vdots          \\
      \vdots          & \ddots          & \ddots          & \ddots        & \Cmatsf^{(n-1)} & \Cmatsf^{(n-2)} \\
      \vdots          &                 & \ddots          & \Cmatsf^{(1)} & \Cmatsf^{(0)}   & \Cmatsf^{(n-1)} \\
      \Cmatsf^{(n-1)} & \cdots          & \cdots          & \Cmatsf^{(2)} & \Cmatsf^{(1)}   & \Cmatsf^{(0)}
    \rightmatrix
  \end{equation*}
  where $\Cmatsf^{(i)} = \circulant({\cvec_i}),\ \forall i \in \Iset_n^+$.
  Let $\Kmat = \leftmat \cvec_0, \cvec_1, \cdots, \cvec_{n-1} \rightmat^\top$ then, the singular values of the doubly-block circulant matrix $\Amat$ are the modulus of the entries of $\Umat_n^\top \Kmat \Umat_n$.
\end{theorem}


\noindent
To prove \Cref{theorem:ch3-singular_values_doubly_block_circulant}, \citet{sedghi2018singular} used the diagonalization of doubly-block circulant matrices (see~\Cref{chapter:ch2-background}, \Cref{equation:ch2-diagonalization_doubly_block_circulant_matrix}).
The main advantage of this approach is that the singular values of a doubly-block circulant matrix can be computed with the Fast Fourier Transform algorithm (see \Cref{section:ch2-a_primer_on_circulant_and_toeplitz_matrices}) which offers a reduced complexity compared to classical approaches for computing the singular values of a matrix.
However, this approach exhibits several limitations.
First, this method results in a loose approximation of the maximal singular value of a convolutional layer which does not use the circulant padding which is often the case in practical settings.
Also, the complexity of their algorithm is dependent on the size of the input which can be high for large datasets.
Finally, for multi-channel convolution, their method requires the computation of the spectral norm of $n^2$ matrices each of size $\cin \times \cout$ as stated in the following theorem:

\begin{theorem}[Theorem 6 from \citet{sedghi2018singular}]
  Let $\Mmat$ be the matrix encoding the linear transform computed by a multi-channel convolutional layer.
  Let $\Kmat \in \Rbb^{\cin \times \cout \times n \times n}$ such that $(\Kmat)_{i,j}$ for all $i,j$ be constructed as in \Cref{theorem:ch3-singular_values_doubly_block_circulant}, 
  Let $\widetilde{\Kmat}_{i,j} = \Umat^\top \leftmat \Kmat \rightmat_{i,j} \Umat_n $ and define the following operator matrix 
  \begin{equation}
    \Pmat(i,j) = \leftmatrix 
    \leftmat (\widetilde{\Kmat})_{(0,0)} \rightmat_{i, j} & \cdots & \leftmat (\widetilde{\Kmat})_{(0, \cout-1)} \rightmat_{i, j} \\
    \vdots & & \vdots \\
    \leftmat (\widetilde{\Kmat})_{(\cin-1, 0)} \rightmat_{i, j} & \cdots & \leftmat (\widetilde{\Kmat})_{(\cin-1, \cout-1)} \rightmat_{i, j}
    \rightmatrix
  \end{equation}
  Then
  \begin{equation}
    \sigma(\Mmat) = \bigcup_{i, j = 0}^{n-1} \sigma \left(  \Pmat(i,j) \right).
  \end{equation}
  \removespace
\end{theorem}



% , their method requires the computation of the spectral norm of n
% 2 matrices (each matrix of
% size cout × cin) for every convolution layer making it impractical to use during training.


In the same vein, \citet{singla2019bounding} have used the properties of convolutions to devise several bounds on the singular values of convolution layers.
Recall from \Cref{subsubsection:ch2-relation_with_the_convolution_operator} that a convolution kernel is a 4 dimensional tensor of size $\cout \times \cin \times k_1 \times k_2$.
\citet{singla2019bounding} have demonstrated that the largest singular value of a convolution layer $\layer_\Kmat$ parameterized by a kernel $\Kmat$ can be upper-bounded as follows:

\begin{theorem}[Reformulation of Theorem 1 from \citet{singla2019bounding}]
  Let $\Kmat \in \Rbb^{\cout \times \cin \times k_1 \times k_2}$ be the kernel of a convolution layer $\layer_\Kmat$, then,
  \begin{equation}
    \lip{\layer_\Kmat} \leq \min \left\{ \sqrt{k_1 k_2} \norm{\Rmat}_2, \sqrt{k_2 k_2} \norm{\Smat}_2 \right\}
  \end{equation}
  where $\Rmat$ and $\Smat$ are matrices of size $k_1 \cout \times k_2 \cin$ and $k_2 \cout \times k_1 \cin$ defined as follows:
  \begin{align}
    \Rmat &= \leftmatrix
      % \Kmat_{0,0,:,:}       & \Kmat_{0,1,:,:}         & \cdots & \Kmat_{0,c_{in}-1,:,:} \\
      % \Kmat_{1,0,:,:}       & \Kmat_{1,1,:,:}         & \cdots & \Kmat_{1,c_{in}-1,:,:} \\
      % \vdots                & \vdots                  & \ddots & \vdots                 \\
      % \Kmat_{\cout-1,0,:,:} & \Kmat_{\cout-1,1,:,:} & \cdots & \Kmat_{\cout-1,\cin-1,:,:}
      (\Kmat)_{0,0}       & \cdots & (\Kmat)_{0,\cin-1} \\
      (\Kmat)_{1,0}       & \cdots & (\Kmat)_{1,\cin-1} \\
      \vdots              & \ddots & \vdots             \\
      (\Kmat)_{\cout-1,0} & \cdots & (\Kmat)_{\cout-1,\cin-1}
    \rightmatrix \\[0.5cm]
    \Smat &= \leftmatrix
      (\Kmat)_{0,0}^\top       & \cdots & (\Kmat)_{0,\cin-1}^\top \\
      (\Kmat)_{1,0}^\top       & \cdots & (\Kmat)_{1,\cin-1}^\top \\
      \vdots                   & \ddots & \vdots                  \\
      (\Kmat)_{\cout-1,0}^\top & \cdots & (\Kmat)_{\cin-1,\cout-1}^\top
    \rightmatrix
  \end{align}
  \removespace
\end{theorem}

In order to prove this result, \citet{singla2019bounding} built upon the work of \citet{sedghi2018singular} and have also only considered circulant convolutions (performed by doubly-block circulant matrices).
Instead of proposing a method to compute \emph{all} singular values of the equivalent doubly-block circulant matrix, their method is an upper-bound on the largest singular value of the Jacobian of the convolution. 
Because this method is independent of the input dimension, the computational complexity is substantially reduced compared to the approach of \citet{sedghi2018singular}, however, the reduction in computational complexity is at the expense of accuracy as we will show in~\Cref{chapter:ch5-lipschitz_bound}.
 

% => link between margin and robustness
% - \citet{tsuzuku2018lipschitz}: use power method but do not normalize

% =>> GLOBAL BOUND
%
% => global bound / semi-definite programming
% - \citet{scaman2018lipschitz}
% - \cite{fazlyab2019efficient}
% - \citet{latorre2020lipschitz}
%
% =>> POWER METHOD 
% => spectral normalization (they all use the power method \citet{golub2000eigenvalue})
% - first paper on spectral normalization \citet{gouk2018regularisation}
% - \citet{yoshida2017spectral,miyato2018spectral}: with power method
% - \citet{farnia2018generalizable}: with power method specific for convolutional layers
% - \citet{tsuzuku2018lipschitz}: use power method but do not normalize
%
% =>> WORK ON CONVOLUTION
%
% => orthogonal convolutions
% - \citet{cisse2017parseval}
% - \citet{li2019preventing}
% - \citet{wang2020Orthogonal} (orthogonal convolution)
%
% => Upper bound on convolution
% - \citet{sedghi2018singular}
% - \citet{singla2019bounding}


% The product of the Lipschitz constant of each layer is an upper-bound for the Lipschitz constant of the entire network, and it can be used as a surrogate to perform Lipschitz regularization.
% Since most common activation functions (such as ReLU) have a Lipschitz constant equal to one, the main bottleneck is to compute the Lipschitz constant of the underlying linear application which is equal to its maximal singular value.
% The work in this line of research mainly relies on the celebrated iterative algorithm by~\citet{golub2000eigenvalue} used to approximate the maximum singular value of a linear function.

% The last few years have witnessed a growing interest in Lipschitz regularization of neural networks, with the aim of improving their generalization~\cite{bartlett2017spectrally}, their robustness to adversarial attacks~\cite{tsuzuku2018lipschitz, farnia2018generalizable}, or their generation abilities (\eg for GANs: \citealt{miyato2018spectral,arjovsky2017wasserstein}).
% Unfortunately computing  the exact Lipschitz constant of a neural network is NP-hard~\cite{scaman2018lipschitz} and in practice, existing techniques such as~\citet{scaman2018lipschitz, NIPS2019_9319} or~\citet{latorre2020lipschitz} are difficult to implement for neural networks with more than one or two layers, which hinders their use in deep learning applications.

% To overcome this difficulty, most of the work has focused on computing the Lipschitz constant of \emph{individual layers} instead.
% The product of the Lipschitz constant of each layer is an upper-bound for the Lipschitz constant of the entire network, and it can be used as a surrogate to perform Lipschitz regularization.
% Since most common activation functions (such as ReLU) have a Lipschitz constant equal to one, the main bottleneck is to compute the Lipschitz constant of the underlying linear application which is equal to its maximal singular value.
% The work in this line of research mainly relies on the celebrated iterative algorithm by~\citet{golub2000eigenvalue} used to approximate the maximum singular value of a linear function.
% Although generic and accurate, this technique is also computationally expensive, which impedes its usage in large training settings. 













%%%%%%%%%%%%%%%%%%%%%%%%%%%%%%%%%%%%%%%%%%%%%%%%%%%%%%%%%%%%%%%%%%%%%%%%%%%%%%%%
\subsection{Discussion}
%%%%%%%%%%%%%%%%%%%%%%%%%%%%%%%%%%%%%%%%%%%%%%%%%%%%%%%%%%%%%%%%%%%%%%%%%%%%%%%%

We have presented state-of-the-art methods for regularizing the Lipschitz constant of neural networks with the aim to improve their robustness against adversarial attacks.
We have shown that the power method~\cite{golub2000eigenvalue} is a popular technique for approximating the maximal singular value of a matrix.
Multiple works in deep learning use this method in a wide variety of settings, for example, robustness \cite{farnia2018generalizable,tsuzuku2018lipschitz}, generalization~\cite{yoshida2017spectral,gouk2018regularisation} or to stabilize the training of Generative Adversarial Networks (GANs) \cite{miyato2018spectral}.
Despite a number of interesting results, using the power method is expensive and results in prohibitive training times. 
Other approaches to regularize the Lipschitz constant of neural networks have been proposed by~\citet{sedghi2018singular} and ~\citet{singla2019bounding}.
The method of~\citet{sedghi2018singular,singla2019bounding} exploits the properties of circulant matrices to approximate the maximal singular value of a convolutional layer.
Although interesting, theses method results in a loose approximation of the maximal singular value of a convolutional layer.
Our work is positioned at the intersection between these works, we will introduce a new approach for regularizing the Lipschitz constant of neural networks, that is more efficient than the power method and more accurate than methods relying on the structure of convolutions.






% %%%%%%%%%%%%%%%%%%%%%%%%%%%%%%%%%%%%%%%%%%%%%%%%%%%%%%%%%%%%%%%%%%%%%%%%%%%%%%%%
% \section{Position of the Contribution Regarding the State-of-the-Art}
% \label{section:ch3-position_of_the_contribution_regarding_the_state-of-the-art}
% %%%%%%%%%%%%%%%%%%%%%%%%%%%%%%%%%%%%%%%%%%%%%%%%%%%%%%%%%%%%%%%%%%%%%%%%%%%%%%%%
%
% In the first section, we have shown current methods and techniques for designing compact neural networks with structured matrices. 
% Our contributions on \emph{Deep Diagonal Circulant Neural Networks} are a direct follow-up to the work of~\citet{cheng2015exploration,sindhwani2015structured,moczulski2016acdc,thomas2018learning} focusing on compact neural networks with \emph{structured matrices}.
% More precisely, we extend the work of \citet{moczulski2016acdc} by training \emph{fully structured networks} (\ie, networks with structured layers only) hence demonstrating that diagonal circulant layers are able to model complex relations between inputs and outputs.
% Although, this diagonal circulant layers fit in the low displacement rank framework, we demonstrate much better performances in practice.
% Indeed, thanks to a solid theoretical analysis and thorough experiments, we were able to train deep (up to 40 layers) circulant neural networks, and apply, for the first time, this structured architecture in the context of large-scale video classification.
% This contrasts with previous experiments in which only one or a few dense layers were replaced inside a large redundant network such as VGG~\cite{simonyan2014very}.

% The first one considers the use of specific memory representation as well as efficient data structures.
% This technique is interesting and has the advantage to be applicable to any neural networks, therefore is complementary to other methods.
% The second method consists of using structured linear layers to reduce the number of parameters and leverage fast matrix-product algorithms. 
% Finally, recent works devised efficient and compact neural network architectures by tuning the width and depth parameters of neural networks.
% It is now clear that convolutional neural networks are state-of-the-art for image classification and detection.
% Moreover, recent architectures~\cite{tan2019efficientnet} have been found with architecture search algorithms and have very competitive results. 
% However, neural networks based on structured matrices (different than convolution) can still be interesting for other use cases or small-footprint deep learning like smartphone or IoT devices. 


% In the second section of this chapter, we have presented current methods for regularizing the Lipschitz constant of neural networks with the aim to improve their robustness against adversarial attacks.
% We have shown that the power method~\cite{golub2000eigenvalue} is a popular technique for approximating the maximal singular value of a matrix.
% Multiple works in deep learning use this method in a wide variety of settings, for example, robustness \cite{farnia2018generalizable,tsuzuku2018lipschitz}, generalization~\cite{yoshida2017spectral,gouk2018regularisation} or to stabilize the training of Generative Adversarial Networks (GANs) \cite{miyato2018spectral}.
% Despite a number of interesting results, using the power method is expensive and results in prohibitive training times. 
% Other approaches to regularize the Lipschitz constant of neural networks have been proposed by~\citet{sedghi2018singular} and ~\citet{singla2019bounding}.
% The method of~\citet{sedghi2018singular,singla2019bounding} exploits the properties of circulant matrices to approximate the maximal singular value of a convolutional layer.
% Although interesting, theses method results in a loose approximation of the maximal singular value of a convolutional layer.
% Our work is positioned at the intersection between these works, we will show that a new approach for regularizing the Lipschitz constant of neural networks is more efficient than the power method and more accurate than methods based on the structure of convolutions.


% A popular technique for approximating the maximal singular value of a matrix is the power method~\cite{golub2000eigenvalue}, an iterative algorithm which yields a good approximation of the maximum singular value when the algorithm is able to run for a sufficient number of iterations.
%
% \citet{yoshida2017spectral, miyato2018spectral} have used the power method to normalize the spectral norm of each layer of a neural network, and showed that the resulting models offered improved generalization performance and generated better examples when they were used in the context of GANs. 
% \citealt{farnia2018generalizable} built upon the work of ~\citet{miyato2018spectral} and proposed a power method specific for convolutional layers that uses the deconvolution operation and avoid the computation of the gradient.
% They used it in combination with adversarial training. 
% In the same vein, \citet{gouk2018regularisation} demonstrated that regularized neural networks using the power method also offered improvements over their non-regularized counterparts. 
% Furthermore, \citet{tsuzuku2018lipschitz} have shown that a neural network can be more robust to some adversarial attacks,  if the prediction margin of the network (\ie, the difference between the first and the second maximum logit) is higher than a minimum threshold that depends on the global Lipschitz constant of the network.
% Building on this observation, they use the power method to compute an upper bound on the global Lipschitz constant, and maximize the prediction margin during training.
% Finally, \citet{scaman2018lipschitz} have used automatic differentiation combined with the power method to compute a tighter bound on the global Lipschitz constant of neural networks.
% Despite a number of interesting results, using the power method is expensive and results in prohibitive training times. 
%
% Other approaches to regularize the Lipschitz constant of neural networks have been proposed by~\citet{sedghi2018singular} and ~\citet{singla2019bounding}.
% The method of~\citet{sedghi2018singular} exploits the properties of circulant matrices to approximate the maximal singular value of a convolutional layer.
% Although interesting, this method results in a loose approximation of the maximal singular value of a convolutional layer.
% Furthermore, the complexity of their algorithm is dependent on the convolution input which can be high for large datasets such as ImageNet.
% More recently, \citet{singla2019bounding} have successfully bounded the operator norm of the Jacobian matrix of a convolution layer by the Frobenius norm of the reshaped kernel.
% This technique has the advantage to be very fast to compute and to be independent of the input size but it also results in a loose approximation. 
%
% To build robust neural networks, \citet{cisse2017parseval} and ~\citet{li2019preventing} have proposed to constrain the Lipschitz constant of neural networks by using orthogonal convolutions.
% \citet{cisse2017parseval} use the concept of \emph{parseval tight frames}, to constrain their networks.
% \citet{li2019preventing} built upon the work of~\citet{cisse2017parseval} to propose an efficient construction method of orthogonal convolutions.  
% Also, recent work~\cite{fazlyab2019efficient,latorre2020lipschitz} has proposed a tight bound on the Lipschitz constant of the full network with the use of semi-definite programming.
% These works are theoretically interesting but lack scalability (\ie, the bound can only be computed on small networks).
%
% Finally, in parallel to the development of the results in this paper, we discovered that \citet{yi2020asymptotic} have studied the asymptotic distribution of the singular values of convolutional layers by using a related approach. However, this author does not investigate the robustness applications of Lipschitz regularization.



