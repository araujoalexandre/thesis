
The regularization of the Lipschitz constant of neural networks has been a growing interest in the training of neural networks.
Indeed, numerous research have shown that neural networks with a low Lipschitz constant exhibit better generalization~\cite{bartlett2017spectrally} and higher robustness to adversarial attacks~\cite{szegedy2013intriguing,tsuzuku2018lipschitz, farnia2018generalizable}.
The Lipschitz constant is a measure of the stability of the network, if the Lipschitz constant is high, the network will tend to be more sensitive to input perturbation.
Formally, the Lipschitz constant of a function is defined as follows: 
\begin{definition}[Lipschitz Constant]
  The Lipschitz constant with respect to the $\ell_2$-norm of a function $f: \Rbb^n \rightarrow \Rbb^m$ is defined as follows:
  \begin{equation}
    \lip{f} \triangleq \sup_{\substack{\xvec, \yvec \in \Rbb^n \\ \xvec \neq \yvec}} \frac{\norm{f(\xvec) - f(\yvec)}_2}{\norm{\xvec - \yvec}_2}
  \end{equation}
  If $\lip{f}=k$, we denote the function $f$ as $k$-Lipschitz.
  \label{definition:ch3-lipschitz_constant}
\end{definition}
\noindent
Intuitively, the Lipschitz constant $k$ of a function $f$ is a bound on the slope of $f$.
Meaning, if the input changes by $\epsilon$, the output changes by at most $k\epsilon$.
Therefore, the Lipschitz constant of a function can also be expressed using the differential operator as follows:
\begin{theorem}[Rademacher Theorem 3.1.6 of \citet{federer2014geometric}]
  If $f: \Rbb^n \rightarrow \Rbb^m$ is a locally Lipschitz continuous function, then $f$ is differentiable almost everywhere. Moreover, if $f$ is Lipschitz continuous, then
  \begin{align}
    \lip{f} = \sup_{\xvec \in \Rbb^n} \norm{\mathrm{D}_\xvec f(\xvec)}_2
  \end{align}
  where $\mathrm{D}_\xvec$ is the differential operator of $f$ at $\xvec$.
  \label{theorem:ch3-lipschitz_differential_op}
\end{theorem}


\citet{tsuzuku2018lipschitz} have studied the relationship between the robustness of a neural network and its Lipschitz constant. 
By the definition of the Lipschitz constant, we have the following:
\begin{equation}
  \norm{N(\xvec) - N(\xvec + \adv)}_2 \leq \lip{N} \norm{\adv}_2
\end{equation}
Let us denote the margin of the neural network $N$ with respect to the tuple $(\xvec, y)$ as follows:
\begin{equation}
  \mathcal{M}^{N}_{(\xvec, y)} \triangleq N(\xvec)_y - \max_{i \neq y} \left( N(\xvec) \right)_i
\end{equation}
This margin has been studied in relationship to generalization bounds \cite{langford2002pac,bartlett2017spectrally,neyshabur2018pacbayesian}.
Then, we have the following proposition:
\begin{proposition}[\citet{tsuzuku2018lipschitz}]
  \begin{equation}
    \mathcal{M}^{N}_{(\xvec, y)} \geq \sqrt{2} \lip{N} \norm{\adv}_2 \quad \Longrightarrow \quad \mathcal{M}^{N}_{(\xvec + \adv, y)} \geq 0
    \label{equation:ch3-margin_guarded_area}
  \end{equation}
\end{proposition}
\noindent
If the inequality on the right-hand side of \Cref{equation:ch3-margin_guarded_area} is verified then the adversarial margin is positive, \ie, the network correctly predicts the label. 
From this proposition, we can conclude that for a given neural network with specific margins, a lower Lipschitz constant allow for an increase in robustness. 
Note that the margin is already maximized in a multi-class setting with the cross-entropy loss as stated in~\citet{hein2017formal}.
A multitude of work have tried to reduce the Lipschitz constant in order to improve adversarial robustness.
However, \citet{scaman2018lipschitz} have shown that computing the exact Lipschitz constant of a neural network is NP-hard.

\begin{problem}[\citet{scaman2018lipschitz}]
  Let us define the problem associated to the exact computation of the Lipschitz constant of a $2$-layer neural network with $\relu$ activation:
  \begin{itemize}[topsep=0pt,noitemsep]
    \item[] \textbf{Input:} Two matrices $\Wmat^{(1)} \in \Rbb^{l \times n}$ and $\Wmat^{(2)} \in \Rbb^{m \times l}$, and a constant $c \geq 0$.
    \item[] \textbf{Question:} Let $N = \Wmat^{(2)} \circ \rho \circ \Wmat^{(1)}$ where $\rho$ is the $\relu$ activation function. \emph{Is the Lipschitz constant $\lip{N} \leq c$ ?}
  \end{itemize}
  \label{problem:ch3-lipschitz_computation}
\end{problem}


\noindent
The following theorem shows that, even for shallow neural network, exact Lipschitz computation is not achievable in polynomial time (assuming that $\mathbf{P} \neq \textbf{NP}$):
\begin{theorem}[\citet{scaman2018lipschitz}]
  \Cref{problem:ch3-lipschitz_computation} is \textbf{NP}-hard.
\end{theorem}


\noindent
To overcome this difficulty, researchers have relied on devising tight upper bound of the Lipschitz constant.
For example, \citet{scaman2018lipschitz} have shown that the Lipschitz constant of neural network $N$ can be explicitly formulated using \Cref{theorem:ch3-lipschitz_differential_op} and the chain rule:
\begin{equation}
  \lip{\nn} = \sup_{x \in \Rbb^n} \norm{\Wmat^{(p)} \diag(\rho'_\depth(\theta_\depth)) \dots \Wmat^{(2)} \diag(\rho'_1(\theta_1)) \Wmat^{(1)}}_2,
  \label{equation:ch3-decomposition_jacobian_lipschitz}
\end{equation}
where $\theta_i = \phi^{\act_i}_{\Wmat^{(i)}, \bvec^{(i)}} \circ \cdots \circ \phi^{\act_1}_{\Wmat^{(1)}, \bvec^{(1)}}(\xvec)$ is the intermediate output after $i$ layers.
The Lipschitz of neural network $N$ can then be upper bounded as follows:
\begin{align}
  \lip{\nn} &\leq \max_{\forall i,\ \sigma_i \in [0, 1]^{w^{(i+1)}}} \norm{\Wmat^{(\depth)} \diag(\sigma_{\depth-1}) \dots \diag(\sigma_1) \Wmat^{(1)}}_2 \notag \\
  &\leq \max_{\forall i,\ \sigma_i \in [0, 1]^{w^{(i+1)}}} \norm{ \pmb{\Sigma}^{(\depth)} \Vmat^{(\depth)\top} \diag(\sigma_{\depth-1}) \dots \diag(\sigma_1) \Umat^{(1)} \pmb{\Sigma}^{(1)}}_2 \notag \\
  &\leq \prod_{i=1}^{\depth-1} \max_{\sigma_i \in [0, 1]^{w^{(i+1)}}} \norm{\widetilde{\pmb{\Sigma}}^{(i+1)} \Vmat^{(i+1)\top} \diag(\sigma_{i+1}) \Umat^{(i)} \widetilde{\pmb{\Sigma}}^{(i)}}_2 
\end{align}
where $\widetilde{\pmb{\Sigma}}^{(i)} = \pmb{\Sigma}^{(i)}$ if $i \in \{1, \depth\}$ and $\widetilde{\pmb{\Sigma}}^{(i)} = {\pmb{\Sigma}^{(i)}}^{1/2}$ otherwise.
The first inequality is due to the fact that the derivatives of the activation functions are bounded, \ie, $\rho_i(\xvec) \in [0, 1]^{w^{(i+1)}}$, the second inequality is obtained by decomposing each weight matrix $\Wmat^{(i)}$ with the \emph{Singular Value Decomposition} such that $\Wmat^{(i)} = \Umat^{(i)} \pmb{\Sigma}^{(i)} \Vmat^{(i)\top}$ and finally, the last inequality is due to the submultiplicativity of the operator norm.
Although accurate, this bound is still computationally expensive to compute due to the singular value decomposition and the optimization for each layer. 
In the same line of research, recent work~\cite{fazlyab2019safety,fazlyab2019efficient,latorre2020lipschitz} has proposed a tight bound on the Lipschitz constant of the full network with the use of semi-definite programming.
More precisely, \citet{fazlyab2019efficient} have demonstrated the following result:
\begin{theorem}[Lipschitz bounds \citet{fazlyab2019efficient}]
  Consider a neural network $N: \Rbb^n \rightarrow \Rbb^m$ such that $N(\xvec) = \Wmat^{(1)} \rho(\Wmat^{(0)} \xvec + \bvec^{(0)}) + \bvec^{(1)}$.
  Suppose the activation function $\rho$ is \emph{slope-restricted} in the sector $[\alpha,\beta]$, more precisely:
  \begin{equation}
    \alpha \leq \frac{\rho(y) - \rho(x)}{y-x} \leq \beta \quad \forall x,y \in \mathbb{R}. 
  \end{equation}
  Define the set $\mathcal{T}_{n}$ as the following:
  \begin{equation*}
    \mathcal{T}_n = \{\Tmat \in \mathbb{S}^n \mid \Tmat = \sum_{i=1}^{n} \lambda_{ii} \evec^{(i)} \evec^{(i)\top} + \sum_{1 \leq i<j \leq n} \lambda_{ij}(\evec^{(i)} - \evec^{(j)})(\evec^{(i)}-\evec^{(j)})^\top, \lambda_{ij} \geq 0 \}.
  \end{equation*}
  where $\mathbb{S}$ is the set of all symmetric matrices of size $n \times n$.
  Suppose there exists a constant $c>0$ such that the matrix inequality
  \begin{align}
    \Mmat(c,\Tmat) \triangleq
      \leftmatrix
      -2\alpha \beta \Wmat^{(0)\top} \Tmat \Wmat^{(0)} - c \Imat_n & (\alpha+\beta) \Wmat^{(0)\top} \Tmat  \\
      (\alpha+\beta) \Tmat \Wmat^{(0)} & -2\Tmat+\Wmat^{(1)\top} \Wmat^{(1)}
      \rightmatrix
      \leq 0,
  \end{align}
  holds for some $\Tmat \in \mathcal{T}_{n}$. Then $\norm{N(\xvec)-N(\yvec)}_2 \leq \sqrt{c} \norm{\xvec-\yvec}_2$ for all  $\xvec,\yvec \in \mathbb{R}^n$.
  \label{theorem:ch3-lipschite_semidefinite_programming}
\end{theorem}
\noindent
From \Cref{theorem:ch3-lipschite_semidefinite_programming}, the constant $c$ is an upper bound on the Lipschitz constant of the network.
The authors proposed to find the tightest bound by solving the following optimization problem (Semi-Definite Program):
\begin{align} \label{eq: SDP one layer}
  \textrm{minimize} \quad c \quad \text{ subject to} \quad \Mmat(c,\Tmat) \leq 0 \quad \text{and} \quad \Tmat \in \mathcal{T}_{n},
\end{align}
where the decision variables are $(c,\Tmat) \in \Rbb_+ \times \mathcal{T}_n$.
Note that $\Mmat(c,\Tmat)$ is linear in $c$ and $\Tmat$ and the set $\mathcal{T}_n$ is convex.
Although, these works on devising a global bound on the Lipschitz constant of a neural network are theoretically interesting, they lack scalability, they can only be computed on small networks and cannot be used during the training of a neural network for regularization purposes.

\drawstar

\noindent
In order to constraint the Lipschitz constant of neural networks, researchers have devised techniques to reduce the Lipschitz constant of \emph{individual layers} instead. 
The global Lipschitz of a neural network can easily be upper bound by the product of the spectral norm of each weight matrix as follows:
\begin{proposition}[\citet{scaman2018lipschitz}] Let $N$ be a neural network of $\depth$ layers with 1-Lipschitz activation functions (\eg ReLU,
  Leaky ReLU, Tanh, Sigmoid, etc.), then, the Lipschitz constant of the neural network can be upper bounded as follows:
  \begin{equation}
    \lip{N} \leq \prod_{i=1}^\depth \norm{\Wmat^{(i)}}_2 \enspace,
    \label{equation:ch3-naive_upper_bound_lipschitz}
  \end{equation}
  where $\Wmat^{(i)}$ are the weights matrices of the neural network.
  \label{proposition:ch3-naive_upper_bound_lipschitz}
\end{proposition}

\begin{remark}
  The Lipschitz constant of a layer $\Phi_{\Wmat, \bvec}^\rho$ (with a 1-Lipschitz activation function) is equal to the spectral norm of the matrix $\Wmat$ (largest singular value).
  Let $\Phi_{\Wmat, \bvec}^\rho: \Rbb^n \rightarrow \Rbb^m$ such that $\Phi_{\Wmat, \bvec}^\rho = \rho(\Wmat \xvec + \bvec)$ then by definition of the Lipschitz constant (\Cref{definition:ch3-lipschitz_constant}) and of the operator norm, we have:
  \begin{equation}
    \lip{\Phi_{\Wmat, \bvec}^\rho} = \sup_{\substack{\xvec \in \Rbb^n \\ \xvec \neq 0}} \frac{\norm{\Wmat \xvec}_2}{\norm{\xvec}_2} = \norm{\Wmat}_2
  \end{equation}
\end{remark}

The trivial bound given by the product of layer-wise Lipschitz constants in \Cref{equation:ch3-naive_upper_bound_lipschitz} is known to be loose and pessimistic~\citet{combettes2019lipschitz}.
Furthermore, we can show that reducing the Lipschitz of each layer independently does not imply that the global Lipschitz constant of the network will be reduced. 
\begin{proposition}
  Let $N_1(\xvec) = \Amat^{(2)} \rho(\Amat^{(1)} \xvec)$ and $N_2(\xvec) = \Bmat^{(2)} \rho(\Bmat^{(1)} \xvec)$ where $\rho$ is the $\relu$ activation function, then if $\norm{\Amat^{(1)}}_2 \leq \norm{\Bmat^{(1)}}_2$ and $\norm{\Amat^{(2)}}_2 \leq \norm{\Bmat^{(1)}}_2$, this does not implies that $\lip{N_1} \leq \lip{N_2}$.
  \label{proposition:ch3-limit_bound_lipschitz}
\end{proposition}
\begin{proof}[\Cref{proposition:ch3-limit_bound_lipschitz}]
  Let
  \begin{align*}
    \Amat^{(1)} &= \leftmatrix 
      \phantom{+}0 & -1 \\ -1 & \phantom{+}0
    \rightmatrix \quad
    \Amat^{(2)}  = \leftmatrix
      -1 & -1 \\ -1 & \phantom{+}0
    \rightmatrix \\
    \Bmat^{(1)} &= \leftmatrix
      \phantom{+}0 & \phantom{+}0 \\ \phantom{+}0 & -1
    \rightmatrix \quad
    \Bmat^{(2)} = \leftmatrix
      -1 & -1 \\ -1 & -1
    \rightmatrix
  \end{align*}
  then: \vspace{-0.5cm}
  \begin{equation*}
    \norm{\Amat^{(1)}}_2 = 1,\ \norm{\Amat^{(2)}}_2 = \sqrt{2}
    \quad \text{and} \quad
    \norm{\Bmat^{(1)}}_2 = 1,\ \norm{\Bmat^{(2)}}_2 = 2
  \end{equation*}
  From \Cref{theorem:ch3-lipschitz_differential_op} and the chain rule, the Lipschitz constant of the networks $N_1$ and $N_2$ can be expressed as follows:
  \begin{align*}
    \lip{N_1} &= \sup_{\xvec \in [0, 1]^2} \norm{\Amat^{(2)} \diag\left(\xvec\right) \Amat^{(1)}}_2 \\
    \lip{N_2} &= \sup_{\xvec \in [0, 1]^2} \norm{\Bmat^{(2)} \diag\left(\xvec\right) \Bmat^{(1)}}_2
  \end{align*}
  It is easy to verify that:
  \begin{equation*}
    \lip{N_1} = \frac{1 + \sqrt{5}}{2} \approx 1.618 \quad \text{and} \quad \lip{N_2} = \sqrt{2} \approx 1.414
  \end{equation*}
  which concludes the proof.
\end{proof}

\noindent
\citet{huster2018limitations} have shown several limitations on the expressive power of neural networks where the product of layer-wise Lipschitz constants is constraint.
However, the bound in \Cref{equation:ch3-naive_upper_bound_lipschitz} appear in multiple generalization bound~\cite{neyshabur2017,bartlett2017spectrally,golowich2018} and adversarial generalization~\cite{farnia2018generalizable} which could mean that reducing the bound would improve the generalization capabilities of neural networks.


% the upper bound presented in Section 3 appears in multiple generalisation bounds (Neyshabur,
% 2017; Bartlett et al., 2017; Golowich et al., 2018), and we show empirically in
% this paper that it is an effective aide for controlling the generalisation performance of a deep network.

% This observation as led researchers to develop several techniques to constraint the Lipschitz constant of a neural network.

Based on this theoretical insight, researchers have developed several techniques to constraint the Lipschitz constant of each layer in order to improve the generalization and robustness of neural networks.
A technique to enforce 1-Lipschitz layers is to impose or promote an orthogonality constraint on the weight matrices.
A square orthogonal matrix $\Mmat$ is a matrix whose columns and rows are orthogonal unit vectors and all eigenvalues are equal to 1.
\citet{cisse2017parseval} and more recently \citet{wang2020orthogonal,huang2020controllable} have proposed to minimize the following term:
\begin{equation}
  \frac{\beta}{2} \norm{\Wmat^\top \Wmat - \Imat}_2  \enspace, 
\end{equation}
to promote the orthogonality constraint.
On the other hand, \citet{li2019preventing} used the orthogonal projection proposed by \citet{kautsky1994matrix} and \citet{xiao2018dynamical} to build convolutional neural networks with orthogonal convolutions.
All works show an increase in robustness against adversarial attacks due to the decrease in network sensitivity.
However, the expressivity of the network with orthogonal weight matrices is clearly limited, thus the natural accuracy obtained by these networks is far from being state-of-the-art. 


\begin{algorithm}[ht]
  \caption{Power method for producing the largest singular value, $\sigma_1$, of a non-square matrix, $\Wmat$ \cite{gouk2018regularisation,golub2000eigenvalue}}
  \begin{algorithmic}[1]
    \Require{affine function $f(\xvec) = \Wmat \xvec + \bvec$, number of iteration $N$}
    \Ensure{approximation of the Lipschitz constant $\lip{f}$}
    \State Randomly initialise $\xvec$
    \For{$i = 1$ \textbf{to} $N$}
      \State $\xvec \gets \Wmat^\top \Wmat \xvec / \norm{\xvec}_2$
    \EndFor
    \State \textbf{return} $\norm{\Wmat \xvec}_2 / \norm{\xvec}_2$
  \end{algorithmic}
  \label{algorithm:ch3-power_method}
\end{algorithm}

\begin{algorithm}[ht]
  \caption{Convolutional power iteration \cite{farnia2018generalizable}}
  \begin{algorithmic}[1]
    \Require{2d-convolution function $f: \Rbb^{n \times n} \rightarrow \Rbb^{m \times m}$ with kernel $k$, 2d-convolution-transpose\protect\footnotemark~function $g: \Rbb^{n \times n} \rightarrow \Rbb^{m \times m}$ with kernel $k$ number of iteration $N$}
    \Ensure{approximation of the Lipschitz constant $\lip{f}$}
    \State Initialize $\xvec$ with a random vector matching the shape of the convolution input
    \For{$i = 1$ \textbf{to} $N$}
      \State $\xvec \gets f(\xvec) / \norm{f(\xvec)}_2 $
      \State $\xvec \gets g(\xvec) / \norm{g(\xvec)}_2$
    \EndFor
    \State \textbf{return} $\norm{f(\xvec)}_2 / \norm{\xvec}_2$
  \end{algorithmic}
  \label{algorithm:ch3-power_method_generic}
\end{algorithm}
\footnotetext{As for the 2d-convolution function, the 2d-convolution-transpose is efficiently implemented in recent Deep Learning framework \cite{paszke2019pytorch,tensorflow2015-whitepaper}} 


Another technique, called \emph{Spectral Normalization}, consists in normalizing each weight matrix by its largest singular value, thus imposing each layer to be 1-Lipschitz.
This technique leads the network also having a Lipschitz constant of 1 by \Cref{proposition:ch3-naive_upper_bound_lipschitz}.
\citet{yoshida2017spectral} were the first to propose such method to improve the generalization of neural networks.
They use the power method~\cite{golub2000eigenvalue} to compute the largest singular value of each weight matrix.

The power method (also known as the power iteration) is an iterative algorithm: given a matrix, the algorithm provably converges (with a sufficient number of iterations) to the largest singular value of the matrix.
The rate of convergence of the algorithm depends on the ratio of the second-largest singular value to the largest singular value which can lead in certain cases to slow convergence.
The pseudocode of the power method is presented in \Cref{algorithm:ch3-power_method}.

In the context of deep learning, given that the algorithm is performed for each layer of the network at each step of the training, a small number of iterations is set in order to keep the overall computations feasible.
Spectral Normalization has been shown to be an effective method for improving generalization and robustness against adversarial attacks \cite{miyato2018spectral,gouk2018regularisation,farnia2018generalizable}.
\citet{farnia2018generalizable,ryu2019plug} extended the power method to convolutional layers where matrix $\Wmat$ is not explicitly constructed.
They proposed a similar algorithm (\Cref{algorithm:ch3-power_method_generic}) for convolutional function.



\drawstar

\noindent
In the context of convolutional neural networks, the power method is not the only technique available for approximating the largest singular value (Lipschitz constant) of a convolutional layer.
Several work have devise bound or approximation on the largest singular value of convolutional layers by exploiting the \emph{structure} of the convolution operation \cite{sedghi2018singular,bibi2019deep,singla2019bounding}.








% \citet{sedghi2018singular} have observed that a doubly-block Toeplitz matrix can be approximated by a 
%
% \citet{sedghi2018singular} have exploited the properties 

% \citet{gray2006toeplitz} have observed that band-Toeplitz matrices can be `approximated' by band-circulant matrices.
% This `approximation' is formalized by a mathematical concept called \emph{}j

To approximate the singular values of a convolutional layer, \citet{sedghi2018singular} have exploited the properties of doubly-block circulant matrices (\ie, a circulant block matrix where each block is also a circulant matrix).
Indeed, a doubly-block circulant matrice is the matrix representation of linear convolutional with circulant padding.
In their work, \citet{sedghi2018singular} assume that the properties of doubly-block circulant matrices are `close' to the properties of a doubly-block Toeplitz matrix.

To compute the singular values of doubly-block circulant matrices, \citet{sedghi2018singular} have demonstrated the following result:
\begin{theorem}[Theorem 5 of \citet{sedghi2018singular}]
  Let $\Amat$ be a doubly-block circulant matrix such that:
  \begin{equation*}
    \Amat = \leftmatrix
      \Cmat_0     & \Cmat_{n-1} & \Cmat_{n-2} & \cdots  & \cdots      & \Cmat_1     \\
      \Cmat_1     & \Cmat_0     & \Cmat_{n-1} & \ddots  &             & \vdots      \\
      \Cmat_2     & \Cmat_1     & \ddots      & \ddots  & \ddots      & \vdots      \\
      \vdots      & \ddots      & \ddots      & \ddots  & \Cmat_{n-1} & \Cmat_{n-2} \\
      \vdots      &             & \ddots      & \Cmat_1 & \Cmat_{0}   & \Cmat_{n-1} \\
      \Cmat_{n-1} & \cdots      & \cdots      & \Cmat_2 & \Cmat_{1}   & \Cmat_0
    \rightmatrix
  \end{equation*}
  where $\Cmat_i = \circulant({\cvec_i}),\ \forall i \in \{0,\dots,n-1\}$.
  Let $\Kmat = \leftmat \cvec_0, \cvec_1, \cdots, \cvec_{n-1} \rightmat^\top$ then, the singular values of the doubly-block circulant matrix $\Amat$ are the magnitudes of the entries of $\Umat_n^\top \Kmat \Umat_n$.
  \label{theorem:ch3-singular_values_doubly_block_circulant}
\end{theorem}

\noindent
To prove \Cref{theorem:ch3-singular_values_doubly_block_circulant}, \citet{sedghi2018singular} used the following \Cref{theorem:ch2-diaginalization_doubly_block_circulant_matrix} presented in \Cref{chapter:ch2-background}.
The main advantage of this approach is that the singular values of a doubly-block circulant matrix can be computed with the Fast Fourier transform algorithm (see \Cref{subsection:ch2-circulant_matrices}) which offers a reduced complexity compare to classical approaches for computing the singular values of a matrix. 
However, this approach exhibits several limitations.
First, this method results in a loose approximation of the maximal singular value of a convolutional layer which does not use the circulant padding which is often the case in practical settings.
Also, the complexity of their algorithm is dependent on the size of the input which can be high for large datasets.
Finally, for multi-channel convolution, their method requires the computation of the spectral norm of $n^2$ matrices each of size $\cin \times \cout$ as stated in the following theorem:

\begin{theorem}[Theorem 6 of \citet{sedghi2018singular}] 
  Let $\Mmat$ be the matrix encoding the linear transformation computed by a multi-channel convolutional layer.
  Let $\Kmat \in \Rbb^{\cin \times \cout n \times n}$ such that $(\Kmat)_{i,j}$ for all $i,j$ be constructed as in \Cref{theorem:ch3-singular_values_doubly_block_circulant}, 
  Let $\widetilde{\Kmat}_{i,j} = \Umat^\top \leftmat \Kmat \rightmat_{i,j} \Umat_n $ and define the following operator matrix 
  \begin{equation}
    \Pmat(i,j) = \leftmatrix 
    \leftmat \widetilde{\Kmat}_{0,0} \rightmat_{i, j} & \cdots & \leftmat \widetilde{\Kmat}_{0, \cout-1} \rightmat_{i, j} \\
    \vdots & & \vdots \\
    \leftmat \widetilde{\Kmat}_{\cin-1, 0} \rightmat_{i, j} & \cdots & \leftmat \widetilde{\Kmat}_{\cin-1, \cout-1} \rightmat_{i, j}
    \rightmatrix
  \end{equation}
  Then
  \begin{equation}
    \sigma(\Mmat) = \bigcup_{i, j = 0}^{n-1} \sigma \left(  \Pmat(i,j) \right).
  \end{equation}
\end{theorem}



% , their method requires the computation of the spectral norm of n
% 2 matrices (each matrix of
% size cout × cin) for every convolution layer making it impractical to use during training.


In the same line of research, \citet{singla2019bounding} have used the properties of convolution to devised several bound on the singular values of convolution layers.
Recall from \Cref{subsection:ch2-the_structure_of_the_convolution_operator} that a convolution kernel is a 4 dimensional tensor of size $\cout \times \cin \times k_1 \times k_2$.
\citet{singla2019bounding} have demonstrated that the largest singular value (Lipschitz constant) of a convolutional layers $\Phi_\Kmat$ parameterized by a kernel $\Kmat$ can be upper-bound as follows:

\begin{theorem}[Reformulation of Theorem 1 of \citet{singla2019bounding}]
  Let $\Kmat \in \Rbb^{\cout \times \cin \times k_1 \times k_2}$ be the kernel of a convolutional layer $\Phi_\Kmat$ then,  
  \begin{equation}
    \lip{\Phi_\Kmat} \leq \min \left\{ \sqrt{k_1 k_2} \norm{\Rmat}_2, \sqrt{k_2 k_2} \norm{\Smat}_2 \right\}
  \end{equation}
  where $\Rmat$ and $\Smat$ are matrices of size $k_1 \cout \times k_2 \cin$ and $k_2 \cout \times k_1 \cin$ respectively and are a `reshape' of the tensor kernel $\Kmat$.
\end{theorem}


On order to prove this result, \citet{singla2019bounding} build upon the work of \citet{sedghi2018singular} and have also only considered circulant convolution (performed by doubly-block circulant matrices).
Instead of proposing a method to compute \emph{all} singular values of the equivalent doubly-block circulant matrix, their method is an upper-bound on the largest singular value of the Jacobian of the convolution. 
Because this method is independent of the input dimension the computational complexity is substantially reduced compare to the approach of \citet{sedghi2018singular}, however, the reduction in computational complexity is at the expense of the accuracy.
 







% Fortunately, the structures of the linear transformation matrices can be exploited to reduce the computational complexity of SVDs.
% Of particular relevance is the work by \citet{sedghi2018the,bibi2018deep}, in which the linear convolutional layer is treated as a circular one by a ``wrapping round'' operation.
% In doing so, the linear transformation matrices are endowed with a circulant structure, by which efficient methods were proposed to compute a circular approximation of the convolutional layers with substantially reduced complexity.

% To further reduce computational complexity, upper bounds of spectral norm of the circular convolutional layers were derived in \citet{Singla2019} at the expense of degraded accuracy.


% Fortunately, the structures of the linear transformation matrices can be exploited to reduce the computational complexity of SVDs.
% Of particular relevance is the work by \citet{sedghi2018the,bibi2018deep}, in which the linear convolutional layer is treated as a circular one by a ``wrapping round'' operation.
% In doing so, the linear transformation matrices are endowed with a circulant structure, by which efficient methods were proposed to compute a circular approximation of the convolutional layers with substantially reduced complexity.
% To further reduce computational complexity, upper bounds of spectral norm of the circular convolutional layers were derived in \citet{Singla2019} at the expense of degraded accuracy.
%
% As a matter of fact, such a ``wrapping round'' operation is not always endowed in many convolutional layers, for which a linear, rather than circular, convolutional operation is applied.
% With such a linear convolution, the linear transformation matrix has a Toeplitz structure, which includes the circulant one as a special case.
% This has been pointed out by a number of previous works, e.g., \citet{goodfellow2016deep,wang2019orthogonal,appuswamy2016structured}, that the two-dimensional single-channel convolutional layer results in a doubly block Toeplitz matrix.
% A question then arises as to how close is the circular approximation to the exact linear Toeplitz case.

% Although some theoretical analysis bounded the gap between large Toeplitz and circulant matrices \citet{ZhuTIT2017}, it seems only applied to Hermitian matrices (or symmetric for real matrices). The linear transformation matrices of linear convolution are {\em asymmetric} real matrices, which are {\em non-Hermitian} matrices.
% \citet{zhu2017asymptotic}




% this concept of \emph{parseval tight frames} \cite{kovacevic2008introduction}, to constrain their networks.
% More precisly, they folowing projection that they perfom after each gradient update:
% \begin{equation}
%   \Wmat \leftarrow (1 + \beta) \Wmat + \beta \Wmat \Wmat^T \Wmat 
% \end{equation}
% which is equivalent to apply one update for minimizing the following term: $\frac{\beta}{2} \norm{\Wmat^\top \Wmat - \Imat}_2$.
%
% Orthogonal Convolutional Neural Networks \citet{wang2020orthogonal}
% Controllable Orthogonalization in Training DNNs \citet{huang2020controllable}
%
% \citet{li2019preventing} built upon the work of~\citet{cisse2017parseval} to propose an efficient construction method of orthogonal convolutions.  
%
% We first summarize the orthogonal convolution representations from Kautsky and Turcajová [27] and Xiao et al. [50]
% [27] A matrix approach to discrete wavelets \citet{kautsky1994matrix}
% [50] 10000 layers \citet{xiao2018dynamical}


% \citet{miyato2018spectral} were the first to propose such method in the context of Generative Adversarial Networks, 
% - first paper on spectral normalization \citet{gouk2018regularisation}
% - \citet{yoshida2017spectral,miyato2018spectral}: with power method
% - \citet{farnia2018generalizable}: with power method specific for convolutional layers

% => Power method \citet{golub2000eigenvalue}
%
% =>> Enforcing Lipschitz constraint by normalization 
% => \citet{miyato2018spectral}: normalization with power method for Generative Adversarial Networks
% => \citet{gouk2018regularisation}: normalization with power method for generalization
% => \citet{farnia2018generalizable}: normalization with power method for adversarial robustness 
%
% =>> Enforcing Lipschitz constraint by orthogonalization
% => \citet{cisse2017parseval}
% => \citet{li2019preventing}
%
% =>> Regularization: explain problem of regulzarization (\citet{}) but is justified by Adv Generalization Bartlett + Farnia 
% => \citet{yoshida2017spectral}: regularization with power method for generalization
% => \citet{wang2020orthogonal} (orthogonal convolution)

% => Upper bound on convolution
% - \citet{sedghi2018singular}
% - \citet{singla2019bounding}

% Several recent papers have utilized this concept or an extension of it to additional layer types. [14] uses it to analyze the theoretical sensitivity of deep neural
% networks. [4] and [12] enforce constraints on the singular values of matrices as
% a way of increasing robustness to existing attacks. Finally, [16] penalizes the
% spectral norms of matrices and uses equation 6 to compute a Lipschitz constant
% for the network.

% [4] Parseval networks: Improving robustness to adversarial examples
% [12] L2-nonexpansive neural networks
% [16] Lipschitz-margin training: Scalable certification of perturbation invariance for deep neural networks

% =>> GLOBAL BOUND
% =>> POWER METHOD 
% =>> WORK ON CONVOLUTION
%   => orthogonal convolutions
%   => Upper bound on convolution

% \todotext{normalization of weight matrices to force Lipschitz 1}
% \citet{farnia2018generalizable,miyato2018spectral,yoshida2017spectral,li2019preventing}






% => link between margin and robustness
% - \citet{tsuzuku2018lipschitz}: use power method but do not normalize

% =>> GLOBAL BOUND
%
% => global bound / semi-definite programming
% - \citet{scaman2018lipschitz}
% - \cite{fazlyab2019efficient}
% - \citet{latorre2020lipschitz}
%
% =>> POWER METHOD 
% => spectral normalization (they all use the power iteration method \citet{golub2000eigenvalue})
% - first paper on spectral normalization \citet{gouk2018regularisation}
% - \citet{yoshida2017spectral,miyato2018spectral}: with power method
% - \citet{farnia2018generalizable}: with power method specific for convolutional layers
% - \citet{tsuzuku2018lipschitz}: use power method but do not normalize
%
% =>> WORK ON CONVOLUTION
%
% => orthogonal convolutions
% - \citet{cisse2017parseval}
% - \citet{li2019preventing}
% - \citet{wang2020Orthogonal} (orthogonal convolution)
%
% => Upper bound on convolution
% - \citet{sedghi2018singular}
% - \citet{singla2019bounding}


% The product of the Lipschitz constant of each layer is an upper-bound for the Lipschitz constant of the entire network, and it can be used as a surrogate to perform Lipschitz regularization.
% Since most common activation functions (such as ReLU) have a Lipschitz constant equal to one, the main bottleneck is to compute the Lipschitz constant of the underlying linear application which is equal to its maximal singular value.
% The work in this line of research mainly relies on the celebrated iterative algorithm by~\citet{golub2000eigenvalue} used to approximate the maximum singular value of a linear function.

% The last few years have witnessed a growing interest in Lipschitz regularization of neural networks, with the aim of improving their generalization~\cite{bartlett2017spectrally}, their robustness to adversarial attacks~\cite{tsuzuku2018lipschitz, farnia2018generalizable}, or their generation abilities (\eg for GANs: \citealt{miyato2018spectral,arjovsky2017wasserstein}).
% Unfortunately computing  the exact Lipschitz constant of a neural network is NP-hard~\cite{scaman2018lipschitz} and in practice, existing techniques such as~\citet{scaman2018lipschitz, NIPS2019_9319} or~\citet{latorre2020lipschitz} are difficult to implement for neural networks with more than one or two layers, which hinders their use in deep learning applications.

% To overcome this difficulty, most of the work has focused on computing the Lipschitz constant of \emph{individual layers} instead.
% The product of the Lipschitz constant of each layer is an upper-bound for the Lipschitz constant of the entire network, and it can be used as a surrogate to perform Lipschitz regularization.
% Since most common activation functions (such as ReLU) have a Lipschitz constant equal to one, the main bottleneck is to compute the Lipschitz constant of the underlying linear application which is equal to its maximal singular value.
% The work in this line of research mainly relies on the celebrated iterative algorithm by~\citet{golub2000eigenvalue} used to approximate the maximum singular value of a linear function.
% Although generic and accurate, this technique is also computationally expensive, which impedes its usage in large training settings. 











