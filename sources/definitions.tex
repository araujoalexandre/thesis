
%%%%%%%%%%%%%%%%%%%%%%%%%%%%%%%%%%%%%%%%%%%%%%%%%%%%%%%%%%%%%%%%%%%%%%%%%%%%%%%
% Basic packages
%%%%%%%%%%%%%%%%%%%%%%%%%%%%%%%%%%%%%%%%%%%%%%%%%%%%%%%%%%%%%%%%%%%%%%%%%%%%%%%

\usepackage[utf8]{inputenc} % allow utf-8 input
\usepackage[T1]{fontenc}    % use 8-bit T1 fonts
\usepackage{url}            % simple URL typesetting
\usepackage{microtype}      % microtypography
\usepackage{calc}           % perform arithmetic on the arguments
\usepackage{xspace}         % commands that appear not to eat spaces
\usepackage{enumitem}       % control layout of itemize, enumerate
\usepackage{numprint}       % prints numbers with a separator every three digits
\usepackage{epigraph}       % for citation

\usepackage[title,
            toc,
	    titletoc,
	    page,
	    header,
	  ]{appendix}       % extra control of appendices

\usepackage[nomaketitle
	   ]{psl-cover}     % package for PSL cover

%%%%%%%%%%%%%%%%%%%%%%%%%%%%%%%%%%%%%%%%%%%%%%%%%%%%%%%%%%%%%%%%%%%%%%%%%%%%%%%
% Minitoc
%%%%%%%%%%%%%%%%%%%%%%%%%%%%%%%%%%%%%%%%%%%%%%%%%%%%%%%%%%%%%%%%%%%%%%%%%%%%%%%

\usepackage{etoc}
\newlength\tocrulewidth
\setlength{\tocrulewidth}{0.8pt}
\etocsettocstyle{
\addsec*{Contents\\\vspace{-0.5cm}\rule{\textwidth}{\tocrulewidth}}}{
\noindent\rule{\linewidth}{\tocrulewidth}}

%%%%%%%%%%%%%%%%%%%%%%%%%%%%%%%%%%%%%%%%%%%%%%%%%%%%%%%%%%%%%%%%%%%%%%%%%%%%%%%
% Glossary
%%%%%%%%%%%%%%%%%%%%%%%%%%%%%%%%%%%%%%%%%%%%%%%%%%%%%%%%%%%%%%%%%%%%%%%%%%%%%%%

\newlength\marginglossary
\setlength{\marginglossary}{2.4cm}

%%%%%%%%%%%%%%%%%%%%%%%%%%%%%%%%%%%%%%%%%%%%%%%%%%%%%%%%%%%%%%%%%%%%%%%%%%%%%%%
% Mathematical packages
%%%%%%%%%%%%%%%%%%%%%%%%%%%%%%%%%%%%%%%%%%%%%%%%%%%%%%%%%%%%%%%%%%%%%%%%%%%%%%%

\usepackage{amsmath}        % Basic mathematical typography
\usepackage{mathtools}      % Mathematical tools to use with amsmath
\usepackage{amssymb}
\usepackage{amsthm}         % Basic mathematical environments for proofs etc.
\usepackage{mathdots}       % dots commands for matrices 
\usepackage{amsfonts}       % blackboard math symbols
\usepackage{nicefrac}       % compact symbols for 1/2, etc.
\usepackage{physics}        % typesetting equations for physics simpler
\usepackage{bm}             % Access bold symbols in maths mode

%%%%%%%%%%%%%%%%%%%%%%%%%%%%%%%%%%%%%%%%%%%%%%%%%%%%%%%%%%%%%%%%%%%%%%%%%%%%%%%
% Graphics packages
%%%%%%%%%%%%%%%%%%%%%%%%%%%%%%%%%%%%%%%%%%%%%%%%%%%%%%%%%%%%%%%%%%%%%%%%%%%%%%%

\usepackage{pgfplots}
\usepackage{graphicx}       % enhanced support for graphics
\usepackage{tikz}           % create graphic elements in latex
\usepackage{tikz-cd}        % Create commutative diagrams with TikZ
\usetikzlibrary{positioning}
\usetikzlibrary{arrows.meta}
\usepgfplotslibrary{groupplots}

%%%%%%%%%%%%%%%%%%%%%%%%%%%%%%%%%%%%%%%%%%%%%%%%%%%%%%%%%%%%%%%%%%%%%%%%%%%%%%%
% Table packages
%%%%%%%%%%%%%%%%%%%%%%%%%%%%%%%%%%%%%%%%%%%%%%%%%%%%%%%%%%%%%%%%%%%%%%%%%%%%%%%

\usepackage{array}          % extending the array and tabular environments
\usepackage{multirow}       % create tabular cells spanning multiple rows
\usepackage{tabularx}       % tabulars with adjustable-width columns
\usepackage{booktabs}       % professional-quality tables
\usepackage{makecell}

%%%%%%%%%%%%%%%%%%%%%%%%%%%%%%%%%%%%%%%%%%%%%%%%%%%%%%%%%%%%%%%%%%%%%%%%%%%%%%%
% Some of my favorite personal adjustments
%%%%%%%%%%%%%%%%%%%%%%%%%%%%%%%%%%%%%%%%%%%%%%%%%%%%%%%%%%%%%%%%%%%%%%%%%%%%%%%
%
% These are the adjustments that I consider necessary for typesetting
% a nice thesis. However, they are *not* included in the template, as
% I do not want to force you to use them.

% This ensures that I am able to typeset bold font in table while still aligning the numbers
% correctly.
\usepackage{etoolbox}

\usepackage[binary-units=true]{siunitx}
\DeclareSIUnit\px{px}

\sisetup{%
  detect-all           = true,
  detect-family        = true,
  detect-mode          = true,
  detect-shape         = true,
  detect-weight        = true,
  detect-inline-weight = math,
}

%%%%%%%%%%%%%%%%%%%%%%%%%%%%%%%%%%%%%%%%%%%%%%%%%%%%%%%%%%%%%%%%%%%%%%%%%%%%%%%
% Hyperlinks
%%%%%%%%%%%%%%%%%%%%%%%%%%%%%%%%%%%%%%%%%%%%%%%%%%%%%%%%%%%%%%%%%%%%%%%%%%%%%%%

\definecolor{mydarkblue}{rgb}{0,0.08,0.45}
\usepackage[%
  colorlinks = true,
  citecolor  = mydarkblue,
  linkcolor  = mydarkblue,
  urlcolor   = mydarkblue,
  ]{hyperref}
\usepackage{bookmark}       % Bookmarks in the resulting PDF

%%%%%%%%%%%%%%%%%%%%%%%%%%%%%%%%%%%%%%%%%%%%%%%%%%%%%%%%%%%%%%%%%%%%%%%%%%%%%%%
% Bibliography
%%%%%%%%%%%%%%%%%%%%%%%%%%%%%%%%%%%%%%%%%%%%%%%%%%%%%%%%%%%%%%%%%%%%%%%%%%%%%%%
%
% I like the bibliography to be extremely plain, showing only a numeric
% identifier and citing everything in simple brackets. The first names,
% if present, will be initialized. DOIs and URLs will be preserved.

\usepackage[%
  autocite     = plain,
  backend      = biber,
  doi          = false,
  url          = false,
  uniquename   = init,
  giveninits   = true,
  hyperref     = true,
  maxbibnames  = 99,
  maxcitenames = 2,
  sortcites    = false,
  style        = authoryear-comp,
  natbib       = true,
  backref      = true,
  abbreviate   = true,
  ]{biblatex}

\makeatletter
\let\abx@macro@citeOrig\abx@macro@cite
\renewbibmacro{cite}{%
\bibhyperref{%
\let\bibhyperref\relax\relax%
\abx@macro@citeOrig%
  }%
}
\let\abx@macro@textciteOrig\abx@macro@textcite
\renewbibmacro{textcite}{%
  \bibhyperref{%
    \let\bibhyperref\relax\relax%
    \abx@macro@textciteOrig%
  }%
}%
\makeatother

\DefineBibliographyStrings{english}{and={and}}
\DefineBibliographyStrings{english}{in={in}}
\DefineBibliographyStrings{english}{page={page}}
\DefineBibliographyStrings{english}{pages={pages}}
\DefineBibliographyStrings{english}{andothers={et al.}}
\DefineBibliographyStrings{english}{byeditor={editor}}
\DefineBibliographyStrings{english}{volume={volume}}
\DefineBibliographyStrings{english}{backrefpage={see page}}
\DefineBibliographyStrings{english}{backrefpages={see pages}}
\renewcommand{\cite}[1]{\citep{#1}}
\renewcommand{\finalnamedelim}{ \& }
\renewcommand{\textcitedelim}{
  \iflastcitekey{
    \unspace\textcolor{black}{\text{\ and}}\unspace
  }{
  \addcomma}\space}

%%%%%%%%%%%%%%%%%%%%%%%%%%%%%%%%%%%%%%%%%%%%%%%%%%%%%%%%%%%%%%%%%%%%%%%%
% Some adjustments to make the bibliography more clean
%%%%%%%%%%%%%%%%%%%%%%%%%%%%%%%%%%%%%%%%%%%%%%%%%%%%%%%%%%%%%%%%%%%%%%%%
%
% The subsequent commands do the following:
%  - Removing the month field from the bibliography
%  - Fixing the Oxford commma
%  - Suppress the "in" for journal articles
%  - Remove the parentheses of the year in an article
%  - Delimit volume and issue of an article by a colon ":" instead of
%    a dot ""
%  - Use commas to separate the location of publishers from their name
%  - Remove the abbreviation for technical reports
%  - Display the label of bibliographic entries without brackets in the
%    bibliography
%  - Ensure that DOIs are followed by a non-breakable space
%  - Use hair spaces between initials of authors
%  - Make the font size of citations smaller
%  - Fixing ordinal numbers (1st, 2nd, 3rd, and so) on by using
%    superscripts

% Remove the month field from the bibliography. It does not serve a good
% purpose, I guess. And often, it cannot be used because the journals
% have some crazy issue policies.
\AtEveryBibitem{\clearfield{month}}
\AtEveryCitekey{\clearfield{month}}

% Fixing the Oxford comma. Not sure whether this is the proper solution.
% More information is available under [1] and [2].
%
% [1] http://tex.stackexchange.com/questions/97712/biblatex-apa-style-is-missing-a-comma-in-the-references-why
% [2] http://tex.stackexchange.com/questions/44048/use-et-al-in-biblatex-custom-style
%
\AtBeginBibliography{%
  \renewcommand*{\finalnamedelim}{%
    \ifthenelse{\value{listcount} > 2}{%
      \addcomma
      \addspace
      \bibstring{and}%
    }{%
      \addspace
      \bibstring{and}%
    }
  }
}

% Suppress "in" for journal articles. This is unnecessary in my opinion
% because the journal title is typeset in italics anyway.
\renewbibmacro{in:}{%
  \ifentrytype{article}
  {%
  }%
  % else
  {%
    \printtext{\bibstring{in}\intitlepunct}%
  }%
}

% Remove the parentheses for the year in an article. This removes a lot
% of undesired parentheses in the bibliography, thereby improving the
% readability. Moreover, it makes the look of the bibliography more
% consistent.
\renewbibmacro*{issue+date}{%
  \setunit{\addcomma\space}
    \iffieldundef{issue}
      {\usebibmacro{date}}
      {\printfield{issue}%
       \setunit*{\addspace}%
       \usebibmacro{date}}%
  \newunit}

% Delimit the volume and the number of an article by a colon instead of
% by a dot, which I consider to be more readable.
\renewbibmacro*{volume+number+eid}{%
  \printfield{volume}%
  \setunit*{\addcolon}%
  \printfield{number}%
  \setunit{\addcomma\space}%
  \printfield{eid}%
}

% Do not use a colon for the publisher location. Instead, connect
% publisher, location, and date via commas.
\renewbibmacro*{publisher+location+date}{%
  \printlist{publisher}%
  \setunit*{\addcomma\space}%
  \printlist{location}%
  \setunit*{\addcomma\space}%
  \usebibmacro{date}%
  \newunit%
}

% Ditto for other entry types.
\renewbibmacro*{organization+location+date}{%
  \printlist{location}%
  \setunit*{\addcomma\space}%
  \printlist{organization}%
  \setunit*{\addcomma\space}%
  \usebibmacro{date}%
  \newunit%
}

% Do not abbreviate "technical report".
\DefineBibliographyStrings{english}{%
  techreport = {technical report},
}

% Display the label of a bibliographic entry in bare style, without any
% brackets. I like this more than the default.
%
% Note that this is *really* the proper and official way of doing this.
\DeclareFieldFormat{labelnumberwidth}{#1\adddot}

% Ensure that DOIs are followed by a non-breakable space.
\DeclareFieldFormat{doi}{%
  \mkbibacro{DOI}\addcolon\addnbspace
    \ifhyperref
      {\href{http://dx.doi.org/#1}{\nolinkurl{#1}}}
      %
      {\nolinkurl{#1}}
}

% Use proper hair spaces between initials as suggested by Bringhurst and
% others.
\renewcommand*\bibinitdelim {\addnbthinspace}
\renewcommand*\bibnamedelima{\addnbthinspace}
\renewcommand*\bibnamedelimb{\addnbthinspace}
\renewcommand*\bibnamedelimi{\addnbthinspace}

% Make the font size of citations smaller. Depending on your selected
% font, you might not need this.
\renewcommand*{\citesetup}{%
  \biburlsetup
  \small
}

\DeclareLanguageMapping{british}{bibliography-correct-ordinals}
\DeclareLanguageMapping{english}{bibliography-correct-ordinals}

\bibliography{bibliography}

\setlength\bibitemsep{10pt}

%%%%%%%%%%%%%%%%%%%%%%%%%%%%%%%%%%%%%%%%%%%%%%%%%%%%%%%%%%%%%%%%%%%%%%%%%%%%%%%
% Fonts
%%%%%%%%%%%%%%%%%%%%%%%%%%%%%%%%%%%%%%%%%%%%%%%%%%%%%%%%%%%%%%%%%%%%%%%%%%%%%%%
% \ifxetexorluatex
%   \setmainfont{Minion Pro}
% \else
%   \usepackage[lf]{ebgaramond}
%   \usepackage[oldstyle,scale=0.7]{sourcecodepro}
%   \singlespacing
% \fi

%%%%%%%%%%%%%%%%%%%%%%%%%%%%%%%%%%%%%%%%%%%%%%%%%%%%%%%%%%%%%%%%%%%%%%%%%%%%%%%
% Debugging
%%%%%%%%%%%%%%%%%%%%%%%%%%%%%%%%%%%%%%%%%%%%%%%%%%%%%%%%%%%%%%%%%%%%%%%%%%%%%%%
\usepackage{lineno}
% \linenumbers
\usepackage{todonotes}
\let\todoorg\todo
\renewcommand{\todo}[1]{\todoorg[inline]{#1}}

%%%%%%%%%%%%%%%%%%%%%%%%%%%%%%%%%%%%%%%%%%%%%%%%%%%%%%%%%%%%%%%%%%%%%%%%%%%%%%%
% algorithm
%%%%%%%%%%%%%%%%%%%%%%%%%%%%%%%%%%%%%%%%%%%%%%%%%%%%%%%%%%%%%%%%%%%%%%%%%%%%%%%
\usepackage{algorithm}
\usepackage{algpseudocode}

%%%%%%%%%%%%%%%%%%%%%%%%%%%%%%%%%%%%%%%%%%%%%%%%%%%%%%%%%%%%%%%%%%%%%%%%%%%%%%%
% Theorems
%%%%%%%%%%%%%%%%%%%%%%%%%%%%%%%%%%%%%%%%%%%%%%%%%%%%%%%%%%%%%%%%%%%%%%%%%%%%%%%

\newtheorem{definition}   {Definition}
\newtheorem{theorem}      {Theorem}
\newtheorem{corollary}    {Corollary}
\newtheorem{lemma}        {Lemma}
\newtheorem{proposition}  {Proposition}
\newtheorem{property}     {Property}
\newtheorem{remark}       {Remark}

\renewcommand\qedsymbol{$\blacksquare$}

\newcommand\proofrefth[1]   {\textbf{Proof of Theorem}~\ref{#1}}
\newcommand\proofrefcor[1]  {\textbf{Proof of Corollary}~\ref{#1}}
\newcommand\proofreflem[1]  {\textbf{Proof of Lemma}~\ref{#1}}
\newcommand\proofrefprop[1] {\textbf{Proof of Proposition}~\ref{#1}}

%%%%%%%%%%%%%%%%%%%%%%%%%%%%%%%%%%%%%%%%%%%%%%%%%%%%%%%%%%%%%%%%%%%%%%%%%%%%%%%
% Specific Operators
%%%%%%%%%%%%%%%%%%%%%%%%%%%%%%%%%%%%%%%%%%%%%%%%%%%%%%%%%%%%%%%%%%%%%%%%%%%%%%%

\DeclareMathOperator{\reshape}   {vec}
\DeclareMathOperator{\pad}       {pad}
\DeclareMathOperator{\lip}       {Lip}
\DeclareMathOperator{\lipbound}  {LipBound}
\DeclareMathOperator{\relu}      {ReLU}
\DeclareMathOperator*{\argmax}   {arg\,max}
\DeclareMathOperator*{\argmin}   {arg\,min}
\DeclareMathOperator*{\Vol}      {Vol}

\DeclareMathOperator{\sign}{sign}
\DeclareMathOperator*{\Risk}{Risk}
\DeclareMathOperator*{\advRisk}{Risk_{\alpha}}
\DeclareMathOperator*{\PCadvRisk}{PC-Risk_{\alpha}}
\DeclareMathOperator*{\randRisk}{Rand-Risk}
\DeclareMathOperator*{\randadvRisk}{Rand-Risk_{\alpha}}
\DeclareMathOperator*{\ranfPCadvRisk}{Rand-PC-Risk_{\alpha}}
\DeclareMathOperator*{\B}{B(\alpha)}
\DeclareMathOperator*{\probmap}{M}
\DeclareMathOperator*{\EoT}{EoT}


%%%%%%%%%%%%%%%%%%%%%%%%%%%%%%%%%%%%%%%%%%%%%%%%%%%%%%%%%%%%%%%%%%%%%%%%%%%%%%%
% Custom Commands
%%%%%%%%%%%%%%%%%%%%%%%%%%%%%%%%%%%%%%%%%%%%%%%%%%%%%%%%%%%%%%%%%%%%%%%%%%%%%%%

% \newcolumntype{C}[1]{>{\centering\arraybackslash}m{#1}}

\newcolumntype{L}[1]{>{\raggedright\arraybackslash}p{#1}}
\newcolumntype{C}[1]{>{\centering\arraybackslash}p{#1}}

% complex number i
\newcommand{\ci}{\ensuremath \mathbf{i}}

% definition of index a_h and B_h
\newcommand{\seqidx}{h}
\newcommand{\seqsetN}{N}
\newcommand{\seqsetM}{M}

% channel-in and channels-out
\newcommand{\cin}  {\mathrm{cin}}
\newcommand{\cout} {\mathrm{cout}}

% norms
\newcommand{\lzero} {\ensuremath \ell_0 \xspace}
\newcommand{\lone}  {\ensuremath \ell_1 \xspace}
\newcommand{\ltwo}  {\ensuremath \ell_2 \xspace}
\newcommand{\linf}  {\ensuremath \ell_\infty \xspace}
\newcommand{\lp}    {\ensuremath \ell_p \xspace}

% command for sets
\newcommand{\Nbb}{\ensuremath \mathbb{N}}
\newcommand{\Zbb}{\ensuremath \mathbb{Z}}
\newcommand{\Rbb}{\ensuremath \mathbb{R}}
\newcommand{\Cbb}{\ensuremath \mathbb{C}}
\newcommand{\Pbb}{\ensuremath \mathbb{P}}
\newcommand{\Ebb}{\ensuremath \mathbb{E}}
\newcommand{\Rnn}{\Rbb^{n \times n}}
\newcommand{\Cnn}{\Cbb^{n \times n}}
% \newcommand{\Rn}{\Rbb^{n}}
\newcommand{\Cn}{\Cbb^{n}}
\newcommand{\Nxx}[2]{{\ensuremath \Nbb^{#1 \times #2}}}
\newcommand{\Zxx}[2]{{\ensuremath \Zbb^{#1 \times #2}}}
\newcommand{\Cxx}[2]{{\ensuremath \Cbb^{#1 \times #2}}}
\newcommand{\Rxx}[2]{{\ensuremath \Rbb^{#1 \times #2}}}

% command for matrices
\newcommand{\Amat}{\ensuremath \mathbf{A}}
\newcommand{\Bmat}{\ensuremath \mathbf{B}}
\newcommand{\Cmat}{\ensuremath \mathbf{C}}
\newcommand{\Dmat}{\ensuremath \mathbf{D}}
\newcommand{\Emat}{\ensuremath \mathbf{E}}
\newcommand{\Fmat}{\ensuremath \mathbf{F}}
\newcommand{\Gmat}{\ensuremath \mathbf{G}}
\newcommand{\Hmat}{\ensuremath \mathbf{H}}
\newcommand{\Imat}{\ensuremath \mathbf{I}}
\newcommand{\Jmat}{\ensuremath \mathbf{J}}
\newcommand{\Kmat}{\ensuremath \mathbf{K}}
\newcommand{\Lmat}{\ensuremath \mathbf{L}}
\newcommand{\Mmat}{\ensuremath \mathbf{M}}
\newcommand{\Nmat}{\ensuremath \mathbf{N}}
\newcommand{\Omat}{\ensuremath \mathbf{O}}
\newcommand{\Pmat}{\ensuremath \mathbf{P}}
\newcommand{\Qmat}{\ensuremath \mathbf{Q}}
\newcommand{\Rmat}{\ensuremath \mathbf{R}}
\newcommand{\Smat}{\ensuremath \mathbf{S}}
\newcommand{\Tmat}{\ensuremath \mathbf{T}}
\newcommand{\Umat}{\ensuremath \mathbf{U}}
\newcommand{\Vmat}{\ensuremath \mathbf{V}}
\newcommand{\Wmat}{\ensuremath \mathbf{W}}
\newcommand{\Xmat}{\ensuremath \mathbf{X}}
\newcommand{\Ymat}{\ensuremath \mathbf{Y}}
\newcommand{\Zmat}{\ensuremath \mathbf{Z}}

% command for matrices
\newcommand{\avec}{\ensuremath \mathbf{a}}
\newcommand{\bvec}{\ensuremath \mathbf{b}}
\newcommand{\cvec}{\ensuremath \mathbf{c}}
\newcommand{\dvec}{\ensuremath \mathbf{d}}
\newcommand{\evec}{\ensuremath \mathbf{e}}
\newcommand{\fvec}{\ensuremath \mathbf{f}}
\newcommand{\gvec}{\ensuremath \mathbf{g}}
\newcommand{\hvec}{\ensuremath \mathbf{h}}
\newcommand{\ivec}{\ensuremath \mathbf{i}}
\newcommand{\jvec}{\ensuremath \mathbf{j}}
\newcommand{\kvec}{\ensuremath \mathbf{k}}
\newcommand{\lvec}{\ensuremath \mathbf{l}}
\newcommand{\mvec}{\ensuremath \mathbf{m}}
\newcommand{\nvec}{\ensuremath \mathbf{n}}
\newcommand{\ovec}{\ensuremath \mathbf{o}}
\newcommand{\pvec}{\ensuremath \mathbf{p}}
\newcommand{\qvec}{\ensuremath \mathbf{q}}
\newcommand{\rvec}{\ensuremath \mathbf{r}}
\newcommand{\svec}{\ensuremath \mathbf{s}}
\newcommand{\tvec}{\ensuremath \mathbf{t}}
\newcommand{\uvec}{\ensuremath \mathbf{u}}
\newcommand{\vvec}{\ensuremath \mathbf{v}}
\newcommand{\wvec}{\ensuremath \mathbf{w}}
\newcommand{\xvec}{\ensuremath \mathbf{x}}
\newcommand{\yvec}{\ensuremath \mathbf{y}}
\newcommand{\zvec}{\ensuremath \mathbf{z}}

% matrix brackets
\newcommand{\leftmatrix}  {\begin{pmatrix}}
\newcommand{\rightmatrix} {\end{pmatrix}}
\newcommand{\leftmat}     {\left(}
\newcommand{\rightmat}    {\right)}

% circulant & diagonal
\newcommand{\circulant}   {\ensuremath \mathrm{circ}}
\newcommand{\diagonal}    {\ensuremath \mathrm{diag}}

% variance and covariance
\newcommand{\Cov}{\ensuremath \mathrm{Cov}}
\newcommand{\Var}{\ensuremath \mathrm{Var}}


\newcommand{\F}{\ensuremath \mathcal{F}}

%%%%%%%%%%%%%%%%%%%%%%%%%%%%%%%%%%%%%%%%%%%%%%%%%%%%%%%%%%%%%%%%%%%%%%%%%%%%%%%
% Text shortcut
%%%%%%%%%%%%%%%%%%%%%%%%%%%%%%%%%%%%%%%%%%%%%%%%%%%%%%%%%%%%%%%%%%%%%%%%%%%%%%%

\newcommand{\yt}    {\textit{YouTube-8M}\xspace}
\newcommand{\eg}    {\textit{e.g.}\xspace}
\newcommand{\ie}    {\textit{i.e.}\xspace}
\newcommand{\aka}   {\textit{a.k.a}\xspace}
\newcommand{\vs}    {\textit{vs.}\xspace}
\newcommand{\wrt}   {w.r.t.\xspace}
\newcommand{\st}    { s.t.\xspace}
\newcommand{\ACDC}  {ACDC\xspace}
\newcommand{\AFDF}  {AFCF\xspace}

% \newcommand{\drn}{Deep ReLU network\xspace}
% \newcommand{\drns}{Deep ReLU networks\xspace}
% \newcommand{\DCRN}{DCRN\xspace}
% \newcommand{\DCRNs}{DCRNs\xspace}
% \newcommand{\divisiblerank}{divisible rank}
% \newcommand{\totalrank}{total rank}


