
%%%%%%%%%%%%%%%%%%%%%%%%%%%%%%%%%%%%%%%%%%%%%%%%%%%%%%%%%%%%%%%%%%%%%%%%%%%%%%%
% Basic packages
%%%%%%%%%%%%%%%%%%%%%%%%%%%%%%%%%%%%%%%%%%%%%%%%%%%%%%%%%%%%%%%%%%%%%%%%%%%%%%%

\usepackage[utf8]{inputenc}                                 % allow utf-8 input
\usepackage[T1]{fontenc}                                    % use 8-bit T1 fonts
\usepackage{url}                                            % simple URL typesetting
\usepackage{microtype}                                      % microtypography
\usepackage{calc}                                           % perform arithmetic on the arguments
\usepackage{xspace}                                         % commands that appear not to eat spaces
\usepackage{enumitem}                                       % control layout of itemize, enumerate
\usepackage{numprint}                                       % prints numbers with a separator every three digits
\usepackage{epigraph}                                       % for citation
\usepackage[super]{nth}                                     % Generate English ordinal numbers
\usepackage{bbding}                                         % A symbol font
\usepackage[title, toc, titletoc, page, header]{appendix}   % extra control of appendices
\usepackage[nomaketitle]{psl-cover}                         % package for PSL cover
\usepackage{etoc}                                           % Support for Mini Table of Contents
\usepackage{etoolbox}                                       % toolbox
\usepackage[binary-units=true]{siunitx}                     % A comprehensive (SI) units package

%%%%%%%%%%%%%%%%%%%%%%%%%%%%%%%%%%%%%%%%%%%%%%%%%%%%%%%%%%%%%%%%%%%%%%%%%%%%%%%
% Mathematical packages
%%%%%%%%%%%%%%%%%%%%%%%%%%%%%%%%%%%%%%%%%%%%%%%%%%%%%%%%%%%%%%%%%%%%%%%%%%%%%%%

\usepackage{mathtools}      % Mathematical tools to use with amsmath
\usepackage{amssymb}
\usepackage{amsthm}         % Basic mathematical environments for proofs etc.
\usepackage{mathdots}       % dots commands for matrices 
\usepackage{amsfonts}       % blackboard math symbols
\usepackage{nicefrac}       % compact symbols for 1/2, etc.
\usepackage{physics}        % typesetting equations for physics simpler
\usepackage{bm}             % Access bold symbols in maths mode
\usepackage{centernot}      % The package provides \centernot
\usepackage{thmtools}       % Extensions to theorem environments

%%%%%%%%%%%%%%%%%%%%%%%%%%%%%%%%%%%%%%%%%%%%%%%%%%%%%%%%%%%%%%%%%%%%%%%%%%%%%%%
% Graphics packages
%%%%%%%%%%%%%%%%%%%%%%%%%%%%%%%%%%%%%%%%%%%%%%%%%%%%%%%%%%%%%%%%%%%%%%%%%%%%%%%

\usepackage{pgfplots}
\usepackage{graphicx}       % enhanced support for graphics
\usepackage{tikz}           % create graphic elements in latex
\usepackage{tikz-cd}        % Create commutative diagrams with TikZ
\usetikzlibrary{positioning}
\usetikzlibrary{arrows.meta}
\usepgfplotslibrary{groupplots}

%%%%%%%%%%%%%%%%%%%%%%%%%%%%%%%%%%%%%%%%%%%%%%%%%%%%%%%%%%%%%%%%%%%%%%%%%%%%%%%
% Table packages
%%%%%%%%%%%%%%%%%%%%%%%%%%%%%%%%%%%%%%%%%%%%%%%%%%%%%%%%%%%%%%%%%%%%%%%%%%%%%%%

\usepackage{array}          % extending the array and tabular environments
\usepackage{multirow}       % create tabular cells spanning multiple rows
\usepackage{tabularx}       % tabulars with adjustable-width columns
\usepackage{booktabs}       % professional-quality tables
\usepackage{makecell}

% This ensures that I am able to typeset bold font in table while still aligning the numbers correctly.
\DeclareSIUnit\px{px}
\sisetup{%
  separate-uncertainty,
  detect-all           = true,
  detect-family        = true,
  detect-mode          = true,
  detect-shape         = true,
  detect-weight        = true,
  detect-inline-weight = math,
  multi-part-units     = single
}

\newcolumntype{L}[1]{>{\raggedright\arraybackslash}p{#1}}
\newcolumntype{C}[1]{>{\centering\arraybackslash}p{#1}}

%%%%%%%%%%%%%%%%%%%%%%%%%%%%%%%%%%%%%%%%%%%%%%%%%%%%%%%%%%%%%%%%%%%%%%%%%%%%%%%
% Hyperlinks
%%%%%%%%%%%%%%%%%%%%%%%%%%%%%%%%%%%%%%%%%%%%%%%%%%%%%%%%%%%%%%%%%%%%%%%%%%%%%%%

% \definecolor{mydarkblue}{rgb}{0,0.08,0.45}
% \definecolor{mydarkblue}{rgb}{0,0.14,0.45}
% \definecolor{mydarkblue}{rgb}{0.25, 0.41, 0.88}
% \definecolor{mydarkblue}{rgb}{0.0, 0.22, 0.66}
\definecolor{mydarkblue}{rgb}{0.02, 0.40, 0.62}
\usepackage[
  colorlinks=true,
  allcolors=mydarkblue,
  ]{hyperref}
\usepackage{cleveref}       % Intelligent cross-referencing
\usepackage{bookmark}       % Bookmarks in the resulting PDF


%%%%%%%%%%%%%%%%%%%%%%%%%%%%%%%%%%%%%%%%%%%%%%%%%%%%%%%%%%%%%%%%%%%%%%%%%%%%%%%
% Algoithms
%%%%%%%%%%%%%%%%%%%%%%%%%%%%%%%%%%%%%%%%%%%%%%%%%%%%%%%%%%%%%%%%%%%%%%%%%%%%%%%

% should be loaded after hyperref
\usepackage[chapter]{algorithm} % Support for typesetting algorithms
\usepackage{algpseudocode}      % Support for typesetting pseudocode

\makeatletter
\newcounter{algorithmicH}% New algorithmic-like hyperref counter
\let\oldalgorithmic\algorithmic
\renewcommand{\algorithmic}{%
  \stepcounter{algorithmicH}% Step counter
  \oldalgorithmic}% Do what was always done with algorithmic environment
\renewcommand{\theHALG@line}{ALG@line.\thealgorithmicH.\arabic{ALG@line}}
\makeatother


%%%%%%%%%%%%%%%%%%%%%%%%%%%%%%%%%%%%%%%%%%%%%%%%%%%%%%%%%%%%%%%%%%%%%%%%%%%%%%%
% Parameters
%%%%%%%%%%%%%%%%%%%%%%%%%%%%%%%%%%%%%%%%%%%%%%%%%%%%%%%%%%%%%%%%%%%%%%%%%%%%%%%

% Fonts
\DeclareFontFamily{T1}{calligra}{}
\DeclareFontShape{T1}{calligra}{m}{n}{<->s*[1.44]callig15}{}
\DeclareMathAlphabet\mathcalligra   {T1} {calligra} {m}  {n}
\DeclareMathAlphabet\mathzapf       {T1} {pzc}      {mb} {it}
\DeclareMathAlphabet\mathchorus     {T1} {qzc}      {m}  {n}
\DeclareMathAlphabet\mathrsfso      {U}  {rsfso}    {m}  {n}

\DeclareUnicodeCharacter{2212}{-}

% Minitoc
\newlength\tocrulewidth
\setlength{\tocrulewidth}{0.8pt}
\etocsettocstyle{
  \addsec*{Contents \\ \vspace{-0.5cm}
    \rule{\textwidth}{\tocrulewidth}
    \vspace{-1cm plus0mm minus0mm}
  }
}{
  \noindent\rule{\linewidth}{\tocrulewidth}
}

\newcommand{\localtoc}{
  \begingroup
    \localtableofcontents
    % \vspace{1cm}
  \endgroup
}


%%%%%%%%%%%%%%%%%%%%%%%%%%%%%%%%%%%%%%%%%%%%%%%%%%%%%%%%%%%%%%%%%%%%%%%%%%%%%%%
% Bibliography
%%%%%%%%%%%%%%%%%%%%%%%%%%%%%%%%%%%%%%%%%%%%%%%%%%%%%%%%%%%%%%%%%%%%%%%%%%%%%%%

\usepackage[%
  style        = authoryear-comp, % autocite=inline, sortcites=true, labeldateparts=true, uniquename=full, uniquelist=true.
  backend      = biber,
  autocite     = plain,
  uniquename   = false,
  uniquelist   = false,
  sorting      = ynt,
  sortcites    = true,
  doi          = false,
  url          = false,
  giveninits   = true,
  hyperref     = true,
  maxbibnames  = 99,
  maxcitenames = 2,
  natbib       = true,
  backref      = true,
  abbreviate   = true,
  dashed       = false,
  ]{biblatex}

\makeatletter
\let\abx@macro@citeOrig\abx@macro@cite
\renewbibmacro{cite}{%
  \bibhyperref{%
  \let\bibhyperref\relax\relax%
  \abx@macro@citeOrig%
    }%
}
% \let\abx@macro@textciteOrig\abx@macro@textcite
% \renewbibmacro{textcite}{%
%   \bibhyperref{%
%     \let\bibhyperref\relax\relax%
%     \abx@macro@textciteOrig%
%   }%
% }%

% code that fix the color of the last bracket
% https://tex.stackexchange.com/a/25972
\DeclareCiteCommand{\textcite}
 {\boolfalse{cbx:parens}}
  {\usebibmacro{citeindex}%
    \printtext[bibhyperref]{\printnames{labelname}%
      \printtext{ (\printfield{year}\printtext{)}}}}
 {\ifbool{cbx:parens}
  {\bibcloseparen\global\boolfalse{cbx:parens}}
  {}%
 \textcitedelim}
{\usebibmacro{textcite:postnote}}

% code that fix the color of the last bracket
% https://tex.stackexchange.com/a/425289

% \newcommand*{\linkblx@startlink}[1]{%
%   \blx@sfsave\hyper@natlinkstart{\the\c@refsection @#1}\blx@sfrest}
% \newcommand*{\linkblx@startlinkentry}{%
%   \linkblx@startlink{\abx@field@entrykey}}
% \newcommand*{\linkblx@endlink}{%
%   \blx@sfsave\hyper@natlinkend\blx@sfrest}

% \DeclareCiteCommand{\textcite}
%   {\boolfalse{cbx:parens}}
%   {\DeclareFieldFormat{bibhyperref}{####1}%
%    \linkblx@startlinkentry
%    \usebibmacro{citeindex}%
%    \iffirstcitekey
%      {\setcounter{textcitetotal}{1}}
%      {\stepcounter{textcitetotal}}%
%    \printtext[bibhyperref]{\usebibmacro{textcite}}%
%    \iflastcitekey
%      {}
%      {\ifbool{cbx:parens}
%         {\bibcloseparen\global\boolfalse{cbx:parens}}
%         {}}%
%    \ifnumequal
%      {\value{citecount}}{\value{citetotal}}
%      {\usebibmacro{textcite:postnote}}
%      {}%
%    \linkblx@endlink}
%   {}
%   {}

\makeatother



\DefineBibliographyStrings{english}{and={and}}
\DefineBibliographyStrings{english}{in={in}}
\DefineBibliographyStrings{english}{page={page}}
\DefineBibliographyStrings{english}{pages={pages}}
\DefineBibliographyStrings{english}{andothers={et al.}}
\DefineBibliographyStrings{english}{byeditor={editor}}
\DefineBibliographyStrings{english}{volume={volume}}
\DefineBibliographyStrings{english}{backrefpage={see page}}
\DefineBibliographyStrings{english}{backrefpages={see pages}}
\DefineBibliographyStrings{english}{phdthesis={PhD Thesis}}
\renewcommand{\cite}[1]{\citep{#1}}
\renewcommand{\finalnamedelim}{ \& }
\renewcommand{\textcitedelim}{
  \iflastcitekey{
    \unspace\textcolor{black}{\text{\ and}}\unspace
  }{
  \addcomma}\space}

%%%%%%%%%%%%%%%%%%%%%%%%%%%%%%%%%%%%%%%%%%%%%%%%%%%%%%%%%%%%%%%%%%%%%%%%
% Some adjustments to make the bibliography more clean
%%%%%%%%%%%%%%%%%%%%%%%%%%%%%%%%%%%%%%%%%%%%%%%%%%%%%%%%%%%%%%%%%%%%%%%%
%
% The subsequent commands do the following:
%  - Removing the month field from the bibliography
%  - Fixing the Oxford commma
%  - Suppress the "in" for journal articles
%  - Remove the parentheses of the year in an article
%  - Delimit volume and issue of an article by a colon ":" instead of
%    a dot ""
%  - Use commas to separate the location of publishers from their name
%  - Remove the abbreviation for technical reports
%  - Display the label of bibliographic entries without brackets in the
%    bibliography
%  - Ensure that DOIs are followed by a non-breakable space
%  - Use hair spaces between initials of authors
%  - Make the font size of citations smaller
%  - Fixing ordinal numbers (1st, 2nd, 3rd, and so) on by using
%    superscripts

% Remove the month field from the bibliography. It does not serve a good
% purpose, I guess. And often, it cannot be used because the journals
% have some crazy issue policies.
\AtEveryBibitem{\clearfield{month}}
\AtEveryCitekey{\clearfield{month}}

% Fixing the Oxford comma. Not sure whether this is the proper solution.
% More information is available under [1] and [2].
%
% [1] http://tex.stackexchange.com/questions/97712/biblatex-apa-style-is-missing-a-comma-in-the-references-why
% [2] http://tex.stackexchange.com/questions/44048/use-et-al-in-biblatex-custom-style
%
\AtBeginBibliography{%
  \renewcommand*{\finalnamedelim}{%
    \ifthenelse{\value{listcount} > 2}{%
      \addcomma
      \addspace
      \bibstring{and}%
    }{%
      \addspace
      \bibstring{and}%
    }
  }
}

% Suppress "in" for journal articles. This is unnecessary in my opinion
% because the journal title is typeset in italics anyway.
\renewbibmacro{in:}{%
  \ifentrytype{article}
  {%
  }%
  % else
  {%
    \printtext{\bibstring{in}\intitlepunct}%
  }%
}

% Remove the parentheses for the year in an article. This removes a lot
% of undesired parentheses in the bibliography, thereby improving the
% readability. Moreover, it makes the look of the bibliography more
% consistent.
\renewbibmacro*{issue+date}{%
  \setunit{\addcomma\space}
    \iffieldundef{issue}
      {\usebibmacro{date}}
      {\printfield{issue}%
       \setunit*{\addspace}%
       \usebibmacro{date}}%
  \newunit}

% Delimit the volume and the number of an article by a colon instead of
% by a dot, which I consider to be more readable.
\renewbibmacro*{volume+number+eid}{%
  \printfield{volume}%
  \setunit*{\addcolon}%
  \printfield{number}%
  \setunit{\addcomma\space}%
  \printfield{eid}%
}

% Do not use a colon for the publisher location. Instead, connect
% publisher, location, and date via commas.
\renewbibmacro*{publisher+location+date}{%
  \printlist{publisher}%
  \setunit*{\addcomma\space}%
  \printlist{location}%
  \setunit*{\addcomma\space}%
  \usebibmacro{date}%
  \newunit%
}

% Ditto for other entry types.
\renewbibmacro*{organization+location+date}{%
  \printlist{location}%
  \setunit*{\addcomma\space}%
  \printlist{organization}%
  \setunit*{\addcomma\space}%
  \usebibmacro{date}%
  \newunit%
}

% Do not abbreviate "technical report".
\DefineBibliographyStrings{english}{%
  techreport = {technical report},
}

% Display the label of a bibliographic entry in bare style, without any
% brackets. I like this more than the default.
%
% Note that this is *really* the proper and official way of doing this.
\DeclareFieldFormat{labelnumberwidth}{#1\adddot}

% Ensure that DOIs are followed by a non-breakable space.
\DeclareFieldFormat{doi}{%
  \mkbibacro{DOI}\addcolon\addnbspace
    \ifhyperref
      {\href{http://dx.doi.org/#1}{\nolinkurl{#1}}}
      %
      {\nolinkurl{#1}}
}

% Use proper hair spaces between initials as suggested by Bringhurst and
% others.
\renewcommand*\bibinitdelim {\addnbthinspace}
\renewcommand*\bibnamedelima{\addnbthinspace}
\renewcommand*\bibnamedelimb{\addnbthinspace}
\renewcommand*\bibnamedelimi{\addnbthinspace}

% Make the font size of citations smaller. Depending on your selected
% font, you might not need this.
\renewcommand*{\citesetup}{%
  \biburlsetup
  \small
}

\DeclareLanguageMapping{british}{bibliography-correct-ordinals}
\DeclareLanguageMapping{english}{bibliography-correct-ordinals}

\addbibresource{bibliography/springer.bib}
\addbibresource{bibliography/neurips.bib}
\addbibresource{bibliography/linear_algebra.bib}
\addbibresource{bibliography/aaai.bib}
\addbibresource{bibliography/icml.bib}
\addbibresource{bibliography/iclr.bib}
\addbibresource{bibliography/iccv.bib}
\addbibresource{bibliography/eccv.bib}
\addbibresource{bibliography/cvpr.bib}
\addbibresource{bibliography/arxiv.bib}
\addbibresource{bibliography/other.bib}

\setlength\bibitemsep{5pt}



%%%%%%%%%%%%%%%%%%%%%%%%%%%%%%%%%%%%%%%%%%%%%%%%%%%%%%%%%%%%%%%%%%%%%%%%%%%%%%%
% Theorems
%%%%%%%%%%%%%%%%%%%%%%%%%%%%%%%%%%%%%%%%%%%%%%%%%%%%%%%%%%%%%%%%%%%%%%%%%%%%%%%


% maybe use nobreak to prevent theorem to be on multiple pages
% mdframed={
%   nobreak=true,
%   topline=false,
%   bottomline=false,
%   rightline=false,
%   leftline=false,
%   linewidth=0pt,
%   backgroundcolor=white,
%   innerleftmargin=0pt,
%   innerrightmargin=0pt,
% }


\declaretheoremstyle[
  spaceabove=6pt, spacebelow=6pt,
  bodyfont=\itshape,
  postheadspace=1em,
  qed=,
]{style}

\declaretheoremstyle[
  spaceabove=6pt, spacebelow=6pt,
  bodyfont=\itshape,
  postheadspace=1em,
  qed=,
  shaded={
    rulecolor=mydarkblue,
    rulewidth=2pt,
    bgcolor=white,
    padding=0.02\textwidth,
    textwidth=0.96\textwidth},
]{maintheoremstyle}


\declaretheorem[
  numberwithin=chapter,
  name=Theorem,
  style=style,
  refname={theorem,theorems},
  Refname={Theorem,Theorems}]{theorem}
\declaretheorem[
  numberwithin=chapter,
  name=Corollary,
  style=style,
  refname={corollary,corollaries},
  Refname={Corollary,Corollaries}]{corollary}

\declaretheorem[
  numberwithin=chapter,
  name=Definition,
  style=style,
  refname={definition,definitions},
  Refname={Definition,Definitions}]{definition}
\declaretheorem[
  numberwithin=chapter,
  name=Lemma,
  style=style,
  refname={lemma,lemmas},
  Refname={Lemma,Lemmas}]{lemma}
\declaretheorem[
  numberwithin=chapter,
  name=Proposition,
  style=style,
  refname={proposition,propositions},
  Refname={Proposition,Propositions}]{proposition}
\declaretheorem[
  numberwithin=chapter,
  name=Problem,
  style=style,
  refname={problem,problems},
  Refname={Problem,Problems}]{problem}
\declaretheorem[
  numberwithin=chapter,
  numbered=no,
  name=Remark,
  style=style,
  refname={remark,remarks},
  Refname={Remark,Remarks}]{remark}

\declaretheorem[
  numberwithin=chapter,
  name=Theorem,
  style=maintheoremstyle,
  sibling=theorem,
  refname={theorem,theorems},
  Refname={Theorem,Theorems}]{maintheorem}
\declaretheorem[
  numberwithin=chapter,
  name=Corollary,
  style=maintheoremstyle,
  sibling=corollary,
  refname={corollary,corollaries},
  Refname={Corollary,Corollaries}]{maincorollary}

\declaretheorem[
  numberwithin=chapter,
  numbered=no,
  name=Property,
  style=style,
  refname={property,properties},
  Refname={Property,Properties}]{property}
\declaretheorem[
  numberwithin=chapter,
  numbered=no,
  name=Properties,
  style=style,
  refname={property,properties},
  Refname={Property,Properties}]{properties}





\let\proof\relax
\let\endproof\relax

\declaretheoremstyle[
  notebraces={}{},
  % spaceabove=30pt,
  % spacebelow=30pt,
  headfont=\bfseries\itshape,
  notefont=\bfseries\itshape,
  postheadspace=1em,
  postfoothook=\vspace{2em},
  qed=$\blacksquare$,
  mdframed={
    topline=false,
    bottomline=false,
    rightline=false,
    linecolor=lightgray,
    linewidth=3pt,
    backgroundcolor=white,
    innerleftmargin=0.02\textwidth,
    innerrightmargin=0pt,
    % innertopmargin=30pt,
    % innerbottommargin=0pt
  }
]{proofstyle}

\declaretheorem[
  numbered=no,
  name=Proof of,
  style=proofstyle,
  refname={proof,proofs},
  Refname={Proof,Proofs}
]{proof}



%%%%%%%%%%%%%%%%%%%%%%%%%%%%%%%%%%%%%%%%%%%%%%%%%%%%%%%%%%%%%%%%%%%%%%%%%%%%%%%
% Text commands
%%%%%%%%%%%%%%%%%%%%%%%%%%%%%%%%%%%%%%%%%%%%%%%%%%%%%%%%%%%%%%%%%%%%%%%%%%%%%%%

\newcommand{\yt}    {\textit{\mbox{YouTube-8M}}\xspace}
\newcommand{\eg}    {\textit{e.g.}\xspace}
\newcommand{\ie}    {\textit{i.e.}\xspace}
\newcommand{\cf}    {\textit{cf.}\xspace}
\newcommand{\aka}   {\textit{a.k.a.}\xspace}
\newcommand{\pdf}   {p.d.f.\xspace}
\newcommand{\st}    { s.t.\xspace}
% \newcommand{\vs}    {\textit{vs.}\xspace}
% \newcommand{\wrt}   {w.r.t.\xspace}

\newcommand{\DC}    {$\mathbf{DC}$\xspace}
\newcommand{\ACDC}  {$\mathbf{ACDC}$\xspace}
% \newcommand{\AFDF}  {$\mathbf{AFDF}^{-1}$\xspace}

% Star for separating paragraph
\newcommand{\drawstar}{
  \begin{center}
    \color{mydarkblue}{\FourStar}
  \end{center}
}

% \newcommand{\drn}{Deep ReLU network\xspace}
% \newcommand{\drns}{Deep ReLU networks\xspace}
% \newcommand{\DCRN}{DCRN\xspace}
% \newcommand{\DCRNs}{DCRNs\xspace}
% \newcommand{\divisiblerank}{divisible rank}
% \newcommand{\totalrank}{total rank}


%%%%%%%%%%%%%%%%%%%%%%%%%%%%%%%%%%%%%%%%%%%%%%%%%%%%%%%%%%%%%%%%%%%%%%%%%%%%%%%
% Specific Operators
%%%%%%%%%%%%%%%%%%%%%%%%%%%%%%%%%%%%%%%%%%%%%%%%%%%%%%%%%%%%%%%%%%%%%%%%%%%%%%%

\DeclareMathOperator*{\argmax}       {arg\,max}
\DeclareMathOperator*{\argmin}       {arg\,min}
\DeclareMathOperator*{\Ebb}          {\mathbb{E}}
\DeclareMathOperator*{\Pbb}          {\mathbb{P}}
\DeclareMathOperator*{\Risk}         {Risk}
\DeclareMathOperator*{\advRisk}      {Risk_{\alpha}}
\DeclareMathOperator*{\PCadvRisk}    {PC-Risk_{\alpha}}
\DeclareMathOperator*{\randRisk}     {Rand-Risk}
\DeclareMathOperator*{\randadvRisk}  {Rand-Risk_{\alpha}}
\DeclareMathOperator*{\ranfPCadvRisk}{Rand-PC-Risk_{\alpha}}
\DeclareMathOperator*{\B}            {B(\alpha)}
\DeclareMathOperator*{\probmap}      {M}

% \DeclareMathOperator{\lipop}{Lip}
% \DeclareMathOperator{\lipbound}{LipBound}
% \newcommand{\lip}[1]{\lipop{(#1)}}
% \newcommand{\lipbound}[1]{\lipboundop{(#1)}}

% \DeclareMathOperator{\lipbound}{LipBound}
% \DeclareMathOperator{\reshape}       {vec}
% \DeclareMathOperator{\pad}           {pad}
% \DeclareMathOperator{\relu}          {ReLU}

% \relax\vec
% \renewcommand{\vec}    {\ensuremath{\mathrm{vec}}\xspace}
\newcommand{\relu}     {\ensuremath{\mathrm{ReLU}}\xspace}
\newcommand{\pad}      {\ensuremath{\mathrm{pad}}\xspace}
\newcommand{\lipp}[2]   {\ensuremath{\mathrm{Lip}_{#1} (#2)}}
\newcommand{\lip}[1]   {\ensuremath{\mathrm{Lip}(#1)}}
\newcommand{\lipbound} {\ensuremath{\mathrm{LipBound}}}
\newcommand{\diag}     {\ensuremath{\mathrm{diag}}}
\newcommand{\sign}     {\ensuremath{\mathrm{sign}}}
\renewcommand{\mod}[1] {\ \mathrm{mod}\ #1}



% \DeclareMathOperator*{\diag}         {diag}
% \DeclareMathOperator{\sign}          {sign}
\DeclareMathOperator*{\EoT}          {EoT}
\DeclareMathOperator*{\Vol}          {Vol}

% text in math environment 
\newcommand{\circulant}   {\ensuremath \mathrm{circ}}
\newcommand{\diagonal}    {\ensuremath \mathrm{diag}}
\newcommand{\krylov}      {\ensuremath \mathrm{krylov}}
\newcommand{\Cov}         {\ensuremath \mathrm{Cov}}
\newcommand{\Var}         {\ensuremath \mathrm{Var}}
\newcommand{\cin}         {\ensuremath c_{in}}
\newcommand{\cout}        {\ensuremath c_{out}}

%%%%%%%%%%%%%%%%%%%%%%%%%%%%%%%%%%%%%%%%%%%%%%%%%%%%%%%%%%%%%%%%%%%%%%%%%%%%%%%
% Custom Commands
%%%%%%%%%%%%%%%%%%%%%%%%%%%%%%%%%%%%%%%%%%%%%%%%%%%%%%%%%%%%%%%%%%%%%%%%%%%%%%%


% TODO delete these commands
\newcommand{\seqidx}{h}
\newcommand{\seqsetN}{N}
\newcommand{\seqsetM}{M}
\newcommand{\Cnn}{\Cbb^{n \times n}}
\newcommand{\Cn}{\Cbb^{n}}
\newcommand{\F}{\ensuremath \mathcal{F}}


\newcommand{\scalefigure}{0.80}
\newcommand{\removespace}{\vspace{-\baselineskip}}

% complex number i
\newcommand{\ci}{\ensuremath \mathbf{i}}


% norms
\newcommand{\lzero} {\ensuremath \ell_0 \xspace}
\newcommand{\lone}  {\ensuremath \ell_1 \xspace}
\newcommand{\ltwo}  {\ensuremath \ell_2 \xspace}
\newcommand{\linf}  {\ensuremath \ell_\infty \xspace}
\newcommand{\lp}    {\ensuremath \ell_p \xspace}
\newcommand{\fro}   {\ensuremath \mathrm{F}}

% command for sets
\newcommand{\Nbb}{\ensuremath \mathbb{N}}
\newcommand{\Zbb}{\ensuremath \mathbb{Z}}
\newcommand{\Rbb}{\ensuremath \mathbb{R}}
\newcommand{\Cbb}{\ensuremath \mathbb{C}}
\newcommand{\Xset}{\ensuremath \mathcal{X}}
\newcommand{\Yset}{\ensuremath \mathcal{Y}}
\newcommand{\bigO}{\ensuremath \mathcal{O}}

% Vector of ones
\newcommand{\onevec}[1]{\ensuremath \mathbf{1}_{#1}}
\newcommand{\zerovec}[1]{\ensuremath \mathbf{0}_{#1}}

% neural networks
\newcommand{\nn}{N}
\newcommand{\dcnn}{DCNN\xspace}
\newcommand{\dcnns}{DCNNs\xspace}
\newcommand{\act}{\rho}
\newcommand{\layer}{\phi}
\newcommand{\weights}{\Omega}
\newcommand{\depth}{p}
\newcommand{\dimw}{w}
\newcommand{\dimb}{b}
\newcommand{\adv}{\pmb{\tau}}


% operator low displacement rank
\newcommand{\triangleopup}{\pmb{\bigtriangleup}}
\newcommand{\triangleopdown}{\rotatebox[x=0pt,y=4pt]{180}{$\triangleopup$}}

% matrix brackets
\newcommand{\leftmatrix}       {\begin{pmatrix}}
\newcommand{\rightmatrix}      {\end{pmatrix}}
\newcommand{\leftmatrixsmall}  {\begin{psmallmatrix}}
\newcommand{\rightmatrixsmall} {\end{psmallmatrix}}
\newcommand{\leftmat}          {\left(}
\newcommand{\rightmat}         {\right)}
\newcommand{\smallddots}       {\scalebox{.40}{$\ddots$}}


% command for matrices
\newcommand{\Amat}{\ensuremath \mathbf{A}}
\newcommand{\Bmat}{\ensuremath \mathbf{B}}
\newcommand{\Cmat}{\ensuremath \mathbf{C}}
\newcommand{\Dmat}{\ensuremath \mathbf{D}}
\newcommand{\Emat}{\ensuremath \mathbf{E}}
\newcommand{\Fmat}{\ensuremath \mathbf{F}}
\newcommand{\Gmat}{\ensuremath \mathbf{G}}
\newcommand{\Hmat}{\ensuremath \mathbf{H}}
\newcommand{\Imat}{\ensuremath \mathbf{I}}
\newcommand{\Jmat}{\ensuremath \mathbf{J}}
\newcommand{\Kmat}{\ensuremath \mathbf{K}}
\newcommand{\Lmat}{\ensuremath \mathbf{L}}
\newcommand{\Mmat}{\ensuremath \mathbf{M}}
\newcommand{\Nmat}{\ensuremath \mathbf{N}}
\newcommand{\Omat}{\ensuremath \mathbf{O}}
\newcommand{\Pmat}{\ensuremath \mathbf{P}}
\newcommand{\Qmat}{\ensuremath \mathbf{Q}}
\newcommand{\Rmat}{\ensuremath \mathbf{R}}
\newcommand{\Smat}{\ensuremath \mathbf{S}}
\newcommand{\Tmat}{\ensuremath \mathbf{T}}
\newcommand{\Umat}{\ensuremath \mathbf{U}}
\newcommand{\Vmat}{\ensuremath \mathbf{V}}
\newcommand{\Wmat}{\ensuremath \mathbf{W}}
\newcommand{\Xmat}{\ensuremath \mathbf{X}}
\newcommand{\Ymat}{\ensuremath \mathbf{Y}}
\newcommand{\Zmat}{\ensuremath \mathbf{Z}}

% command for matrices
\newcommand{\avec}{\ensuremath \mathbf{a}}
\newcommand{\bvec}{\ensuremath \mathbf{b}}
\newcommand{\cvec}{\ensuremath \mathbf{c}}
\newcommand{\dvec}{\ensuremath \mathbf{d}}
\newcommand{\evec}{\ensuremath \mathbf{e}}
\newcommand{\fvec}{\ensuremath \mathbf{f}}
\newcommand{\gvec}{\ensuremath \mathbf{g}}
\newcommand{\hvec}{\ensuremath \mathbf{h}}
\newcommand{\ivec}{\ensuremath \mathbf{i}}
\newcommand{\jvec}{\ensuremath \mathbf{j}}
\newcommand{\kvec}{\ensuremath \mathbf{k}}
\newcommand{\lvec}{\ensuremath \mathbf{l}}
\newcommand{\mvec}{\ensuremath \mathbf{m}}
\newcommand{\nvec}{\ensuremath \mathbf{n}}
\newcommand{\ovec}{\ensuremath \mathbf{o}}
\newcommand{\pvec}{\ensuremath \mathbf{p}}
\newcommand{\qvec}{\ensuremath \mathbf{q}}
\newcommand{\rvec}{\ensuremath \mathbf{r}}
\newcommand{\svec}{\ensuremath \mathbf{s}}
\newcommand{\tvec}{\ensuremath \mathbf{t}}
\newcommand{\uvec}{\ensuremath \mathbf{u}}
\newcommand{\vvec}{\ensuremath \mathbf{v}}
\newcommand{\wvec}{\ensuremath \mathbf{w}}
\newcommand{\xvec}{\ensuremath \mathbf{x}}
\newcommand{\yvec}{\ensuremath \mathbf{y}}
\newcommand{\zvec}{\ensuremath \mathbf{z}}


%%%%%%%%%%%%%%%%%%%%%%%%%%%%%%%%%%%%%%%%%%%%%%%%%%%%%%%%%%%%%%%%%%%%%%%%%%%%%%%
% Debugging
%%%%%%%%%%%%%%%%%%%%%%%%%%%%%%%%%%%%%%%%%%%%%%%%%%%%%%%%%%%%%%%%%%%%%%%%%%%%%%%
\usepackage{lineno}
\usepackage{todonotes}
\usepackage{printlen}

\let\todoorg\todo
\renewcommand{\todo}[1]{\todoorg[inline]{#1}}
\newcommand{\Omit}[1]{}

\newcommand{\drawline}{
  \begin{center}
    \rule{0.5\linewidth}{1pt}
  \end{center}
  \newpage
}

% todotext command 
\newcommand{\todotext}[1]{{\color{red} \noindent \textbf{#1}}}
\newcommand{\todotextok}[1]{{\color{green} \noindent \textbf{#1}}}

% command to count page of chapter
\newcommand{\countpages}[2]{\the\numexpr\getpagerefnumber{#2}-\getpagerefnumber{#1}\relax}

