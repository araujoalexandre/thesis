%%%%%%%%%%%%%%%%%%%%%%%%%%%%%%%%%%%%%%%%%%%%%%%%%%%%%%%%%%%%%%%%%%%%%%%%%%%%%%%
\chapter{Technical Proofs}
\label{chapter_appendix:technical_proofs}
%%%%%%%%%%%%%%%%%%%%%%%%%%%%%%%%%%%%%%%%%%%%%%%%%%%%%%%%%%%%%%%%%%%%%%%%%%%%%%%
\localtableofcontents

%%%%%%%%%%%%%%%%%%%%%%%%%%%%%%%%%%%%%%%%%%%%%%%%%%%%%%%%%%%%%%%%%%%%%%%%%%%%%%%
\section{Proofs of Chapter~\ref{chapter:diagonal_circulant_neural_network}}
%%%%%%%%%%%%%%%%%%%%%%%%%%%%%%%%%%%%%%%%%%%%%%%%%%%%%%%%%%%%%%%%%%%%%%%%%%%%%%%


\begin{proof}[\proofrefth{theorem:ch3-rank-decomposition}]
Let $\Umat \mathbf{\Sigma} \Vmat^{T}$ be the SVD decomposition of $\Mmat$ where $\Umat,\Vmat$ and $\mathbf{\Sigma}$ are $n \times n$ matrices.
Because $\Mmat$ is of rank $k$, the last $n-k$ columns of $\Umat$ and $\Vmat$ are null.
In the following, we will first decompose $\Umat$ into a product of matrices $\Wmat\Rmat\Omat$, where $\Rmat$ and $\Omat$ are respectively circulant and diagonal matrices, and $\Wmat$ is a matrix which will be further decomposed into a product of diagonal and circulant matrices.
Then, we will apply the same decomposition technique to $\Vmat$.
Ultimately, we will get a product of $4k+2$ matrices alternatively diagonal and circulant.  

Let $\Rmat = \circulant(r_{1}\ldots r_{n})$. Let $\Omat$ be a $n \times n$ diagonal matrix where $\Omat_{i,i} = 1$ if $i \le k$ and $0$ otherwise. The $k$ first columns of the product $\Rmat\Omat$ will be equal to that of $\Rmat$, and the $n-k$ last colomns of $\Rmat\Omat$ will be zeros. For example, if $k=2$, we have: 
\begin{equation}
  \Rmat\Omat = \leftmatrix
  r_{1} & r_{n} & 0 & \cdots & 0\\
  r_{2} & r_{1}\\
  r_{3} & r_{2} & \vdots &  & \vdots\\
  \vdots & \vdots\\
  r_{n} & r_{n-1} & 0 & \cdots & 0
  \rightmatrix
\end{equation}

Let us define $k$ diagonal matrices $\Dmat_{i} = \diagonal(d_{i1} \ldots d_{in})$ for $i \in [k]$.
For now, the values of $d_{ij}$ are unknown, but we will show how to compute them.
Let $\Wmat = \sum_{i=1}^{k} \Dmat_{i} \Smat^{i-1}$ where $\Smat$ is the \emph{cyclic shift} matrix 
%$S \in \mathbb{R}^{n\times n}$
define as follows:
\begin{equation}
  \Smat = \leftmatrix 0 &  &  &  & 1 \\
  1 & 0 \\
   & 1 & \ddots \\
   &  & \ddots & 0 \\
   &  &  & 1 & 0
  \rightmatrix
\end{equation}


Note that the $n-k$ last columns of the product $\Wmat\Rmat\Omat$ will be zeros.
For example, with $k=2$, we have: 
\begin{equation}
  \Wmat = \leftmatrix
  d_{1,1} &  &  &  & d_{2,1} \\
  d_{2,2} & d_{1,2} \\
   & d_{2,3} & \ddots \\
   &  & \ddots \\
   &  &  & d_{2,n} & d_{1,n}
  \rightmatrix
\end{equation}

\begin{equation}
  \Wmat\Rmat\Omat = \leftmatrix
  r_{1}d_{11}+r_{n}d_{21} & r_{n}d_{11}+r_{n-1}d_{21} & 0 & \cdots & 0 \\
  r_{2}d_{12}+r_{1}d_{22} & r_{1}d_{12}+r_{n}d_{22}\\
   &  & \vdots &  & \vdots \\
  \vdots & \vdots\\
  r_{n}d_{1n}+r_{n-1}d_{2n} & r_{n-1}d_{1n}+r_{n-2}d_{2n} & 0 & \cdots & 0
  \rightmatrix
\end{equation}
We want to find the values of $d_{ij}$ such that $\Wmat \Rmat \Omat = \Umat$. We can formulate this as linear equation system. In case $k=2$, we get:
\begin{equation}
  \leftmatrix
  r_{n} & r_{1}\\
  r_{n-1} & r_{n}\\
   &  & r_{1} & r_{2}\\
   &  & r_{n} & r_{1}\\
   &  &  &  & r_{2} & r_{3}\\
   &  &  &  & r_{1} & r_{2}\\
   &  &  &  &  &  & \ddots\\
   &  &  &  &  &  &  & \ddots
  \rightmatrix \times \leftmatrix
  d_{2,1}\\
  d_{1,1}\\
  d_{2,2}\\
  d_{1,2}\\
  d_{2,3}\\
  d_{1,3}\\
  \vdots\\
  \vdots
  \rightmatrix = \leftmatrix
  \Umat_{1,1}\\
  \Umat_{1,2}\\
  \Umat_{2,1}\\
  \Umat_{2,2}\\
  \\
  \\
  \vdots\\
  \\
  \rightmatrix
\end{equation}

The $i^{th}$ bloc of the bloc-diagonal matrix is a Toeplitz matrix induced by a subsequence of length $k$ of $(r_1,\ldots r_n,r_1 \ldots r_n)$.
Set $r_{j}=1$ for all $j\in\{k,2k,3k,\ldots n\}$ and set $r_{j}=0$ for all other values of $j$.
Then it is easy to see that each bloc is a permutation of the identity matrix.
Thus, all blocs are invertible.
This entails that the block diagonal matrix above is also invertible.
So by solving this set of linear equations, we find $d_{1,1}\ldots d_{k,n}$ such that $\Wmat\Rmat\Omat=\Umat$.
We can apply the same idea to factorize $\Vmat=\Wmat'.\Rmat.\Omat$ for some matrix $\Wmat'$.
Finally, we get 
\begin{equation}
  \Amat = \Umat \mathbf{\Sigma} \Vmat^\top = \Wmat\Rmat\Omat \mathbf{\Sigma} \Omat^\top \Rmat^\top \Wmat^{'\top}
\end{equation}

Thanks to Theorem~\ref{theorem:ch3-huhtanen}, $\Wmat$ and $\Wmat'$ can both be factorized in a product of $2k-1$ circulant and diagonal matrices.
Note that $\Omat \mathbf{\Sigma} \Omat^\top$ is diagonal, because all three are diagonal.
Overall, $\Amat$ can be represented with a product of $4k+2$ matrices, alternatively diagonal and circulant.
\end{proof}




%%%%%%%%%%%%%%%%%%%%%%%%%%%%%%%%%%%%%%%%%%%%%%%%%%%%%%%%%%%%%%%%%%%%%%%%%%%%%%%
\section{Proofs of Chapter~\ref{chapter:lipschitz_bound}}
%%%%%%%%%%%%%%%%%%%%%%%%%%%%%%%%%%%%%%%%%%%%%%%%%%%%%%%%%%%%%%%%%%%%%%%%%%%%%%%

\begin{proof}[\proofreflem{theorem:widom_idenity}]
Let $(i, j)$ be matrix indexes such $(\ \cdot\ )_{i, j}$ correspond to the value at the $i^\textrm{th}$ row and $j^\textrm{th}$ column, let us define the following notation:
\begin{align*}
    i_1 &= \left\lfloor i/n \right\rfloor \quad \quad &&j_1 = \left\lfloor j/n \right\rfloor \\
    i_2 &= i \mod n \quad \quad &&j_2 = j \mod n
\end{align*}

Let us define $\hat{f}$ as the 2 dimensional Fourier transform of the function $f$. We refer to $\hat{f}_{h_1, h_2}$ as the Fourier coefficient indexed by $(h_1, h_2)$ where $h_1$ correspond to the index of the block of the doubly-block Toeplitz and $h_2$ correspond to the index of the value inside the block. More precisely, we have 
\begin{align}
    \leftmat \Dmat(f) \rightmat_{i, j} &= \hat{f}_{(\left\lfloor j/n \right\rfloor - \left\lfloor i/n \right\rfloor), ((j \mod n) - (i \mod n)))} \label{equation:expression_fourier} \\
    \leftmat \Hmat^{\alpha_0}(f) \rightmat_{i, j} &= \hat{f}_{(\left\lfloor j/n \right\rfloor + \left\lfloor i/n \right\rfloor + 1), ((j \mod n) - (i \mod n)))} \\
    \leftmat \Hmat^{\alpha_1}(f) \rightmat_{i, j} &= \hat{f}_{(\left\lfloor j/n \right\rfloor - \left\lfloor i/n \right\rfloor), ((j \mod n) + (i \mod n) + 1))} \\
    \leftmat \Hmat^{\alpha_2}(f) \rightmat_{i, j} &= \hat{f}_{(\left\lfloor j/n \right\rfloor - \left\lfloor i/n \right\rfloor), ((j \mod n) - (i \mod n)))} \\
    \leftmat \Hmat^{\alpha_3}(f) \rightmat_{i, j} &= \hat{f}_{(\left\lfloor j/n \right\rfloor + \left\lfloor i/n \right\rfloor + n), ((j \mod n) + (i \mod n) + 1))}
\end{align}

We simplify the notation of the expressions above as follow:
\begin{align}
    \leftmat \Dmat(f) \rightmat_{i, j} &= \hat{f}_{(j_1 - i_1), (j_2 - i_2 )} \\
    \leftmat \Hmat^{\alpha_0}(f) \rightmat_{i, j} &= \hat{f}_{(j_1 + i_1 + 1), (j_2 - i_2 )} \\
    \leftmat \Hmat^{\alpha_1}(f) \rightmat_{i, j} &= \hat{f}_{(j_1 - i_1), (j_2 + i_2 + 1)} \\
    \leftmat \Hmat^{\alpha_2}(f) \rightmat_{i, j} &= \hat{f}_{(j_1 - i_1), (j_2 - i_2 )} \\
    \leftmat \Hmat^{\alpha_3}(f) \rightmat_{i, j} &= \hat{f}_{(j_1 + i_1 + n), (j_2 + i_2 + 1)}
\end{align}

The convolution theorem states that the Fourier transform of a product of two functions is the convolution of their Fourier coefficients. Therefore, one can observe that the entry $(i, j)$ of the matrix $\Dmat(f g)$ can be express as follows:

\begin{equation*}
    \leftmat \Dmat(f g) \rightmat_{i, j} = \sum_{k_1 = -2n + 1}^{2n-1} \sum_{k_2 = -2n + 1}^{2n-1} \hat{f}_{(k_1-i_1),(k_2-i_2)} \hat{g}_{(j_1-k_1),(j_2-k_2)}. 
\end{equation*}


By splitting the double sums and simplifying, we obtain:
\begin{align} \label{equation:split_double_sum}
\left( \Dmat(f g) \right)_{i, j} &= 
\sum_{k_1, k_2 \in P} \left(
\hat{f}_{(k_1-i_1),(k_2-i_2)} \hat{g}_{(j_1-k_1),(j_2-k_2)} +
\hat{f}_{(-k_1-i_1-1),(k_2-i_2)} \hat{g}_{(j_1+k_1+1),(j_2-k_2)} \right. \notag \\ &\quad+ \left.
\hat{f}_{(k_1-i_1),(-k_2-i_2-1)} \hat{g}_{(j_1-k_1),(j_2+k_2+1)} +
\hat{f}_{(-k_1-i_1-1),(-k_2-i_2-1)} \hat{g}_{(j_1+k_1+1),(j_2+k_2+1)} \right. \notag \\ &\quad+ \left.
\hat{f}_{(k_1-i_1+n),(-k_2-i_2-1)} \hat{g}_{(j_1-k_1-n),(j_2+k_2+1)} +
\hat{f}_{(k_1-i_1+n),(k_2-i_2)} \hat{g}_{(j_1-k_1-n),(j_2-k_2)} \right. \notag \\ &\quad+ \left.
\hat{f}_{(k_1-i_1),(k_2-i_2+n)} \hat{g}_{(j_1-k_1),(j_2-k_2-n)} +
\hat{f}_{(k_1-i_1+n),(k_2-i_2+n)} \hat{g}_{(j_1-k_1-n),(j_2-k_2-n)} \right. \notag \\ &\quad+ \left.
\hat{f}_{(-k_1-i_1-1),(k_2-i_2+n)} \hat{g}_{(j_1+k_1+1),(j_2-k_2-n)}  \right)
\end{align}
where $P = \{ (k_1, k_2)\ |\ k_1, k_2 \in \mathbb{N} \cup 0, 0 \leq k_1 \leq n-1,  0 \leq k_2 \leq n-1 \}$.


Furthermore, we can observe the following:
\begin{equation*}
    \leftmat \Dmat(f) \Dmat(g) \rightmat_{i, j} = \sum_{k = 0}^{n^2} \leftmat\Dmat(f)\rightmat_{i, k} \leftmat\Dmat(g)\rightmat_{k, j}  = \sum_{k_1, k_2 \in P} \hat{f}_{(k_1-i_1),(k_2-i_2)} \hat{g}_{(j_1-k_1),(j_2-k_2)}
\end{equation*}

{\allowdisplaybreaks
\begin{flalign*}
    % # H1_a_.T @ H1_b
    \leftmat \Hmat^{\alpha_1 \top}(f^*) \Hmat^{\alpha_1}(g) \rightmat_{i, j} &=  \sum_{k_1, k_2 \in P} \hat{f}^*_{(k_1+i_1+1),(i_2-k_2)} \hat{g}_{(j_1+k_1+1),(j_2-k_2)} \\
    &=  \sum_{k_1, k_2 \in P} \hat{f}_{(-k_1-i_1-1),(k_2-i_2)} \hat{g}_{(j_1+k_1+1),(j_2-k_2)} \\
    % # H2_a_.T @ H2_b
    \leftmat \Hmat^{\alpha_2 \top}(f^*) \Hmat^{\alpha_2}(g) \rightmat_{i, j} &=  \sum_{k_1, k_2 \in P} \hat{f}^*_{(i_1-k_1),(k_2+i_2+1)} \hat{g}_{(j_1-k_1),(j_2+k_2+1)} \\
    &=  \sum_{k_1, k_2 \in P} \hat{f}_{(k_1-i_1),(-k_2-i_2-1)} \hat{g}_{(j_1-k_1),(j_2+k_2+1)} \\
    % # H3_a_.T @ H3_b
    \leftmat \Hmat^{\alpha_3 \top}(f^*) \Hmat^{\alpha_3}(g) \rightmat_{i, j} &=  \sum_{k_1, k_2 \in P} \hat{f}^*_{(k_1+i_1+1),(k_2+i_2+1)} \hat{g}_{(j_1+k_1+1),(k_2+j_2+1)} \\
    &= \sum_{k_1, k_2 \in P} \hat{f}_{(-k_1-i_1-1),(-k_2-i_2-1)} \hat{g}_{(j_1+k_1+1),(k_2+j_2+1)} \\
    % # H4_a_.T @ H4_b
    \leftmat \Hmat^{\alpha_4 \top}(f^*) \Hmat^{\alpha_4}(g) \rightmat_{i, j} &= \sum_{k_1, k_2 \in P} \hat{f}^*_{(i_1-k_1-n),(k_2+i_2+1)} \hat{g}_{(j_1-k_1-n),(j_2+k_2+1)} \\
    &=  \sum_{k_1, k_2 \in P} \hat{f}_{(k_1-i_1+n),(-k_2-i_2-1)} \hat{g}_{(j_1-k_1-n),(j_2+k_2+1)} \\
\end{flalign*}
}
Let us define the matrix $\Qmat$ of size $n^2 \times n^2$ as the anti-identity matrix. We have the following:

{\allowdisplaybreaks
\begin{flalign*}
    % # Y @ H1_a.T @ H1_b_ @ Y
    \leftmat \Hmat^{\alpha_1 \top}(f) \Hmat^{\alpha_1}(g^*) \rightmat_{i, j} &= \sum_{k_1, k_2 \in P} \hat{f}_{(k_1+i_1+1),(i_2-k_2)} \hat{g}^*_{(j_1+k_1+1),(j_2-k_2)} \\
    &= \sum_{k_1, k_2 \in P} \hat{f}_{(k_1+i_1+1),(i_2-k_2)} \hat{g}_{(-j_1-k_1-1),(k_2-j_2)} \\
    \Leftrightarrow \leftmat \Qmat \Hmat^{\alpha_1 \top}(f) \Hmat^{\alpha_1}(g^*) \Qmat \rightmat_{i, j} &= \sum_{k_1, k_2 \in P} \hat{f}_{(k_1-i_1+n),(k_2-i_2)} \hat{g}_{(j_1-k_1-n),(j_2-k_2)} \\
    % # Y @ H2_a.T @ H2_b_ @ Y
    \leftmat \Hmat^{\alpha_2 \top}(f) \Hmat^{\alpha_2}(g^*) \rightmat_{i, j} &=  \sum_{k_1, k_2 \in P} \hat{f}_{(i_1-k_1),(k_2+i_2+1)} \hat{g}^*_{(j_1-k_1),(j_2+k_2+1)} \\
    &=  \sum_{k_1, k_2 \in P} \hat{f}_{(i_1-k_1),(k_2+i_2+1)} \hat{g}_{(k_1-j_1),(-j_2-k_2-1)} \\
    \Leftrightarrow \leftmat \Qmat \Hmat^{\alpha_2 \top}(f) \Hmat^{\alpha_2}(g^*) \Qmat \rightmat_{i, j} &=  \sum_{k_1, k_2 \in P} \hat{f}_{(k_1-i_1),(k_2-i_2+n)} \hat{g}_{(j_1-k_1),(j_2-k_2-n)} \\
    % # Y @ H3_a.T @ H3_b_ @ Y
    \leftmat \Hmat^{\alpha_3 \top}(f) \Hmat^{\alpha_3}(g^*) \rightmat_{i, j} &=  \sum_{k_1, k_2 \in P}  \hat{f}_{(k_1+i_1+1),(k_2+i_2+1)} \hat{g}^*_{(j_1+k_1+1),(k_2+j_2+1)} \\
    &=  \sum_{k_1, k_2 \in P} \hat{f}_{(k_1+i_1+1),(k_2+i_2+1)} \hat{g}_{(-j_1-k_1-1),(-k_2-j_2-1)} \\
    \Leftrightarrow \leftmat \Qmat \Hmat^{\alpha_3 \top}(f) \Hmat^{\alpha_3}(g^*) \Qmat \rightmat_{i, j} &=  \sum_{k_1, k_2 \in P} \hat{f}_{(k_1-i_1+n),(k_2-i_2+n)} \hat{g}_{(j_1-k_1-n),(-k_2+j_2-n)} \\
    % # Y @ H4_a.T @ H4_b_ @ Y
    \leftmat \Hmat^{\alpha_4 \top}(f) \Hmat^{\alpha_4}(g^*) \rightmat_{i, j} &=  \sum_{k_1, k_2 \in P}  \hat{f}_{(-k_1+i_1-n),(k_2+i_2+1)} \hat{g}^*_{(j_1-k_1-n),(j_2+k_2+1)} \\
    &= \sum_{k_1, k_2 \in P} \hat{f}_{(-k_1+i_1-n),(k_2+i_2+1)} \hat{g}_{(-j_1+k_1+n),(-j_2-k_2-1)} \\
    \Leftrightarrow \leftmat \Qmat \Hmat^{\alpha_4 \top}(f) \Hmat^{\alpha_4}(g^*) \Qmat \rightmat_{i, j} &= \sum_{k_1, k_2 \in P} \hat{f}_{(-k_1-i_1-1),(k_2-i_2+n)} \hat{g}_{(j_1+k_1+1),(j_2-k_2-n)}
\end{flalign*}
}

Now, we can observe from Equation~\ref{equation:split_double_sum} that:
\begin{equation}
    \Dmat(fg) = \Dmat(f)\Dmat(g) + \sum_{p=0}^3 \Hmat^{\alpha_p \top}(f^*) \Hmat^{\alpha_p}(g) + \Qmat \left( \sum_{p=0}^3 \Hmat^{\alpha_p \top}(f) \Hmat^{\alpha_p}(g^*) \right) \Qmat.
\end{equation}
which concludes the proof. 
\end{proof}
