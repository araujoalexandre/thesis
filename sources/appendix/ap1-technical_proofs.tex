%%%%%%%%%%%%%%%%%%%%%%%%%%%%%%%%%%%%%%%%%%%%%%%%%%%%%%%%%%%%%%%%%%%%%%%%%%%%%%%
\chapter{Generalization of Widom Identity}
\label{appendix:ap1-proof_of_the_generalization_of_widom_identity}
%%%%%%%%%%%%%%%%%%%%%%%%%%%%%%%%%%%%%%%%%%%%%%%%%%%%%%%%%%%%%%%%%%%%%%%%%%%%%%%
% \localtoc

This appendix aims at proving a generalization of Widom Identity for doubly-block Toeplitz operators.
The Widom identity, which states the relation between Toeplitz and Hankel operators, was introduced by Harold Widom in a \citeyear{widom1976asymptotic} seminal paper \cite{widom1976asymptotic}.
Let us define the semi-infinite Toeplitz and Hankel operators:
\begin{align}
  \Tmat_\infty(f) &\triangleq \leftmat\frac{1}{2\pi} \int_{0}^{2\pi} e^{-\ci(i-j)\omega}f(\omega) \diff \omega\rightmat_{i,j \in \{0, \dots, \infty\}} \\
  \Hmat_\infty(f) &\triangleq \leftmat\frac{1}{2\pi} \int_{0}^{2\pi} e^{-\ci(i+j+1)\omega}f(\omega) \diff \omega\rightmat_{i,j \in \{0, \dots, \infty\}}
\end{align}
Then, for $f$ and $g$ integrable functions, the Widom identity can be written as follows:
\begin{equation}
  \Tmat_\infty(fg) - \Tmat_\infty(f) \Tmat_\infty(g) = \Hmat_\infty(f) \Hmat_\infty(g^*)
\end{equation}
Note that Widom extend this identity from finite Toeplitz matrices:
\begin{equation} \label{equation:ap1-widom_identity}
  \Tmat_n(fg) - \Tmat_n(f) \Tmat_n(g) = \Hmat_n(f) \Hmat_n(g^*) - \Jmat_n \Hmat_n(f^*) \Hmat_n(g^*) \Jmat_n
\end{equation}
where $\Jmat_n$ is the anti-identity matrix, \ie, the reflexion matrix.

We would like to expend the identity presented in~\Cref{equation:ap1-widom_identity} to finite doubly-block Toeplitz operator.
We will need to generalize the doubly-block Toeplitz operator presented in~\Cref{subsection:ch5-bound_on_the_singular_value_of_doubly-block_toeplitz_matrices}.
Let $\Gmat^{\alpha_p} (f) = \leftmat \Gmat^{\alpha_p}_{i,j}(f) \rightmat_{i,j \in \Iset^+_n}$ where $\Gmat^{\alpha_p}_{i,j}$ is defined as:
\begin{equation}
  \Gmat^{\alpha_p}_{i,j}(f) =\leftmat \frac{1}{4\pi^{2}} \int_{0}^{2\pi} \int_{0}^{2\pi} e^{-\ci \alpha_p(i, j, k, l, \omega_1, \omega_2)}  f(\omega_{1},\omega_{2}) \diff \omega_{1} \diff \omega_{2})
  \rightmat_{k,l \in \Iset^+_n} \enspace.
\end{equation}
Note that as with the operator $\Dmat(f)$ we only consider generating functions as trigonometric polynomials with real coefficients therefore the matrices generated by $\Gmat(f)$ are real. 
And as with the operator $\Dmat(f)$, the matrices generated by the operator $\Gmat^{\alpha_p}$ are of size $n^2 \times n^2$. 

\noindent
We will use the following $\alpha$ functions:
\begin{itemize}
    \item[] $\alpha_0(i, j, k, l, \omega_1, \omega_2) = (-j-i-1)\omega_1 + (k-l)\omega_2$
    \item[] $\alpha_1(i, j, k, l, \omega_1, \omega_2) = (i-j)\omega_1 + (-l-k-1)\omega_2$
    \item[] $\alpha_2(i, j, k, l, \omega_1, \omega_2) = (-j-i-1)\omega_1 + (-l-k-1)\omega_2$
    \item[] $\alpha_3(i, j, k, l, \omega_1, \omega_2) = (-j-i+n)\omega_1 + (-l-k-1)\omega_2$
\end{itemize}

\noindent
We now present the generalization of the Widom identity for Doubly-Block Toeplitz matrices below:
\begin{lemma}[Generalization of Widom Identity] \label{lemma:ap1-widom_idenity}
  Let $f:\Rbb^2 \rightarrow \Cbb$ and $g:\Rbb^2 \rightarrow \Cbb$ be two continuous and $2\pi$-periodic functions. 
  We can decompose the Doubly-Block Toeplitz matrix $\Dmat(fg)$ as follows:
  \begin{equation}
    \Dmat(fg) = \Dmat(f)\Dmat(g) + \sum_{p=0}^3 \Gmat^{\alpha_p \top}(f^*) \Gmat^{\alpha_p}(g) + \Jmat_{n^2} \left( \sum_{p=0}^3 \Gmat^{\alpha_p \top}(f) \Gmat^{\alpha_p }(g^*) \right) \Jmat_{n^2}.
  \end{equation}
  where $\Jmat$ is the reflection of the identity matrix of size $n^2 \times n^2$.
\end{lemma}



\begin{proof}[\Cref{lemma:ap1-widom_idenity}]
Let $(i, j)$ be matrix indexes such $(\ \cdot\ )_{i, j}$ correspond to the value at the $i^\textrm{th}$ row and $j^\textrm{th}$ column, let us define the following notation:
\begin{align*}
    i_1 &= \left\lfloor i/n \right\rfloor \quad \quad &&j_1 = \left\lfloor j/n \right\rfloor \\
    i_2 &= i \mod n \quad \quad &&j_2 = j \mod n
\end{align*}

\noindent
Let us define $\hat{f}$ as the 2 dimensional Fourier transform of the function $f$. We refer to $\hat{f}_{h_1, h_2}$ as the Fourier coefficient indexed by $(h_1, h_2)$ where $h_1$ correspond to the index of the block of the doubly-block Toeplitz and $h_2$ correspond to the index of the value inside the block. More precisely, we have 
\begin{align}
    \leftmat \Dmat(f) \rightmat_{i, j} &= \hat{f}_{(\left\lfloor j/n \right\rfloor - \left\lfloor i/n \right\rfloor), ((j \mod n) - (i \mod n)))} \label{equation:expression_fourier} \\
    \leftmat \Gmat^{\alpha_0}(f) \rightmat_{i, j} &= \hat{f}_{(\left\lfloor j/n \right\rfloor + \left\lfloor i/n \right\rfloor + 1), ((j \mod n) - (i \mod n)))} \\
    \leftmat \Gmat^{\alpha_1}(f) \rightmat_{i, j} &= \hat{f}_{(\left\lfloor j/n \right\rfloor - \left\lfloor i/n \right\rfloor), ((j \mod n) + (i \mod n) + 1))} \\
    \leftmat \Gmat^{\alpha_2}(f) \rightmat_{i, j} &= \hat{f}_{(\left\lfloor j/n \right\rfloor - \left\lfloor i/n \right\rfloor), ((j \mod n) - (i \mod n)))} \\
    \leftmat \Gmat^{\alpha_3}(f) \rightmat_{i, j} &= \hat{f}_{(\left\lfloor j/n \right\rfloor + \left\lfloor i/n \right\rfloor + n), ((j \mod n) + (i \mod n) + 1))}
\end{align}

\noindent
We simplify the notation of the expressions above as follow:
\begin{align}
    \leftmat \Dmat(f) \rightmat_{i, j} &= \hat{f}_{(j_1 - i_1), (j_2 - i_2 )} \\
    \leftmat \Gmat^{\alpha_0}(f) \rightmat_{i, j} &= \hat{f}_{(j_1 + i_1 + 1), (j_2 - i_2 )} \\
    \leftmat \Gmat^{\alpha_1}(f) \rightmat_{i, j} &= \hat{f}_{(j_1 - i_1), (j_2 + i_2 + 1)} \\
    \leftmat \Gmat^{\alpha_2}(f) \rightmat_{i, j} &= \hat{f}_{(j_1 - i_1), (j_2 - i_2 )} \\
    \leftmat \Gmat^{\alpha_3}(f) \rightmat_{i, j} &= \hat{f}_{(j_1 + i_1 + n), (j_2 + i_2 + 1)}
\end{align}

\noindent
The convolution theorem states that the Fourier transform of a product of two functions is the convolution of their Fourier coefficients. Therefore, one can observe that the entry $(i, j)$ of the matrix $\Dmat(f g)$ can be express as follows:
\begin{equation*}
    \leftmat \Dmat(f g) \rightmat_{i, j} = \sum_{k_1 = -2n + 1}^{2n-1} \sum_{k_2 = -2n + 1}^{2n-1} \hat{f}_{(k_1-i_1),(k_2-i_2)} \hat{g}_{(j_1-k_1),(j_2-k_2)}. 
\end{equation*}

\noindent
By splitting the double sums and simplifying, we obtain:
\begin{align} \label{equation:split_double_sum}
  \left( \Dmat(f g) \right)_{i, j} &= 
  \sum_{k_1, k_2 \in P} \left(
    \hat{f}_{(k_1-i_1),(k_2-i_2)} \hat{g}_{(j_1-k_1),(j_2-k_2)} +
    \hat{f}_{(-k_1-i_1-1),(k_2-i_2)} \hat{g}_{(j_1+k_1+1),(j_2-k_2)} \right. \notag \\ &\quad+ \left.
    \hat{f}_{(k_1-i_1),(-k_2-i_2-1)} \hat{g}_{(j_1-k_1),(j_2+k_2+1)} +
    \hat{f}_{(-k_1-i_1-1),(-k_2-i_2-1)} \hat{g}_{(j_1+k_1+1),(j_2+k_2+1)} \right. \notag \\ &\quad+ \left.
    \hat{f}_{(k_1-i_1+n),(-k_2-i_2-1)} \hat{g}_{(j_1-k_1-n),(j_2+k_2+1)} +
    \hat{f}_{(k_1-i_1+n),(k_2-i_2)} \hat{g}_{(j_1-k_1-n),(j_2-k_2)} \right. \notag \\ &\quad+ \left.
    \hat{f}_{(k_1-i_1),(k_2-i_2+n)} \hat{g}_{(j_1-k_1),(j_2-k_2-n)} +
    \hat{f}_{(k_1-i_1+n),(k_2-i_2+n)} \hat{g}_{(j_1-k_1-n),(j_2-k_2-n)} \right. \notag \\ &\quad+ \left.
    \hat{f}_{(-k_1-i_1-1),(k_2-i_2+n)} \hat{g}_{(j_1+k_1+1),(j_2-k_2-n)}  \right)
\end{align}
where $P = \{ (k_1, k_2)\ |\ k_1, k_2 \in \Zbb, 0 \leq k_1 \leq n-1,  0 \leq k_2 \leq n-1 \}$.


\noindent
Furthermore, we can observe the following:
\begin{equation*}
  \leftmat \Dmat(f) \Dmat(g) \rightmat_{i, j} = \sum_{k = 0}^{n^2} \leftmat\Dmat(f)\rightmat_{i, k} \leftmat\Dmat(g)\rightmat_{k, j}  = \sum_{k_1, k_2 \in P} \hat{f}_{(k_1-i_1),(k_2-i_2)} \hat{g}_{(j_1-k_1),(j_2-k_2)}
\end{equation*}

% H1_a_.T @ H1_b
\begin{flalign*}
  \leftmat \Gmat^{\alpha_1 \top}(f^*) \Gmat^{\alpha_1}(g) \rightmat_{i, j} &=  \sum_{k_1, k_2 \in P} \hat{f}^*_{(k_1+i_1+1),(i_2-k_2)} \hat{g}_{(j_1+k_1+1),(j_2-k_2)} \\
  &=  \sum_{k_1, k_2 \in P} \hat{f}_{(-k_1-i_1-1),(k_2-i_2)} \hat{g}_{(j_1+k_1+1),(j_2-k_2)}
\end{flalign*}

% H2_a_.T @ H2_b
\begin{flalign*}
  \leftmat \Gmat^{\alpha_2 \top}(f^*) \Gmat^{\alpha_2}(g) \rightmat_{i, j} &=  \sum_{k_1, k_2 \in P} \hat{f}^*_{(i_1-k_1),(k_2+i_2+1)} \hat{g}_{(j_1-k_1),(j_2+k_2+1)} \\
  &=  \sum_{k_1, k_2 \in P} \hat{f}_{(k_1-i_1),(-k_2-i_2-1)} \hat{g}_{(j_1-k_1),(j_2+k_2+1)}
\end{flalign*}

% H3_a_.T @ H3_b
\begin{flalign*}
  \leftmat \Gmat^{\alpha_3 \top}(f^*) \Gmat^{\alpha_3}(g) \rightmat_{i, j} &=  \sum_{k_1, k_2 \in P} \hat{f}^*_{(k_1+i_1+1),(k_2+i_2+1)} \hat{g}_{(j_1+k_1+1),(k_2+j_2+1)} \\
  &= \sum_{k_1, k_2 \in P} \hat{f}_{(-k_1-i_1-1),(-k_2-i_2-1)} \hat{g}_{(j_1+k_1+1),(k_2+j_2+1)}
\end{flalign*}

% H4_a_.T @ H4_b
\begin{flalign*}
  \leftmat \Gmat^{\alpha_4 \top}(f^*) \Gmat^{\alpha_4}(g) \rightmat_{i, j} &= \sum_{k_1, k_2 \in P} \hat{f}^*_{(i_1-k_1-n),(k_2+i_2+1)} \hat{g}_{(j_1-k_1-n),(j_2+k_2+1)} \\
  &=  \sum_{k_1, k_2 \in P} \hat{f}_{(k_1-i_1+n),(-k_2-i_2-1)} \hat{g}_{(j_1-k_1-n),(j_2+k_2+1)}
\end{flalign*}
\noindent
Let us define the matrix $\Jmat_{n^2}$ of size $n^2 \times n^2$ as the anti-identity matrix. We have the following:

% Y @ H1_a.T @ H1_b_ @ Y
\begin{flalign*}
  \leftmat \Gmat^{\alpha_1 \top}(f) \Gmat^{\alpha_1}(g^*) \rightmat_{i, j} &= \sum_{k_1, k_2 \in P} \hat{f}_{(k_1+i_1+1),(i_2-k_2)} \hat{g}^*_{(j_1+k_1+1),(j_2-k_2)} \\
  &= \sum_{k_1, k_2 \in P} \hat{f}_{(k_1+i_1+1),(i_2-k_2)} \hat{g}_{(-j_1-k_1-1),(k_2-j_2)} \\
  \Leftrightarrow \leftmat \Jmat_{n^2} \Gmat^{\alpha_1 \top}(f) \Gmat^{\alpha_1}(g^*) \Jmat_{n^2} \rightmat_{i, j} &= \sum_{k_1, k_2 \in P} \hat{f}_{(k_1-i_1+n),(k_2-i_2)} \hat{g}_{(j_1-k_1-n),(j_2-k_2)}
\end{flalign*}

% Y @ H2_a.T @ H2_b_ @ Y
\begin{flalign*}
  \leftmat \Gmat^{\alpha_2 \top}(f) \Gmat^{\alpha_2}(g^*) \rightmat_{i, j} &=  \sum_{k_1, k_2 \in P} \hat{f}_{(i_1-k_1),(k_2+i_2+1)} \hat{g}^*_{(j_1-k_1),(j_2+k_2+1)} \\
  &=  \sum_{k_1, k_2 \in P} \hat{f}_{(i_1-k_1),(k_2+i_2+1)} \hat{g}_{(k_1-j_1),(-j_2-k_2-1)} \\
  \Leftrightarrow \leftmat \Jmat^{n^2} \Gmat^{\alpha_2 \top}(f) \Gmat^{\alpha_2}(g^*) \Jmat_{n^2} \rightmat_{i, j} &=  \sum_{k_1, k_2 \in P} \hat{f}_{(k_1-i_1),(k_2-i_2+n)} \hat{g}_{(j_1-k_1),(j_2-k_2-n)}
\end{flalign*}

% Y @ H3_a.T @ H3_b_ @ Y
\begin{flalign*}
  \leftmat \Gmat^{\alpha_3 \top}(f) \Gmat^{\alpha_3}(g^*) \rightmat_{i, j} &=  \sum_{k_1, k_2 \in P}  \hat{f}_{(k_1+i_1+1),(k_2+i_2+1)} \hat{g}^*_{(j_1+k_1+1),(k_2+j_2+1)} \\
  &=  \sum_{k_1, k_2 \in P} \hat{f}_{(k_1+i_1+1),(k_2+i_2+1)} \hat{g}_{(-j_1-k_1-1),(-k_2-j_2-1)} \\
  \Leftrightarrow \leftmat \Jmat_{n^2} \Gmat^{\alpha_3 \top}(f) \Gmat^{\alpha_3}(g^*) \Jmat_{n^2} \rightmat_{i, j} &=  \sum_{k_1, k_2 \in P} \hat{f}_{(k_1-i_1+n),(k_2-i_2+n)} \hat{g}_{(j_1-k_1-n),(-k_2+j_2-n)}
\end{flalign*}

% Y @ H4_a.T @ H4_b_ @ Y
\begin{flalign*}
  \leftmat \Gmat^{\alpha_4 \top}(f) \Gmat^{\alpha_4}(g^*) \rightmat_{i, j} &=  \sum_{k_1, k_2 \in P}  \hat{f}_{(-k_1+i_1-n),(k_2+i_2+1)} \hat{g}^*_{(j_1-k_1-n),(j_2+k_2+1)} \\
  &= \sum_{k_1, k_2 \in P} \hat{f}_{(-k_1+i_1-n),(k_2+i_2+1)} \hat{g}_{(-j_1+k_1+n),(-j_2-k_2-1)} \\
  \Leftrightarrow \leftmat \Jmat_{n^2} \Gmat^{\alpha_4 \top}(f) \Gmat^{\alpha_4}(g^*) \Jmat_{n^2} \rightmat_{i, j} &= \sum_{k_1, k_2 \in P} \hat{f}_{(-k_1-i_1-1),(k_2-i_2+n)} \hat{g}_{(j_1+k_1+1),(j_2-k_2-n)}
\end{flalign*}

\noindent
Now, we can observe from Equation~\ref{equation:split_double_sum} that:
\begin{equation}
  \Dmat(fg) = \Dmat(f)\Dmat(g) + \sum_{p=0}^3 \Gmat^{\alpha_p \top}(f^*) \Gmat^{\alpha_p}(g) + \Jmat_{n^2} \left( \sum_{p=0}^3 \Gmat^{\alpha_p \top}(f) \Gmat^{\alpha_p}(g^*) \right) \Jmat_{n^2}.
\end{equation}
which concludes the proof. 
\end{proof}
