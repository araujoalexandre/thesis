

%%%%%%%%%%%%%%%%%%%%%%%%%%%%%%%%%%%%%%%%%%%%%%%%%%%%%%%%%%%%%%%%%%%%%%%%%%%%%%%
\subsection{Circulant Matrices}
\label{subsection:ch2-circulant_matrices}
%%%%%%%%%%%%%%%%%%%%%%%%%%%%%%%%%%%%%%%%%%%%%%%%%%%%%%%%%%%%%%%%%%%%%%%%%%%%%%%

Circulant matrices are a special case of Toeplitz matrices.
In addition to having constant diagonals, each row of a circulant matrix is a cyclic right shift of the previous one.
% Circulant matrices have been used to efficiently solve linear systems~\cite{golub1996matrix} and years later were used to perform dimensionality reduction~\cite{hinrichs2011johnson,vybiral2011variant}, binary embedding~\cite{yu2014circulant} and kernel approximation~\cite{yu2015compact} in the context of pattern recognition and machine learning.
An $n \times n$ circulant matrix $\Cmat$ is fully determined by a sequence of scalars $\{c_h\}_{h \in \Pset_0}$ as follows:

\begin{equation}
  \Cmat =
  \leftmatrix
    c_0 & c_{n-1} & c_{n-2} & \cdots & \cdots & c_{1} \\
    c_{1} & c_0 & c_{n-1} & \ddots & & \vdots \\
    c_{2} & c_{1} & \ddots & \ddots & \ddots & \vdots \\
    \vdots & \ddots & \ddots & \ddots & c_{n-1} & c_{n-2} \\
    \vdots & & \ddots & c_{1} & c_{0} & c_{n-1} \\
    c_{n-1} & \cdots & \cdots & c_{2} & c_{1} & c_0
  \rightmatrix
\end{equation}

\noindent
Because the circulant matrix $\Cmat$ is fully determined by the sequence $\{c_h\}_{h \in \Pset_0}$, it can be compactly represented in memory using only $n$ values instead of $n^2$.
Circulant matrices can also be characterized by noting that the $(k,j)$ entry of $\Cmat$, $\leftmat \Cmat \rightmat_{j,k}$ is given by
\begin{equation}
  \leftmat \Cmat \rightmat_{j,k} = c_{\left(k-j\right) \mod n} \enspace.
\end{equation}
% We can also denote the sequence $\{c_h\}$ as a vector $\cvec$ such that $\forall i, \cvec_i = c_i$. The circulant matrices can then be expressed with $\Cmat = \circulant(\cvec)$.


\todotext{here}
In numerical analysis, circulant matrices are important because they are diagonalized by a discrete Fourier transform meaning that the columns of the Fourier matrix are the eigenvectors of every circulant matrix. 

Circulant matrices exhibit several interesting properties which are based their link with the Discrete Fourier Transform (DFT).
% on the fact that they can be diagonalized by the matrix expansion of the Discrete Fourier Transform (DFT).

Circulant matrices exhibit several interesting properties  

Circulant matrices arise, for example, in applications involving the discrete Fourier transform (DFT) and the study of cyclic codes for error correction.

Recall that the Discrete Fourier Transform converts a finite sequence into a complex-valued sequence of frequency.
It is possible to expand the discrete Fourier transform as a transformation matrix as follows:
\begin{definition}[Fourier Matrix]
  The Fourier matrix of order $n$ is defined as follows:
  \begin{equation} \label{definition:ch2-fourier_matrix}
    \Umat_n = 
    \leftmatrix
      1      & 1           & 1              & \cdots & 1           \\
      1      & z_n       & z_n^2        & \cdots & z_n^{n-1} \\
      1      & z_n^2     & z_n^4        & \cdots & z_n^{2(n-1)} \\
      \vdots & \vdots      & \vdots         &        & \vdots      \\
      1      & z_n^{n-1} & z_n^{2(n-1)} & \cdots & z_n^{(n-1)(n-1)}
    \rightmatrix
  \end{equation}
  where $z_n = e^{-\frac{2\pi\ci}{n}}$ is an $n^{\text{th}}$ root of unity.
\end{definition}

\noindent


Circulant matrices exhibit several interesting properties which are based their link with the Discrete Fourier Transform (DFT).
% on the fact that they can be diagonalized by the matrix expansion of the Discrete Fourier Transform (DFT).

In numerical analysis, circulant matrices are important because they are diagonalized by a discrete Fourier transform meaning that the columns of the Fourier matrix are the eigenvectors of every circulant matrix. 
k-


Circulant matrices exhibit several interesting properties which are based on the fact that they can be diagonalized by the matrix expansion of the Discrete Fourier Transform (DFT).



\begin{theorem}[Theorem 3.1 of \citet{gray2006toeplitz}] \label{theorem:ch2-diagonalization_circulant_matrix}
  Every $n \times n$ circulant matrix $\Cmat$, has eigenvectors  
  \begin{equation}
    \yvec^{(k)} = \frac{1}{\sqrt{n}} \leftmatrix 1, e^{-\frac{2 \pi \ci k}{n}}, \dots, e^{-\frac{2 \pi \ci k(n-1)}{n}} \rightmatrix^\top
  \end{equation}
  and corresponding eigenvalues
  \begin{equation}
    \psi_k = \sum_{j=0}^{n-1} c_j e^{-\frac{2 \pi \ci}{n} jk}
  \end{equation}
  The circulant matrix $\Cmat$ can be expressed in the form 
  \begin{equation} \label{equation:ch2-diagonalization_circulant_matrix}
    \Cmat = \frac{1}{n} \Umat_n^* \diag(\Umat_n \cvec) \Umat_n
  \end{equation}
  where $\cvec$ is a vector based on the scalars $\{c_h\}_{h \in \Pset_0}$.
  In particular all circulant matrices share the same eigenvectors, the same matrix $\Umat_n$ works for all $n \times n$ circulant matrices.
  Also, circulant matrices commute and are closed under sum and product.
\end{theorem}


The matrix-vector product between the Fourier matrix above and a vector is equivalent to applying the Discrete Fourier Transform on the vector.
This matrix-vector product can be efficiently computed with the \emph{Fast Fourier transform} (FFT) algorithm~\cite{cooley1965algorithm}.
The complexity is reduced from from $\bigO(n^2)$ to $\bigO(n \log n)$.


\noindent
Based on \Cref{theorem:ch2-diagonalization_circulant_matrix} and on the \emph{Fast Fourier transform} (FFT) algorithm, the matrix-vector product between a circulant matrix $\Cmat$ and a vector $\xvec$ can be reduced from $\bigO(n^2)$ to $\bigO(n \log n)$.
If we denote the Fast Fourier transform algorithm by \textbf{FFT} and the inverse by \textbf{IFFT}, the pseudocode for the matrix-product between a circulant matrix and a vector is as follows:
\begin{algorithm}[h]
  \begin{algorithmic}[1]
    \Procedure{CIRCMUL}{$\cvec, \xvec$} \Comment{first column of the circulant matrix $\Cmat$, vector $\xvec$}
      \State $\tilde{\xvec} \gets \textbf{FFT}(\xvec)$
      \State $\tilde{\cvec} \gets \textbf{FFT}(\cvec)$
      \State $\yvec \gets \textbf{IFFT}(\tilde{\xvec} \cdot \tilde{\cvec})$ \Comment{element-wise vector-vector product}
      \State \textbf{return} $\yvec$ \Comment{return the result of the product $\Cmat \xvec$}
    \EndProcedure
  \end{algorithmic}
  \caption{Matrix-vector product with a circulant matrix}
  \label{algorithm:ch2-matrix_vector_product_circulant_matrix}
\end{algorithm}

An important object based on circulant matrices is called \emph{$f$-circulant matrix} and is defined as follows:
\begin{definition}[$f$-circulant matrix] \label{definition:ch2-f_circulant_matrix}
  Given a vector $\xvec$, the $f$-circulant matrix, $\Zmat_f(\xvec)$, is defined as follows:
  \begin{equation}
    \Zmat_f(\xvec) \triangleq
    \leftmatrix
      \xvec_0 & $f$ \xvec_{n-1} & $f$ \xvec_{n-2} & \cdots & \cdots & $f$ \xvec_{1} \\
      \xvec_{1} & \xvec_0 & $f$ \xvec_{n-1} & \ddots & & \vdots \\
      \xvec_{2} & \xvec_{1} & \ddots & \ddots & \ddots & \vdots \\ 
      \vdots & \ddots & \ddots & \ddots & $f$ \xvec_{n-1} & $f$ \xvec_{n-2} \\
      \vdots & & \ddots & \xvec_{1} & \xvec_{0} & $f$ \xvec_{n-1} \\
      \xvec_{n-1} & \cdots & \cdots & \xvec_{2} & \xvec_{1} & \xvec_0
    \rightmatrix
  \end{equation}
  \removespace
\end{definition}
\noindent
% An $f$-unit-circulant, denoted $\Zmat_f$, is a $f$-circulant matrix defined by the vector $\left(0, 1, \dots, 0 \right)$:
An $f$-unit-circulant, $\Zmat_f$, defined by the vector $\left(0, 1, \dots, 0 \right)$ writes:
\begin{equation}
  \Zmat_f = \circulant \big(e^{(1)} \big) + (f - 1) \evec^{(0)} \evec^{(n-1)\top}
\end{equation}
\noindent
Thus, in matrix form, we have:
\begin{equation}
  \Zmat_f = 
    \leftmatrix
      0      & 0      & 0      & \cdots & \cdots & f      \\
      1      & 0      & 0      & \ddots &        & \vdots \\
      0      & 1      & \ddots & \ddots & \ddots & \vdots \\ 
      \vdots & \ddots & \ddots & \ddots & 0      & 0      \\
      \vdots &        & \ddots & 1      & 0      & 0      \\
      0      & \cdots & \cdots & 0      & 1      & 0
    \rightmatrix
\end{equation}

\noindent
As stated by~\citet{sindhwani2015structured}, the $f$-unit-circulant matrix performs a downward shift-and-scale transformation on a vector.
More precisely, the matrix-vector product $\Zmat_f \xvec$ makes a circular shift on the elements and scales the last element of the vector resulting in $\Zmat_f \xvec = \leftmat f \xvec_{n-1}, \xvec_0, \dots, \xvec_{n-2} \rightmat^\top$.
$f$-unit-circulant matrices are one of the building blocks of \emph{low displacement rank operator} presented in \Cref{subsection:ch2-building_compact_neural_networks_with_structured_matrices}.


% The f-unit-circulant matrix is associated with a basic downward shift-and-scale transformation, i.e.,
% the matrix-vector product Zfv shifts the elements of the column vector v “downwards”, and scales
% and brings the last element vn to the “top”, resulting in [fvn, v1, . . . vn−1] T .
% It has several basic
% algebraic properties (see Proposition 1.1 [1]) that are crucial for the results stated in this section



% the complexity of the matrix-vector product between a circulant matrix $\Cmat$ and a vector $\xvec$ can be reduced from $\bigO(n^2)$ to $\bigO(n \log n)$ with the \emph{Fast Fourier transform} (FFT) algorithm~\cite{cooley1965algorithm}.

% More precisely, the eigenvalues $\psi_k$ of the matrix $\Cmat$ correspond to the discrete Fourier transform of the vector characteristic $\cvec$:
% and the eigenvectors $\yvec^{(k)}$ can be expressed as follows:
% \begin{equation}
%   \yvec^{(k)} = \frac{1}{\sqrt{n}} \leftmatrix 1, e^{-\frac{2 \pi \ci k}{n}}, \dots, e^{-\frac{2 \pi \ci k(n-1)}{n}} \rightmatrix^\top
% \end{equation}
% In matrix form, we can diagonalize the matrix $\Cmat$ as follows:
% \begin{equation}
%   \Cmat = \frac{1}{n} \Umat_n^* \diag(\Umat_n \cvec) \Umat_n
% \end{equation}
% where $\Umat_n = \leftmatrix e^{-\frac{2 \pi \ci jk}{n}} \rightmatrix_{j,k = 0}^{n-1}$ is the DFT matrix and $\cvec$ is vector based of the scalars $\{c_h\}_{h \in \Pset_0}$.

% The following Theorem summarize the propreties for circulant matrices derived above.







%%%%%%%%%%%%%%%%%%%%%%%%%%%%%%%%%%%%%%%%%%%%%%%%%%%%%%%%%%%%%%%%%%%%%%%%%%%%%%%
\subsection{Toeplitz Matrices}
\label{subsection:ch2-toeplitz_matrices}
%%%%%%%%%%%%%%%%%%%%%%%%%%%%%%%%%%%%%%%%%%%%%%%%%%%%%%%%%%%%%%%%%%%%%%%%%%%%%%%

\todotext{citer Toeplitz application}
A Toeplitz matrix is a matrix in which each descending diagonal from left to right is constant.
\todotext{set P}
Let $\Pset = \{-n+1, \dots, n-1\}$, an $n\times n$ Toeplitz matrix $\Amat$ is fully determined by a two-sided sequence of scalars $\{a_h\}_{h \in \Pset}$ as follows:
\begin{equation}
  \Amat =
  \leftmatrix
    a_0      & a_1    & a_{2}  & \cdots & \cdots & a_{n-1} \\
    a_{-1}   & a_0    & a_{1}  & \ddots &        & \vdots  \\
    a_{-2}   & a_{-1} & \ddots & \ddots & \ddots & \vdots  \\
    \vdots   & \ddots & \ddots & \ddots & a_{1}  & a_2     \\
    \vdots   &        & \ddots & a_{-1} & a_{0}  & a_1     \\
    a_{-n+1} & \cdots & \cdots & a_{-2} & a_{-1} & a_0
  \rightmatrix
\end{equation}
\noindent
Because the Toeplitz matrix $\Amat$ is fully determined by the sequence $\{a_h\}_{h \in \Pset}$, it can be compactly represented in memory using only $2n-1$ values instead of $n^2$ values required for arbitrary matrices.
Toeplitz matrices can also be characterized by noting that the $(k,j)$ entry of $\Amat$, $\leftmat \Amat \rightmat_{j,k}$ is given by
\begin{equation}
  \leftmat \Amat \rightmat_{j,k} = a_{k-j} \enspace.
\end{equation}

A standard tool for manipulating Toeplitz matrices is the use of Fourier analysis.
Let $\{a_h\}_{h \in \Pset}$ be the sequence of coefficients of the Toeplitz matrix $\Amat \in \Rbb^{n\times n}$.
The complex-valued function 
\begin{equation}
  f(\omega) = \sum_{h \in \Pset} a_h e^{\ci h \omega}
\end{equation}
is the \emph{inverse Fourier transform} of the sequence $\{a_h\}_{h \in \Pset}$ with $\omega \in \Rbb$.
From this function, one can recover the sequence $\{a_h\}_{h \in \Pset}$ using the standard Fourier transform:
\begin{equation}
  a_h = \frac{1}{2\pi} \int_0^{2\pi} e^{-\ci h \omega} f(\omega) d\omega \enspace.
\end{equation}

\noindent
From there, we can define an operator $\Tmat$ mapping integrable functions to matrices:
\begin{equation} \label{equation:ch2-toeplitz_operator}
  \Tmat_n(g) \triangleq \leftmat\frac{1}{2\pi}\int_{0}^{2\pi}e^{-\ci(i-j)\omega}g(\omega)\,d\omega\rightmat_{i,j \in \Pset_0} \enspace.
\end{equation}
where $\Pset_0 = \{0, \ldots, n-1\}$.
\todotext{check english}
In the following of this thesis, when it is clear from context, we will write $\Tmat(g)$ instead of $\Tmat_n(g)$.

\todotext{remove}
We can show that if $f$ is the inverse Fourier transform of $\{a_h\}_{h \in \Pset}$, then $\Tmat_n(f)$ is equal to $\Amat$.
\begingroup
\allowdisplaybreaks
  \begin{align}
      \leftmat \Tmat_n(f) \rightmat_{i, j} &= \frac{1}{2\pi} \int_0^{2\pi} e^{-\ci (i-j)\omega} f(\omega) \,d\omega  \\
      &= \frac{1}{2\pi} \int_0^{2\pi} e^{-\ci (i-j) \omega} \sum_{h \in N} a_h e^{\ci h \omega} \,d\omega  \\
      &= \frac{1}{2\pi} \int_0^{2\pi} \sum_{h \in N} a_h e^{\ci (j - i + h) \omega} \,d\omega  \\
      &= \sum_{h \in N} a_h \frac{1}{2\pi} \int_0^{2\pi} e^{\ci (j - i + h) \omega} \,d\omega 
      = a_{j-i} .
  \end{align}
\endgroup
Because:
\begin{equation}
  \frac{1}{2\pi} \int_0^{2\pi} e^{\ci k \omega} \,d\omega = 
  \begin{cases}
    1 \quad \text{if}\ k = 0, \\
    0 \quad \text{if}\ k \neq 0
  \end{cases}
\end{equation}

% Also, if $F$ is the inverse Fourier transform of $\{\Bmat_h\}_{h \in \Pset}$ as defined above, then the integral in \Cref{equation:expression_toeplitz_matrix} is matrix-valued, and thus $\Tmat(F) \in \Rbb^{mn \times mn}$ is the block matrix $\Bmat$.

% The trigonometric polynomial that \emph{generates} the Toeplitz matrix $\Amat$ can be defined as follows:
% \begin{equation}
%   f_{\Amat}(\omega) \triangleq \sum_{h \in N} a_h e^{\ci h \omega}
% \end{equation}
% The function $f_{\Amat}$ is said to be the \emph{generating function} of $\Amat$.
% To recover the Toeplitz matrix from its generating function, we have the following operator:
% \begin{equation}
%   \leftmat \Tmat(f) \rightmat_{i, j} \triangleq \frac{1}{2\pi} \int_0^{2\pi} e^{-\ci (i - j)\omega} f(\omega) \,d\omega .
%   \label{equation:ch2-toeplitz_operator}
% \end{equation}
